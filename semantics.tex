\section{Semantics}

\begin{definition}[Semantic of FO-LTL]\label{def:fosemantic}
Let $M$ be a counterpart model, $\Gamma$ a context, $\pi = w_0 \overset{cr_0}{\rightsquigarrow} w_1
\overset{cr_1}{\rightsquigarrow} \ldots$, an infinite sequence of accessible worlds, and $\sigma$ a substitution with
context $\Gamma$ and with world $w_0$.  The validity of a formula-in-context $\phi$, denoted
$\sem{\pi}{\sigma}{\Gamma}{\phi}$, is defined inductively as follows:

\begin{align*}
  \pi, \sigma &\vDash_\Gamma tt \\
  \pi, \sigma &\vDash_\Gamma \epsilon_\tau = \eta_\tau &&\text{iff } \sigma(\epsilon) = \sigma(\eta) \\
  \pi, \sigma &\vDash_\Gamma \neg\phi &&\text{iff } \semn{\pi}{\sigma}{\Gamma}{\phi} \\
  \pi, \sigma &\vDash_\Gamma \phi_1 \lor \phi_2
      &&\text{iff } \sem{\pi}{\sigma}{\Gamma}{\phi_1} \text{ or } \sem{\pi}{\sigma}{\Gamma}{\phi_2} \\
  \pi, \sigma &\vDash_\Gamma \existss{x_\tau}{\phi}
      &&\text{iff } \existss{v \in d(w_0)}{\sem{\pi}{\extend_x(\sigma, v)}{\Gamma, x}{\phi}}
        \text{ with } x \not\in \Gamma \\
  \pi, \sigma &\vDash_\Gamma \nextop{\phi}
      &&\text{iff } \sem{\suffix(\pi, 1)}{cr_0 \circ \sigma}{\Gamma}{\phi} \\
  \pi, \sigma &\vDash_\Gamma \until{\phi_1}{\phi_2}
      &&\text{iff } \sem{\pi}{\sigma}{\Gamma}{\phi_2} \\
      &&&\quad\text{ or } (\sem{\pi}{\sigma}{\Gamma}{\phi_1}
        \text { and } \sem{\suffix(\pi, 1)}{cr_0 \circ \sigma}{\Gamma}{\until{\phi_1}{\phi_2}})
\end{align*}
  where $\extend_x : (A \rightharpoonup B) \times B \to (A \cup \set{x} \rightharpoonup B)$ is defined as:
  \[
    \extend_x(\sigma, v)(y) = \begin{cases} v & \text{if } y = x \\ \sigma(y) & \text{otherwise}\end{cases}
  \]
  and $\suffix(w_0 \overset{cr_0}{\rightsquigarrow} w_1 \overset{cr_1}{\rightsquigarrow} \ldots, n) =
  w_n \overset{cr_n}{\rightsquigarrow} w_{n+1} \overset{cr_{n+1}}{\rightsquigarrow} \ldots$
\end{definition}

The formula $tt$ holds for any possible sequence and assignment. The predicate $=$ models equality for typed terms, where
$\epsilon_\tau = \eta_\tau$ is valid if the evaluation is either undefined for both terms or they are equal according to
the standard notion of equality. The negation of a formula is satisfied if the formula without the negation is not
valid. The disjunction of two formulae is valid if either of them is valid. Existentially quantified formulae are valid
if the formula is valid in the context extended with the quantified variable. The sub-formula must be valid in the same
world and with the assignment extended to the context $\Gamma, x$ with a new value within the domain of the algebra,
thus behaving identically on all the variables in the original context $\Gamma$. Next we look at temporal operators.
Formulae containing the next operator are valid if the sub-formula is valid in the next accessible world in the sequence.
The sub-formula is evaluated with the assignement composed with the counterpart relation between the accessible worlds.
Formulae with the until operator, instead, are satisfied either if the post-condition is valid in the current world or 
if the pre-condition is valid and the whole formula is valid in the next accessible world in the sequence, the same as
for the next operator. An alternative, but equivalent formulation for the semantics of the until operator is the following:
\begin{align*}
  \pi, \sigma \vDash_\Gamma \until{\phi_1}{\phi_2}
      &\text{ iff } \existss{j \geq 0}{\sem{\suffix(\pi,j)}{cr_{j-1} \circ \cdots \circ cr_0 \circ \sigma}{\Gamma}{\phi_2}} \\
      &\text{ and } \foralls{0 \leq i < j}{\sem{\suffix(\pi,i)}{cr_{i-1} \circ \cdots \circ cr_0 \circ \sigma}{\Gamma}{\phi_1}}
\end{align*}

\begin{definition}\label{eq:validmodel}
  A formula $\phi \in \mathcal{F}_\Sigma$ is called \emph{valid for the counterpart model $M$}, denoted by $\vDash_M
  \phi$, if $\sem{\pi}{\sigma}{\Gamma}{\phi}$ for every context $\Gamma$, infinite sequence $\pi = w_0 \overset{cr_0}{\rightsquigarrow} w_1 \overset{cr_1}{\rightsquigarrow} \ldots$ and substitution $\sigma$ for the world $w_0$.
\end{definition}
\begin{definition}\label{eq:valid}
  A formula $\phi \in \mathcal{F}_\Sigma$ is called a \emph{consequence} of a set $F \subseteq \mathcal{F}_\Sigma$,
  denoted $F \vDash \phi$, if $\vDash_M \phi$ holds for every model $M$ with $\vDash_M \psi$ for all $\psi \in F$.

  If $F$ is empty, then the formula is called \emph{valid}, denoted by $\vDash \phi$.
\end{definition}

\section{Other operators}
Lets explore the semantics of the other common operators defined in the previous chapters.

\begin{align}
  \sem{\pi}{\sigma}{\Gamma}{\phi_1 \land \phi_2}
    &\Leftrightarrow \semn{\pi}{\sigma}{\Gamma}{\neg\phi_1 \lor \neg\phi_2} \notag \\
    &\Leftrightarrow \semn{\pi}{\sigma}{\Gamma}{\neg\phi_1} \text{ and } \semn{\pi}{\sigma}{\Gamma}{\neg\phi_2} \notag \\
    &\Leftrightarrow \sem{\pi}{\sigma}{\Gamma}{\phi_1} \text{ and } \sem{\pi}{\sigma}{\Gamma}{\phi_2} \label{eq:andsem} \\[1ex]
  \sem{\pi}{\sigma}{\Gamma}{\phi_1 \to \phi_2}
    &\Leftrightarrow \sem{\pi}{\sigma}{\Gamma}{\neg\phi_1 \lor \phi_2} \notag \\
    &\Leftrightarrow \semn{\pi}{\sigma}{\Gamma}{\neg\phi_1} \text{ or } \sem{\pi}{\sigma}{\Gamma}{\phi_2} \notag \\
    &\Leftrightarrow \sem{\pi}{\sigma}{\Gamma}{\phi_1} \text{ implies } \sem{\pi}{\sigma}{\Gamma}{\phi_2} \label{eq:impliessem} \\[1ex]
  \sem{\pi}{\sigma}{\Gamma}{\phi_1 \leftrightarrow \phi_2}
    &\Leftrightarrow (\sem{\pi}{\sigma}{\Gamma}{\phi_1 \to \phi_2}) \text{ and } (\sem{\pi}{\sigma}{\Gamma}{\phi_2 \to \phi_1}) \notag \\
    &\Leftrightarrow (\sem{\pi}{\sigma}{\Gamma}{\phi_1} \text{ implies } \sem{\pi}{\sigma}{\Gamma}{\phi_2}) \notag \\
        &\qquad\quad\text{and } (\sem{\pi}{\sigma}{\Gamma}{\phi_2} \text{ implies } \sem{\pi}{\sigma}{\Gamma}{\phi_1}) \notag \\
    &\Leftrightarrow \sem{\pi}{\sigma}{\Gamma}{\phi_1} \text{ iff } \sem{\pi}{\sigma}{\Gamma}{\phi_2} \label{eq:iffsem} \\[1ex]
  \sem{\pi}{\sigma}{\Gamma}{\foralls{x}{\phi}}
    &\Leftrightarrow \semn{\pi}{\sigma}{\Gamma}{\existss{x}{\neg\phi}} \notag \\
    &\Leftrightarrow \nexists v \in d(w_0) \ldotp \semn{\pi}{\extend_x(\sigma, v)}{\Gamma}{\phi} \notag \\
    &\Leftrightarrow \foralls{v \in d(w_0)}{\sem{\pi}{\extend_x(\sigma, v)}{\Gamma}{\phi}} \label{eq:forallsem} \\[1ex]
  \sem{\pi}{\sigma}{\Gamma}{\eventually{\phi}}
    &\Leftrightarrow \sem{\pi}{\sigma}{\Gamma}{\until{tt}{\phi}} \notag \\
    &\Leftrightarrow \sem{\pi}{\sigma}{\Gamma}{\phi} \text{ or }
        (\sem{\pi}{\sigma}{\Gamma}{tt} \text { and } \sem{\suffix(\pi,1)}{cr_0 \circ \sigma}{\Gamma}{\until{tt}{\phi}}) \notag \\
    &\Leftrightarrow \sem{\pi}{\sigma}{\Gamma}{\phi} \text{ or }
        \sem{\suffix(\pi,1)}{cr_0 \circ \sigma}{\Gamma}{\until{tt}{\phi}} \notag \\
    &\Leftrightarrow \sem{\pi}{\sigma}{\Gamma}{\phi} \lor \nextop{\eventually{\phi}} \label{eq:eventuallyexp} \\
    &\Leftrightarrow \existss{j \geq 0}{\sem{\suffix(\pi,j)}{cr_{j-1} \circ \cdots \circ cr_0 \circ \sigma}{\Gamma}{\phi}} \label{eq:eventuallysem} \\[1ex]
  \sem{\pi}{\sigma}{\Gamma}{\forever{\phi}}
    &\Leftrightarrow \semn{\pi}{\sigma}{\Gamma}{\eventually{\neg\phi}} \notag \\
    &\Leftrightarrow \sem{\pi}{\sigma}{\Gamma}{\phi} \text{ and }
        \semn{\suffix(\pi,i)}{cr_0 \circ \sigma}{\Gamma}{\eventually{\neg\phi}} \notag \\
    &\Leftrightarrow \sem{\pi}{\sigma}{\Gamma}{\phi \land \nextop{\forever{\phi}}} \label{eq:foreverexp} \\
    &\Leftrightarrow \foralls{j \geq 0}{\sem{\suffix(\pi,j)}{cr_{j-1} \circ \cdots \circ cr_0 \circ \sigma}{\Gamma}{\phi}} \label{eq:foreversem}
\end{align}

\section{Example of execution}

\begin{figure}
  \begin{center}
  \begin{tikzpicture}[scale=0.25,>=stealth']
    \MemoryLayout{A,B,C}
    \node[] () at (0, 0) {$w_0$};
    \begin{scope}[shift={(24cm,0cm)}]
      \MemoryLayout{A,B,C,D}
      \node[] () at (0, 0) {$w_1$};
    \end{scope}
    \begin{scope}[shift={(0cm,-10cm)}]
      \MemoryLayout{A,B,C}
      \node[] () at (0, 0) {$w_2$};
    \end{scope}
    \begin{scope}[shift={(24cm,-10cm)}]
      \MemoryLayout{A,B,E,C}
      \node[] () at (0, 0) {$w_3$};
    \end{scope}
    \begin{scope}[shift={(0cm,-20cm)}]
      \MemoryLayout{A,B,C,E}
      \node[] () at (0, 0) {$w_4$};
    \end{scope}
    \draw[->] (20cm,0) -- ++(2cm,0);
    \draw[->] (20cm,-10cm) -- ++(2cm,0);
    \draw[->] (9cm,-4cm) -- ++(0,-2cm);
    \draw[->] (33cm,-4cm) -- ++(0,-2cm);
    \draw[->] (9cm,-14cm) -- ++(0,-2cm);
    \draw[->] (28cm,-14cm) -- ++(-2cm,-2cm);
    % FIXME: draw self-loop for w_4
  \end{tikzpicture}
  \end{center}
  \caption{Example of counterpart model for a memory layout signature.}
  \label{fig:exmemsignature}
\end{figure}

Here we will provide an example of semantic evaluation of a formula, using some of the predicates introduced earlier in the chapter.  Assume we are evaluating according to the counterpart model visually represented by~\Cref{fig:exmemsignature}, with the signature being the idealised memory model in~\Cref{sec:algebra} and the counterpart relations acting as the identity on the objects sort, and on memory location sort as visually represented. 

Lets evaluate the formula $\phi_1 \equiv \existss{x_{\tau_o}}{\existss{l_{\tau_l}}{\rho(x) = l \land \nextop{(\rho(x) \neq l)}}}$, intuitively it identifies worlds where there is at least one object that is moved in memory after one step of computation.
Because this formula has no free variables it will be evaluated in the empty context, thus the only admissible substitution is the empty substitution, which we will denote with $\bot$.

First, by virtue of the definition of the existential quantifier, we need to evaluate the conjunction in the empty context extended by the two quantified variables, $\sem{\pi}{\sigma}{x,l}{\rho(x) = l \land \nextop{(\rho(x) \neq l)}}$.
The left-hand side of the conjunction is valid in all worlds when both variables are evaluated to be undefined or if both are defined and the location is exactly the one represented visually in the figure, e.g. in world $w_0$, with the assignement $\sigma$ such that $\sigma(x) = A$, $\sigma(l) = 0$ it follows that $\sem{\pi}{\sigma}{x,l}{\rho(x) = l}$, with $w_0$ the starting world of the sequence $\pi$.
For the right-hand side we need to check the accessibility relation of the model. The pairs for which the formula $\rho(x) \neq l$ is valid are exactly the complement of the left-hand side, therefore we need to check sequences $\pi$ that evolve into a world in the complement in one step.
All the assignements where only one of the two variables are undefined, evolve via the counterpart relations into one in the complement, however they are not interesting as they cannot satisfy the conjuction.
An example of pair that, instead, makes the whole conjuction valid is $\pi, \sigma$ with $\pi$ starting with the world
$w_3$ and $\sigma$ such that $\sigma(x) = E$,$\sigma(l) = 2$ or $\pi$ starting with the world $w_2$ and $\sigma$ such
that $\sigma(x) = C$,$\sigma(l) = 2$.
Readers may notice that there is still the assignement always undefined for which equalities are always trivially sastisfied, however, the existential quantifier requires that assignments are extended with defined values, thus ultimately matching our intuition for the semantic of the formula.
Thus to hold we must evaluate the formula in a sequence starting with either $w_1$, $w_2$ or $w_3$, however while for
$w_1$ and $w_3$ there is only a single valid sequence, respectively (with some notation abuse since the relation is
unique between each pair of worlds) $\pi1 = w_1w_3w_4w_4\ldots$ and $\pi2 = w_3w_4w_4\ldots$, for the world $w_2$ there
are two possible paths $\pi3 = w_2w_3w_4w_4\ldots$ and $\pi4 = w_2w_4w_4\ldots$ where only the former satisfies the
formula. Thus $\sem{\pi}{\sigma}{\emptyset}{\phi_1}$ with $\pi \in \set{\pi1, \pi2, \pi3}$ and $\sigma = \bot$.

If we wanted to recover the information about the specific object and/or location that moved with time, we need to evaluate open
formulae, thus we need to remove one or both existential quantification. However, since there is no quantification that
avoids undefined values that trivially satisfy equalities, we need a presence predicate, i.e. $\text{presence}(x_\tau) =
\existss{y_\tau}{x = y}$, which is satisfiable only if $y$ is equal to $x$ and $y$ must be defined, therefore
also $x$ must be defined.

Note that the formula is not valid in the model described by the figure since there are sequences which do not satisfy
the formula. A formula that instead is valid is $\phi_2 \equiv \existss{x_{\tau_o}}{\existss{l_{\tau_l}}{\forever{\rho(x)
= l}}}$, that is always satisfied since the object $A$ exists in every model and it never changes position.
