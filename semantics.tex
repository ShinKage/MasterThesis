\section{Semantics}

\begin{definition}[Semantic of FO-LTL]
Let $M$ be a counterpart model, $\Gamma$ a context and $(\sigma, w)$ a pair substitution-world in the context $\Gamma$.
The validity of a formula-in-context $\phi$, denoted $\sem{\sigma}{w}{\Gamma}{\phi}$, is defined inductively as follows:
\begin{align*}
  (\sigma, w) &\vDash_\Gamma tt \\
  (\sigma, w) &\vDash_\Gamma \epsilon_\tau = \eta_\tau &&\text{ iff } \sigma(\epsilon) = \sigma(\eta) \\
  (\sigma, w) &\vDash_\Gamma \neg\phi &&\text{ iff } \semn{\sigma}{w}{\Gamma}{\phi} \\
  (\sigma, w) &\vDash_\Gamma \phi_1 \lor \phi_2
      &&\text{ iff } \sem{\sigma}{w}{\Gamma}{\phi_1} \text{ or } \sem{\sigma}{w}{\Gamma}{\phi_2} \\
  (\sigma, w) &\vDash_\Gamma \existss{x_\tau}{\phi}
      &&\text{ iff } \existss{v \in d(w)}{\sem{\text{ext}_x(\sigma, v)}{w}{\Gamma, x}{\phi}} \\
  (\sigma, w) &\vDash_\Gamma \nextop{\phi}
      &&\text{ iff } \foralls{\worldcr{w}{cr}{w'}}{\sem{cr \circ \sigma}{w'}{\Gamma}{\phi}} \\
  (\sigma, w) &\vDash_\Gamma \until{\phi_1}{\phi_2}
      &&\text{ iff } \sem{\sigma}{w}{\Gamma}{\phi_2} \text{ or } (\sem{\sigma}{w}{\Gamma}{\phi_1}
          \text { and } \foralls{\worldcr{w}{cr}{w'}}{\sem{cr \circ \sigma}{w'}{\Gamma}{\until{\phi_1}{\phi_2}}}) \\
\end{align*}
  Where $\text{ext}_x : (\Gamma \rightharpoonup d(w)) \times d(w) \to (\Gamma \cup \set{x} \rightharpoonup d(w))$ with
  $x \not\in \Gamma$, defined as:
  \[
    \text{ext}_x(\sigma, v)(y) = \begin{cases} v & \text{if } y = x \\ \sigma(y) & \text{otherwise}\end{cases}
  \]
\end{definition}

The formula $tt$ holds for any possible pair assignement-world. The predicate $=$ models equality for typed terms, where
$\epsilon_\tau = \eta_\tau$ is valid if the evaluation is either undefined for both terms or they are equal according to
the standard notion of equality. The negation of a formula is satisfied if the formula without the negation is not
valid. The disjunction of two formulae is valid if either of them is valid. Existentially quantified formulae are valid
if the formula is valid in the context extended with the quantified variable. The sub-formula must be valid in the same
world and with the assignment extended to the context $\Gamma, x$ with a new value within the domain of the algebra,
thus behaving identically on all the variables in the original context $\Gamma$. Next we look at temporal operators.
Formulae containing the next operator are valid if the sub-formula is valid in all the world accessible from $w$. The
sub-formula is evaluated with the assignement composed with the counterpart relation between the accessible worlds.
Formulae with the until operator, instead, are satisfied either if the post-condition is valid in the current world or 
if the pre-condition is valid and the whole formula is valid in all the accessible world, the same as for the next
operator.

\begin{definition}
  A formula $\phi \in \mathcal{F}_\Sigma$ is called \emph{valid for the counterpart model $M$}, denoted by $\vDash_M
  \phi$, if $\sem{\sigma}{w}{\Gamma}{\phi}$ for every context $\Gamma$, world $w$ and substitution $\sigma$.
\end{definition}
\begin{definition}
  A formula $\phi \in \mathcal{F}_\Sigma$ is called a \emph{consequence} of a set $F \subseteq \mathcal{F}_\Sigma$,
  denoted $F \vDash \phi$, if $\vDash_M \phi$ holds for every model $M$ with $\vDash_M \psi$ for all $\psi \in F$.
\begin{definition}
\end{definition}
  A formula $\phi \in \mathcal{F}_\Sigma$ is called \emph{valid}, denoted by $\vDash \phi$, if $\emptyset \vDash \phi$
  holds.
\end{definition}

\section{Other operators}
Let's explore the semantics of the other common operators defined in the previous chapters.

\begin{align}
  \sem{\sigma}{w}{\Gamma}{\phi_1 \land \phi_2}
    &\Leftrightarrow \semn{\sigma}{w}{\Gamma}{\neg\phi_1 \lor \neg\phi_2} \notag \\
    &\Leftrightarrow \semn{\sigma}{w}{\Gamma}{\neg\phi_1} \text{ and } \semn{\sigma}{w}{\Gamma}{\neg\phi_2} \notag \\
    &\Leftrightarrow \sem{\sigma}{w}{\Gamma}{\phi_1} \text{ and } \sem{\sigma}{w}{\Gamma}{\phi_2} \label{eq:andsem} \\[1ex]
  \sem{\sigma}{w}{\Gamma}{\phi_1 \to \phi_2}
    &\Leftrightarrow \sem{\sigma}{w}{\Gamma}{\neg\phi_1 \lor \phi_2} \notag \\
    &\Leftrightarrow \semn{\sigma}{w}{\Gamma}{\neg\phi_1} \text{ or } \sem{\sigma}{w}{\Gamma}{\phi_2} \notag \\
    &\Leftrightarrow \sem{\sigma}{w}{\Gamma}{\phi_1} \text{ implies } \sem{\sigma}{w}{\Gamma}{\phi_2} \label{eq:impliessem} \\[1ex]
  \sem{\sigma}{w}{\Gamma}{\phi_1 \leftrightarrow \phi_2}
    &\Leftrightarrow (\sem{\sigma}{w}{\Gamma}{\phi_1 \to \phi_2}) \text{ and } (\sem{\sigma}{w}{\Gamma}{\phi_2 \to \phi_1}) \notag \\
    &\Leftrightarrow (\sem{\sigma}{w}{\Gamma}{\phi_1} \text{ implies } \sem{\sigma}{w}{\Gamma}{\phi_2}) \notag \\
        &\qquad\quad\text{and } (\sem{\sigma}{w}{\Gamma}{\phi_2} \text{ implies } \sem{\sigma}{w}{\Gamma}{\phi_1}) \notag \\
    &\Leftrightarrow \sem{\sigma}{w}{\Gamma}{\phi_1} \text{ iff } \sem{\sigma}{w}{\Gamma}{\phi_2} \label{eq:iffsem} \\[1ex]
  \sem{\sigma}{w}{\Gamma}{\foralls{x}{\phi}}
    &\Leftrightarrow \semn{\sigma}{w}{\Gamma}{\existss{x}{\neg\phi}} \notag \\
    &\Leftrightarrow \nexists v \in d(w) \ldotp \semn{\sigma[v/x]}{w}{\Gamma}{\phi} \notag \\
    &\Leftrightarrow \foralls{v \in d(w)}{\sem{\sigma[v/x]}{w}{\Gamma}{\phi}} \label{eq:forallsem} \\[1ex]
  \sem{\sigma}{w}{\Gamma}{\eventually{\phi}}
    &\Leftrightarrow \sem{\sigma}{w}{\Gamma}{\until{tt}{\phi}} \notag \\
    &\Leftrightarrow \sem{\sigma}{w}{\Gamma}{\phi} \text{ or }
        (\sem{\sigma}{w}{\Gamma}{tt} \text { and } \foralls{\worldcr{w}{cr}{w'}}{\sem{cr \circ \sigma}{w'}{\Gamma}{\until{tt}{\phi}}}) \notag \\
    &\Leftrightarrow \sem{\sigma}{w}{\Gamma}{\phi} \text{ or }
        \foralls{\worldcr{w}{cr}{w'}}{\sem{cr \circ \sigma}{w'}{\Gamma}{\until{tt}{\phi}}} \notag \\
    &\Leftrightarrow \sem{\sigma}{w}{\Gamma}{\phi} \text{ or } \sem{\sigma}{w}{\Gamma}{\nextop{\eventually{\phi}}} \notag \\
    &\Leftrightarrow \sem{\sigma}{w}{\Gamma}{\phi} \lor \nextop{\eventually{\phi}} \label{eq:eventuallysem} \\[1ex]
  \sem{\sigma}{w}{\Gamma}{\forever{\phi}}
    &\Leftrightarrow \sem{\sigma}{w}{\Gamma}{\neg\eventually{\neg\phi}} \notag \\
    &\Leftrightarrow \sem{\sigma}{w}{\Gamma}{\sem{\sigma}{w}{\Gamma}{\phi} \text{ and }
        \neg\foralls{\worldcr{w}{cr}{w'}}{\sem{cr \circ \sigma}{w'}{\Gamma}{\neg\phi}}} \notag \\
    &\Leftrightarrow \sem{\sigma}{w}{\Gamma}{\sem{\sigma}{w}{\Gamma}{\phi} \text{ and }
        \existss{\worldcr{w}{cr}{w'}}{\sem{cr \circ \sigma}{w'}{\Gamma}{\phi}}} \notag \\
    &\Leftarrow \sem{\sigma}{w}{\Gamma}{\phi \land \nextop{\forever{\phi}}} \label{eq:foreversem}
\end{align}

\section{Example of execution}

\begin{figure}
  \begin{center}
  \begin{tikzpicture}[scale=0.25,>=stealth']
    \MemoryLayout{A,B,C}
    \node[] () at (0, 0) {$w_0$};
    \begin{scope}[shift={(24cm,0cm)}]
      \MemoryLayout{A,B,C,D}
      \node[] () at (0, 0) {$w_1$};
    \end{scope}
    \begin{scope}[shift={(0cm,-10cm)}]
      \MemoryLayout{A,B,C}
      \node[] () at (0, 0) {$w_2$};
    \end{scope}
    \begin{scope}[shift={(24cm,-10cm)}]
      \MemoryLayout{A,B,E,C}
      \node[] () at (0, 0) {$w_3$};
    \end{scope}
    \begin{scope}[shift={(0cm,-20cm)}]
      \MemoryLayout{A,B,C,E}
      \node[] () at (0, 0) {$w_4$};
    \end{scope}
    \draw[->] (20cm,0) -- ++(2cm,0);
    \draw[->] (20cm,-10cm) -- ++(2cm,0);
    \draw[->] (9cm,-4cm) -- ++(0,-2cm);
    \draw[->] (33cm,-4cm) -- ++(0,-2cm);
    \draw[->] (9cm,-14cm) -- ++(0,-2cm);
    \draw[->] (28cm,-14cm) -- ++(-2cm,-2cm);
  \end{tikzpicture}
  \end{center}
  \caption{Example of counterpart model for a memory layout signature.}
  \label{fig:exmemsignature}
\end{figure}

Here we will provide the evaluation of some formulae, using some of the predicates introduced earlier in the chapter.
Assume we are evaluating according to the counterpart model visually represented by~\Cref{fig:exmemsignature}, with
the signature being the idealised memory model in~\Cref{sec:algebra} and the counterpart relations acting as the
identity on the objects sort, and on memory location sort as visually represented.
Let's evaluate the formula $\phi_1 \equiv \existss{x_{\tau_o}}{\existss{l_{\tau_l}}{\rho(x) = l \land \nextop{(\rho(x)
\neq l)}}}$, intuitively it identifies worlds where there is at least one object that is moved in memory after one step
of computation. Because this formula has no free variables it will be evaluated in the empty context, thus the only
admissible substitution is the empty substitution, which we will denote with $\bot$.
First, by virtue of the definition of the existential quantifier, we need to evaluate the conjunction in the empty
context extended by the two quantified variables, $\sem{\sigma}{w}{x,l}{\rho(x) = l \land \nextop{(\rho(x) \neq l)}}$.
The left-hand side of the conjunction is valid in all worlds when both variables are evaluated to be undefined or if
both are defined and the location is exactly the one represented visually in the figure, e.g. in world $w_0$, with
the assignement $\sigma$ such that $\sigma(x) = A$, $\sigma(l) = 0$ it follows that $\sem{\sigma}{w}{x,l}{\rho(x) = l}$.
For the right-hand side we need to check the accessibility relation of the model. The pairs for which the formula
$\rho(x) \neq l$ is valid are exactly the complement of the left-hand side, therefore we need to check which pair evolve
due to the counterpart relations into one of those, for each possible accessible world. All the assignements where only
one of the two variables are undefined, evolve via the counterpart relations into one in this set, however they are not
interesting as they cannot satisfy the conjuction. There are also pair for which the sub-formula inside the next
operator is valid in one trasition but not all of them, an example is $(\sigma, w_2)$ with $\sigma$ such that $\sigma(x) = C$,
$\sigma(l) = 2$. After applying the counterpart relation $\worldcr{w_2}{cr}{w_3}$, $(cr \circ \sigma)(l) = 3$ that
satisfies the inequality $\rho(C) \neq 3$, however with the counterpart relation $\worldcr{w_2}{cr}{w_4}$,
$(cr \circ sigma)(l) = 2$ does not satisfy the inequality, thus $\semn{\sigma}{w_2}{x,l}{\nextop{(\rho(x) \neq l)}}$.
An example of pair that, instead, makes the whole conjuction valid is $(\sigma, w_3)$ with $\sigma$ such that $\sigma(x)
= E$,$\sigma(l) = 2$. Readers may notice that there is still the assignement always undefined for which equalities are
always trivially sastisfied, however, the existential quantifier requires that assignments are extended with defined
values, thus ultimately matching our intuition for the semantic of the formula. As noted before, the final pairs must be
composed of empty substitution, therefore the only pairs for which the formula $\phi_1$ is valid are $(\bot, w_1)$ and
$(\bot, w_3)$.

If we wanted to recover the information about the specific object and/or location that moved with time, we need to evaluate open
formulae, thus we need to remove one or both existential quantification. However, since there is no quantification that
avoids undefined values that trivially satisfy equalities, we need a presence predicate, i.e. $\text{presence}(x_\tau) =
\existss{y_\tau}{x = y}$, which is satisfiable only if $y$ is equal to $x$ and $y$ must be defined, therefore
also $x$ must be defined.
