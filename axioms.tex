\section{Axioms}
The following definition will introduce a Hilbert-style axiomatic system for \ac{FOLTL}. The axioms and inference rules
follow the pre-existing tradition on finitary axiomatic systems for
\ac{FOLTL}~\cite{burgess_axioms_1982,xu_us-tense_1988}, with one notable and somewhat controversial missing axiom that
will be discussed in the following sections.

\begin{definition}[Axiomatic system]\label{def:axioms}
  The axiomatic system for FO-LTL contains the following axioms:
  \begin{enumerate}
    \item $\phi_1 \to (\phi_2 \to \phi_1)$;
    \item $(\phi_1 \to (\phi_2 \to \phi_3)) \to ((\phi_1 \to \phi_2) \to (\phi_1 \to \phi_3))$;
    \item $(\neg\phi_1 \to \neg\phi_2) \to (\phi_2 \to \phi_1)$;
    \item $\neg\nextop{\phi} \leftrightarrow \nextop{\neg\phi}$;
    \item $\nextop{(\phi_1 \to \phi_2)} \to (\nextop{\phi_1} \to \nextop{\phi_2})$;
    \item $\phi \to \nextop{\phi}$ if $\phi$ is rigid;
    \item $\until{\phi_1}{\phi_2} \leftrightarrow \phi_2 \lor (\phi_1 \land \nextop{(\until{\phi_1}{\phi_2})})$;
    \item $\until{\phi_1}{\phi_2} \to \eventually{\phi_2}$;
    \item $\phi[v/x] \to \existss{x}{\phi}$, with $x$ free in $\phi$ and $v$ a value in the domain;
    \item $x = x$;
    \item $x = y \to (\phi \to \phi[y/x])$.
  \end{enumerate}
  And the following inference rules:
  \begin{enumerate}
    \item[mp] $\phi_1, \phi_1 \to \phi_2 \vdash \phi_2$;
    \item[nex] $\phi \vdash \nextop{\phi}$;
    \item[ind] $\phi_1 \to \phi_3 \lor (\phi_2 \land \nextop{\phi_1}) \vdash \phi_1 \to \until{\phi_2}{\phi_3}$;
    \item[par] $\phi_1 \to \phi_2 \vdash (\existss{x}{\phi_1}) \to \phi_2$ with $x$ not free in $\phi_2$.
  \end{enumerate}
\end{definition}

The axioms can be approximately grouped as such: the axioms 1-3 are one of the many combination of axioms that express
classical propositional logic; axioms 4-6 describe the unary temporal operator next and the interaction with
propositional operators; axioms 7-8 describe the binary temporal operator until and its expansion rule; axiom 9
introduces first-order quantification; axioms 10-11 describe equality. Inference rules are also self-explanatory, with
(mp) and (par) derived from \ac{FOL}, \emph{nex} and \emph{ind} allows instead for the introduction of temporal operators.

With the following lemmas and theorems we will provide correctness results for this axiomatic system.

\begin{lemma}\label{lem:propaxioms}
  Let $\phi_1, \phi_2, \phi_3 \in \mathcal{F}_\Sigma$.
  \begin{enumerate}
    \item $\vDash \phi_1 \to (\phi_2 \to \phi_1)$;
    \item $\vDash (\phi_1 \to (\phi_2 \to \phi_3)) \to ((\phi_1 \to \phi_2) \to (\phi_1 \to \phi_3))$;
    \item $\vDash (\neg\phi_1 \to \neg\phi_2) \to (\phi_2 \to \phi_1)$.
  \end{enumerate}
\end{lemma}
\begin{proof}
  Let $M$ be a counterpart model and a context $\Gamma$, a sequence $\pi = w_0 \overset{cr_0}{\rightsquigarrow} w_1 \overset{cr_1}{\rightsquigarrow} w_2 \overset{cr_2}{\rightsquigarrow} \ldots$ and an assignment $\sigma$ over the world $w_0$:
  \begin{enumerate}
    \item From trivial applications of the semantic rules it follows that $\sem{\pi}{\sigma}{\Gamma}{\phi_1 \to (\phi_2
      \to \phi_1)} \Leftrightarrow \semn{\pi}{\sigma}{\Gamma}{\phi_1} \text{ or } \semn{\pi}{\sigma}{\Gamma}{\phi_2}
      \text{ or } \sem{\pi}{\sigma}{\Gamma}{\phi_1}$ which is a tautology by the law of excluded middle.

    \item By~\Cref{eq:iffsem} we show that the stronger formula $\vDash (\phi_1 \to (\phi_2 \to \phi_3)) \leftrightarrow ((\phi_1 \to \phi_2)
      \leftrightarrow (\phi_1 \to \phi_3))$. From the semantic rules:
      \[
        \begin{split}
          &\sem{\pi}{\sigma}{\Gamma}{\phi_1 \to (\phi_2 \to \phi_3)} \\
            &\quad\Leftrightarrow \semn{\pi}{\sigma}{\Gamma}{\phi_1} \text{ or } \semn{\pi}{\sigma}{\Gamma}{\phi_2} \text{ or }
              \sem{\pi}{\sigma}{\Gamma}{\phi_3} \\
            &\quad\Leftrightarrow (\sem{\pi}{\sigma}{\Gamma}{\phi_1} \text{ and } \semn{\pi}{\sigma}{\Gamma}{\phi_2}) \text{ or }
              \semn{\pi}{\sigma}{\Gamma}{\phi_1} \text{ or } \sem{\pi}{\sigma}{\Gamma}{\phi_3} \\
            &\quad\Leftrightarrow \sem{\pi}{\sigma}{\Gamma}{(\phi_1 \to \phi_2) \to (\phi_1 \to \phi_3)}
        \end{split}
      \]

    \item By~\Cref{eq:iffsem} we show the stronger formula $\vDash (\neg\phi_1 \to \neg\phi_2) \leftrightarrow (\phi_2 \to \phi_1)$.
      From the semantic rules:
      \[
        \begin{split}
          \sem{\pi}{\sigma}{\Gamma}{\neg\phi_1 \to \neg\phi_2}
            &\Leftrightarrow \sem{\pi}{\sigma}{\Gamma}{\phi_1} \text{ or } \semn{\pi}{\sigma}{\Gamma}{\phi_2} \\
            &\Leftrightarrow \sem{\pi}{\sigma}{\Gamma}{\phi_2 \to \phi_1}
        \end{split}
      \]
  \end{enumerate}
\end{proof}

\begin{lemma}\label{lem:negnextdist}
  Let $\phi \in \mathcal{F}_\Sigma$, $\vDash \neg\nextop{\phi} \leftrightarrow \nextop{\neg\phi}$.
\end{lemma}
\begin{proof}
  Let $M$ be a counterpart model and a context $\Gamma$, a sequence $\pi = w_0 \overset{cr_0}{\rightsquigarrow} w_1
  \overset{cr_1}{\rightsquigarrow} w_2 \overset{cr_2}{\rightsquigarrow} \ldots$ and an assignment $\sigma$ over the world $w_0$:
  \[
    \begin{split}
      \sem{\pi}{\sigma}{\Gamma}{\neg\nextop{\phi}}
        &\Leftrightarrow \semn{\suffix(\pi,1)}{cr_0 \circ \sigma}{\Gamma}{\phi} \\
        &\Leftrightarrow \sem{\suffix(\pi,1)}{cr_0 \circ \sigma}{\Gamma}{\neg\phi} \\
        &\Leftrightarrow \sem{\pi}{\sigma}{\Gamma}{\nextop{\neg\phi}}
    \end{split}
  \]
\end{proof}

\begin{lemma}\label{lem:impnextdist}
  Let $\phi_1, \phi_2 \in \mathcal{F}_\Sigma$, $\vDash \nextop{(\phi_1 \to \phi_2)} \to (\nextop{\phi_1} \to
  \nextop{\phi_2})$.
\end{lemma}
\begin{proof}
  Let $M$ be a counterpart model and a context $\Gamma$, a sequence $\pi = w_0 \overset{cr_0}{\rightsquigarrow} w_1
  \overset{cr_1}{\rightsquigarrow} w_2 \overset{cr_2}{\rightsquigarrow} \ldots$ and an assignment $\sigma$ over the world $w_0$:
  \[
    \begin{split}
      &\sem{\pi}{\sigma}{\Gamma}{\nextop{(\phi_1 \to \phi_2)}} \\
        &\quad\Leftrightarrow \sem{\suffix(\pi,1)}{cr_0 \circ \sigma}{\Gamma}{\phi_1 \to \phi_2} \\
        &\quad\Leftrightarrow \sem{\suffix(\pi,1)}{cr_0 \circ \sigma}{\Gamma}{\phi_1} \text{ implies }
            \sem{\suffix(\pi,1)}{cr_0 \circ \sigma}{\Gamma}{\phi_2} \\
        &\quad\Leftrightarrow \sem{\pi}{\sigma}{\Gamma}{\nextop{\phi_1} \to \nextop{\phi_2}}
    \end{split}
  \]
\end{proof}

\begin{lemma}\label{lem:nextintro}
  Let $\phi \in \mathcal{F}_\Sigma$ be a rigid formula, $\vDash \phi \to \nextop{\phi}$.
\end{lemma}
\begin{proof}
  Let $M$ be a counterpart model and a context $\Gamma$, a sequence $\pi = w_0 \overset{cr_0}{\rightsquigarrow} w_1
  \overset{cr_1}{\rightsquigarrow} w_2 \overset{cr_2}{\rightsquigarrow} \ldots$ and an assignment $\sigma$ over the
  world $w_0$.
  Let $\sem{\pi}{\sigma}{\Gamma}{\phi}$. By definition, $\sem{\pi}{\sigma}{\Gamma}{\nextop{\phi}}$ iff
  $\sem{\suffix(\pi,1)}{cr_0 \circ \sigma}{\Gamma}{\phi}$.
  Since $\phi$ is a rigid formula, it must hold irregardless of the particular counterpart relation that is applied to
  it, thus $\sem{\suffix(\pi,1)}{cr_0 \circ \sigma}{\Gamma}{\phi}$ must also hold.
\end{proof}

\begin{lemma}\label{lem:untilexp}
  Let $\phi_1, \phi_2 \in \mathcal{F}_\Sigma$, $\vDash \until{\phi_1}{\phi_2} \leftrightarrow \phi_2 \lor (\phi_1 \land
  \nextop{(\until{\phi_1}{\phi_2})})$.
\end{lemma}
\begin{proof}
  Follows trivially by the definition of the operators until and next.
\end{proof}

\begin{lemma}\label{lem:untileventually}
  Let $\phi_1, \phi_2 \in \mathcal{F}_\Sigma$, $\vDash \until{\phi_1}{\phi_2} \to \eventually{\phi_2}$.
\end{lemma}
\begin{proof}
  Let $M$ be a counterpart model and a context $\Gamma$, a sequence $\pi = w_0 \overset{cr_0}{\rightsquigarrow} w_1
  \overset{cr_1}{\rightsquigarrow} w_2 \overset{cr_2}{\rightsquigarrow} \ldots$ and an assignment $\sigma$ over the
  world $w_0$.
  By the definitions of the operators:
  \begin{align*}
    &\sem{\pi}{\sigma}{\Gamma}{\until{\phi_1}{\phi_2}} \\
    &\qquad\Leftrightarrow \existss{j \geq 0}{\sem{\suffix(\pi,j)}{cr_{j-1} \circ \cdots \circ cr_0 \circ \sigma}{\Gamma}{\phi_2}} \\
    &\qquad\qquad\text{ and } \foralls{0 \leq i < j}{\sem{\suffix(\pi,i)}{cr_{i-1} \circ \cdots \circ cr_0 \circ
    \sigma}{\Gamma}{\phi_1}} \\
    &\qquad\Rightarrow \existss{j \geq 0}{\sem{\suffix(\pi,j)}{cr_{j-1} \circ \cdots \circ cr_0 \circ \sigma}{\Gamma}{\phi_2}} \\
    &\qquad\Leftrightarrow \sem{\pi}{\sigma}{\Gamma}{\eventually{\phi_2}}
  \end{align*}
\end{proof}

\begin{lemma}\label{lem:exintro}
  Let $\phi \in \mathcal{F}_\Sigma$ with $x \in X$ free in $\phi$, $\vDash \phi[v/x] \to \existss{x}{\phi}$, if $v$ is a
  value in the domain.
\end{lemma}
\begin{proof}
  Let $M$ be a counterpart model and a context $\Gamma$, a sequence $\pi = w_0 \overset{cr_0}{\rightsquigarrow} w_1
  \overset{cr_1}{\rightsquigarrow} w_2 \overset{cr_2}{\rightsquigarrow} \ldots$ and an assignment $\sigma$ over the
  world $w_0$.
  By hypothesis $\sem{\pi}{\sigma}{\Gamma}{\phi[v/x]}$, with $v \in d(w_0)$, then it holds that
  $\sem{\pi}{\ext{x}{\sigma}{v}}{\Gamma,x}{\phi}$, therefore $\sem{\pi}{\sigma}{\Gamma}{\existss{x}{\phi}}$ holds.
\end{proof}

\begin{lemma}\label{lem:equality}
  Let $\phi \in \mathcal{F}_\Sigma$ and $x, y \in X$,
  \begin{enumerate}
    \item $\vDash x = x$;
    \item $\vDash x = y \to (\phi \to \phi[y/x])$.
  \end{enumerate}
\end{lemma}
\begin{proof}
  Let $M$ be a counterpart model and a context $\Gamma$, a sequence $\pi = w_0 \overset{cr_0}{\rightsquigarrow} w_1
  \overset{cr_1}{\rightsquigarrow} w_2 \overset{cr_2}{\rightsquigarrow} \ldots$ and an assignment $\sigma$ over the
  world $w_0$.
  \begin{enumerate}
    \item By definition, $\sem{\pi}{\sigma}{\Gamma}{x = x}$ holds if $\sigma(x) = \sigma(x)$, which is trivially true.
    \item By assumption it holds that $\sem{\pi}{\sigma}{\Gamma}{x = y}$, thus $\sigma(x) = \sigma(y)$. Assume
      $\sem{\pi}{\sigma}{\Gamma}{\phi}$ holds, then by the congruence property of equality it must also hold
      $\sem{\pi}{\sigma}{\Gamma}{\phi[y/x]}$.
  \end{enumerate}
\end{proof}

\begin{lemma}\label{lem:mp}
  If $F \vDash \phi_1$ and $F \vDash \phi_1 \to \phi_2$, then $F \vDash \phi_2$.
\end{lemma}
\begin{proof}
  Let $M$ be a counterpart model that satisfies $\vDash_M \psi$ for every $\psi \in F$.
  By definition $\sem{\pi}{\sigma}{\Gamma}{\phi_1 \to \phi_2} \Leftrightarrow \sem{\pi}{\sigma}{\Gamma}{\phi_1}
  \text{ implies } \sem{\pi}{\sigma}{\Gamma}{\phi_2}$. By assumption it holds $\sem{\pi}{\sigma}{\Gamma}{\phi_1}$, thus it must
  hold that $\sem{\pi}{\sigma}{\Gamma}{\phi_2}$.
\end{proof}
\begin{lemma}\label{lem:nex}
  If $F \vDash \phi$, then $F \vDash \nextop{\phi}$ and $F \vDash \forever{\phi}$.
\end{lemma}
\begin{proof}
  Let $M$ be a counterpart model that satisfies $\vDash_M \psi$ for every $\psi \in F$ and $\Gamma$.
  By assumption $\sem{\pi}{\sigma}{\Gamma}{\phi}$ holds for every pair sequence $\pi$ and assignment $\sigma$.
  In particular given a sequence $\pi = w_0 \overset{cr_0}{\rightsquigarrow} w_1 \overset{cr_1}{\rightsquigarrow}
  \ldots$ and an assignment $\sigma$, it must also hold $\sem{\suffix(\pi,i)}{cr_{i-1} \circ \cdots \circ cr_0 \circ
  \sigma}{\Gamma}{\phi}$ for every $i \geq 0$ thus $F \vDash \forever{\phi}$ and in particular with $i = 1$ thus $F
  \vDash \nextop{\phi}$.
\end{proof}
\begin{lemma}\label{lem:ind}
  If $F \vDash \phi_1 \to \phi_2$ and $F \vDash \phi_1 \to \nextop{\phi_1}$, then $F \vDash \phi_1 \to \forever{\phi_2}$.
\end{lemma}
\begin{proof}
  Let $M$ be a counterpart model that satisfies $\vDash_M \psi$ for every $\psi \in F$ and $\Gamma$.  If
  $\semn{\pi}{\sigma}{\Gamma}{\phi_1}$ than the conclusion trivially holds. Assume, instead,
  $\sem{\pi}{\sigma}{\Gamma}{\phi_1}$, with $\pi = w_0 \overset{cr_0}{\rightsquigarrow} w_1
  \overset{cr_1}{\rightsquigarrow} \ldots$.
  By expanding the formula we need to show that $\foralls{j \geq 0}{\sem{\suffix(\pi, j)}{cr_{i - 1} \circ \cdots cr_0 \circ
  \sigma}{\Gamma}{\phi_2}}$ holds.
  However, by hypothesis $\sem{\sigma}{w}{\Gamma}{\phi_2}$ and
  $\sem{\sigma}{w}{\Gamma}{\nextop{\phi_1}}$ are valid in every $(\sigma, w)$ such that
  $\sem{\sigma}{w}{\Gamma}{\phi_1}$ is valid, therefore $\phi_1$ is valid in every sequence described before and thus
  $\phi_2$ is valid in every sequence.
\end{proof}
\begin{lemma}\label{lem:par}
  If $F \vDash \phi_1 \to \phi_2$ and $x$ not free in $\phi_2$, then $F \vDash (\existss{x}{\phi_1}) \to \phi_2$.
\end{lemma}
\begin{proof}
  Let $M$ be a counterpart model that satisfies $\vDash_M \psi$ for every $\psi \in F$ and $\Gamma$.
  Assume $\semn{\pi}{\sigma}{\Gamma}{(\existss{x}{\phi_1}) \to \phi_2}$ for some sequence $\pi = w_0 \overset{cr_0}{\rightsquigarrow} w_1 \overset{cr_1}{\rightsquigarrow} \ldots$
  and assignment $\sigma$, it follows that $\sem{\pi}{\sigma}{\Gamma}{\existss{x}{\phi_1}}$ and $\semn{\sigma}{w}{\Gamma}{\phi_2}$.
  Then there is a $v \in d(w_0)$ such that $\sem{\pi}{\ext{x}{\sigma}{v}}{\Gamma, x}{\phi_1}$, by definition of existential
  quantifier. Since $\phi_2$ does not contain $x$ as a free variable, adding an evaluation for $x$ does not change the
  evaluation of $\phi_2$, i.e. $\semn{\pi}{\ext{x}{\sigma}{v}}{\Gamma, x}{\phi_2}$, therefore
  $\semn{\pi}{\ext{x}{\sigma}{v}}{\Gamma, x}{\phi_1 \to \phi_2}$, which is a contradiction to the hypothesis $\vDash_M
  \phi_1 \to \phi_2$.
\end{proof}

\begin{theorem}[Soundness]
  Let $\phi \in \mathcal{F}_\Sigma$ and $F \subseteq \mathcal{F}_\Sigma$, if $F \vdash \phi$ then $F \vDash \phi$.
\end{theorem}
\begin{proof}
  By induction on the derivation of $\phi$ from $F$:
  \begin{enumerate}
    \item if $\phi$ is an axiom: $F \vDash \phi$ is proven by~\Cref{lem:propaxioms} for axioms 1,2,3;
      by~\Cref{lem:negnextdist} for axiom 4; by~\Cref{lem:impnextdist} for axiom 5; by~\Cref{lem:nextintro} for axiom 6;
      by~\Cref{lem:untilexp} for axiom 7; by~\Cref{lem:untileventually} for axiom 8; by~\Cref{lem:exintro} for axiom 9;
      by~\Cref{lem:equality} for axioms 10, 11;
    \item if $\phi \in F$ then $F \vDash \phi$ holds trivially;
    \item if $\phi$ is the conclusion of a (mp) rule with premises $F \vdash \psi$ and $F \vdash \psi \to \phi$: by
      induction hypothesis we have $F \vDash \psi$ and $F \vDash \psi \to \phi$, hence $F \vDash \phi$ follows by
      \Cref{lem:mp};
    \item if $\phi$ is the conclusion of a (nex) rule with premises $F \vdash \psi$, thus $\phi \equiv \nextop{\psi}$: by
      induction hypothesis we have $F \vDash \psi$, hence $F \vDash \nextop{\psi}$ follows by \Cref{lem:nex};
    \item if $\phi$ is the conclusion of a (ind) rule with premises $F \vdash \psi_1 \to \psi_2$ and $F \vdash \psi_1
      \to \nextop{\psi_1}$, thus $\phi \equiv \psi_1 \to \forever{\psi_2}$: by induction hypothesis we have $F \vDash
      \psi_1 \to \psi_2$ and $F \vDash \psi_1 \to \nextop{\psi_2}$, hence $F \vDash \psi_1 \to \forever{\psi_2}$ follows
      by \Cref{lem:ind};
    \item if $\phi$ is the conclusion of a (par) rule with premises $F \vdash \psi_1 \to \psi_2$, thus $\phi \equiv
      \existss{x}{\psi_1} \to \psi_2$ with $x$ not free in $\psi_2$: by induction hypothesis we have $F \vDash \psi_1
      \to \psi_2$, hence $F \vDash \exists{x}{\psi_1} \to \psi_2$ follows by \Cref{lem:par}.
  \end{enumerate}
\end{proof}

\begin{example}\label{ex:alw}
  We show the derivation of some rules to show the capabilities of the deduction system.
  In the following derivation we will denote as (prop) all the rules that are tautology in propositional
  logic, since it's completely described by the first three axioms of the system.

  We will start with a variant of the (ind) rule, $\phi \to \nextop{\phi} \vdash \phi \to \forever{\phi}$, which we will call (ind1):

  \medskip
  \begin{tabularx}{300pt}{cXl}
    (1) & $\phi \to \nextop{\phi}$ & assumption \\
    (2) & $\phi \to \phi$ & (prop) \\
    (3) & $\phi \to \forever{\phi}$ & (ind), (1), (2)
  \end{tabularx}

  \medskip
  Next we show the (alw) rule, $\phi \vdash \forever{\phi}$, which will be useful later for the deduction theorems.

  \medskip
  \begin{tabularx}{300pt}{cXl}
    (1) & $\phi$ & assumption \\
    (2) & $\nextop{\phi}$ & (nex), (1) \\
    (3) & $\phi \to \nextop{\phi}$ & (axiom 1), (2) \\
    (4) & $\phi \to \forever{\phi}$ & (ind1), (3) \\
    (5) & $\forever{\phi}$ & (mp), (1), (4) \\
  \end{tabularx}

  \medskip
  Here we show the (for) rule, i.e. the expansion rule for the forever operator.

  \medskip
  \begin{tabularx}{300pt}{cXl}
    (1) & $(\neg\phi \lor \nextop{(\until{tt}{\neg\phi})}) \to \until{tt}{\neg\phi}$ & axiom 7 \\
    (2) & $\neg\until{tt}{\neg\phi} \to (\neg(\neg\phi \lor \nextop{(\until{tt}{\neg\phi})}))$ & (1), axiom 3 \\
    (3) & $\neg\until{tt}{\neg\phi} \to (\phi \land \neg\nextop{(\until{tt}{\neg\phi})})$ & (2), (prop) \\
    (4) & $\forever{\phi} \to (\phi \land \forever{\phi})$ & (3), definition \\
  \end{tabularx}
\end{example}

As in the propositional calculus and \ac{FOL}, in the next pair of theorems we will investigate the connection between
implication and the derivability of a formula.

\begin{theorem}[Deduction Theorem]
  Let $\phi_1, \phi_2 \in \mathcal{F}_\Sigma$ and $F \subseteq \mathcal{F}_\Sigma$. If $F \cup \set{\phi_1} \vdash
  \phi_2$ and this deriviation of $\phi_2$ does not contains application of the rule (par) for a variable occurring free
  in $\phi_1$, then $F \vdash \forever{\phi_1} \to \phi_2$.
\end{theorem}
\begin{proof}
  We will use induction on the derivation of $\phi_2$ from $F, \phi_1$:
  \begin{enumerate}
    \item if $\phi_2$ is an axiom or $\phi_2 \in F$, then $F \vdash \phi_2$ and $F \vdash \forever{\phi_1} \to \phi_2$ holds as
      $\phi_2$ always holds;
    \item if $\phi_2 \equiv \phi_1$, then $F \vdash \forever{\phi_1} \to \phi_1 \land \nextop{\forever{\phi_1}}$ from
      (for) and by tautology $F \vdash \forever{\phi_1} \to \phi_1$;
    \item if $\phi_2$ is a conclusion of (mp) rule with premises $\psi$ and $\psi \to \phi_2$, then we have $F, \phi_1 \vdash
      \psi$ and $F, \phi_1 \vdash \psi \to \phi_2$ that by induction hypothesis imply $F \vdash \forever{\phi_1} \to \psi$
      and $F \vdash \forever{\phi_1} \to (\psi \to \phi_2)$:

      \begin{tabularx}{300pt}{cXl}
        (1) & $\forever{\phi_1} \to \psi$ & assumption \\
        (2) & $\forever{\phi_1} \to (\psi \to \phi_2)$ & assumption \\
        (3) & $(\forever{\phi_1} \to \psi) \to (\forever{\phi_1} \to \phi_2)$ & (2), axiom 2 \\
        (4) & $\forever{\phi_1} \to \phi_2$ & (1), (3), (mp) \\
      \end{tabularx}
    \item if $\phi_2 \equiv \nextop{\psi}$ is a conclusion of (nex) rule with premise $\psi$, then $F, \phi_1 \vdash \psi$
      and by induction hypothesis $F \vdash \forever{\phi_1} \to \psi$:

      \begin{tabularx}{300pt}{cXl}
        (1) & $\forever{\phi_1} \to \psi$ & assumption \\
        (2) & $\nextop{(\forever{\phi_1} \to \psi)}$ & (1), (nex) \\
        (3) & $\nextop{(\forever{\phi_1} \to \psi)} \to (\nextop{\forever{\phi_1}} \to \nextop{\psi})$ & (2), axiom 5 \\
        (4) & $\nextop{\forever{\phi_1}} \to \nextop{\psi}$ & (2), (3), (mp) \\
        (5) & $\forever{\phi_1} \to \phi_1 \land \nextop{\forever{\phi_1}}$ & (for) \\
        (6) & $\forever{\phi_1} \to \nextop{\forever{\phi_1}}$ & (prop) \\
        (7) & $\forever{\phi_1} \to \nextop{\psi}$ & (4), (6), (prop) \\
      \end{tabularx}

    \item if $\phi_2 \equiv \psi_1 \to \until{\psi_2}{\psi_3}$ is a conclusion of a (ind) rule with premise $\psi_1 \to
      \psi_3 \lor (\psi_2 \land \nextop{\psi_1})$, then $F, \phi_1 \vdash \psi_1 \to \psi_3 \lor (\psi_2 \land
      \nextop{\psi_1})$ and by induction hypothesis $F \vdash \forever{\phi_1} \to (\psi_1 \to \psi_3 \lor (\psi_2 \land
      \nextop{\psi_1}))$:

      \begin{tabularx}{300pt}{cXl}
        (1) & $\forever{\phi_1} \to (\psi_1 \to \psi_3 \lor (\psi_2 \land \nextop{\psi_1}))$ & assumption \\
        (2) & $(\psi_1 \to \psi_3 \lor (\psi_2 \land \nextop{\psi_1})) \to (\psi_1 \to \until{\psi_2}{\psi_3})$ & axiom 7\\
        (3) & $(\forever{\phi_1} \land \psi_1 \to \psi_3 \lor (\psi_2 \land \nextop{\psi_1})) \to (\psi_1 \to
        \until{\psi_2}{\psi_3})$ & (2), (prop) \\
        (4) & $\forever{\phi_1} \land \psi_1 \to \psi_3 \lor (\psi_2 \land \nextop{\psi_1})$ & (1), (prop) \\
        (5) & $\forever{\phi_1} \to (\psi_1 \to \until{\psi_2}{\psi_3})$ & (3),(4),(mp) \\
      \end{tabularx}

    \item if $\phi_2 \equiv (\existss{x}{\psi_1}) \to \psi_2$ is a conclusion of a (par) rule with premise $\psi_1 \to
      \psi_2$ and $x$ not free in $\psi_2$ and $\phi_1$, then $F, \phi_1 \vdash \psi_1 \to \psi_2$ and by induction
      hypothesis $F \vdash \forever{\phi_1} \to (\psi_1 \to \psi_2)$:

      \begin{tabularx}{300pt}{cXl}
        (1) & $\forever{\phi_1} \to (\psi_1 \to \psi_2)$ & assumption \\
        (2) & $\psi_1 \to (\forever{\phi_1} \to \psi_2)$ & (1), prop \\
        (3) & $(\existss{x}{\psi_1}) \to (\forever{\phi_1} \to \psi_2)$ & (2), (par) \\
        (4) & $\forever{\phi_1} \to ((\existss{x}{\psi_1}) \to \psi_2)$ & (3), (prop) \\
      \end{tabularx}
  \end{enumerate}
\end{proof}

\begin{theorem}
  Let $\phi_1, \phi_2 \in \mathcal{F}_\Sigma$ and $F \subseteq \mathcal{F}_\Sigma$. If $F \vdash \forever{\phi_1} \to
  \phi_2$, then $F, \phi_1 \vdash \phi_2$.
\end{theorem}
\begin{proof}
  Assume $F \vdash \forever{\phi_1} \to \phi_2$, then $F,\phi_1 \vdash \forever{\phi_1} \to \phi_2$ also
  holds. By the (alw) rule~(\Cref{ex:alw}) and the trivial derivation $F, \phi_1 \vdash \phi_1$. Therefore
  $F, \phi_1 \vdash \forever{\phi_1}$, then by (mp) rule it follows $F, \phi_1 \vdash \phi_2$.
\end{proof}

\section{The missing axioms}
Readers accustomed to Kripke-based semantics for quantified temporal logic, will know that axiomatic systems contains a
formula called Barcan formula~\cite{barcan_functional_1946}, $\nextop{\existss{x}{\phi}} \to \existss{x}{\nextop{\phi}}$
or one of its many equivalent variations, which is missing in our system.  Let $\pi = w_0
\overset{cr_0}{\rightsquigarrow} w_1 \overset{cr_1}{\rightsquigarrow} \ldots$ be a generic sequence of accessible worlds
and $\sigma$ an assignment for the world $w_0$.
\begin{align*}
  \sem{\pi}{\sigma}{\Gamma}{\nextop{\existss{x_\tau}{\phi}}}
    &\Leftrightarrow \existss{v \in d(w_1)}{\sem{\suffix(\pi, 1)}{\ext{x}{cr_0 \circ \sigma}{v}}{\Gamma,x}{\phi}} \\
  \sem{\pi}{\sigma}{\Gamma}{\existss{x_\tau}{\nextop{\phi}}}
    &\Leftrightarrow \existss{v \in d(w_0)}{\sem{\suffix(\pi, 1)}{cr_0 \circ \ext{x}{\sigma}{v}}{\Gamma,x}{\phi}}
\end{align*}
The existential operators on the two sides quantifies on different domains. 
Assume both domains for the worlds $w_0$ and $w_1$ are non-empty and have the same algebra. We will see if the Barcan
formula holds for the formula $\phi \equiv x_\tau = c$ where $c$ is a constant of type $\tau$ in the algebra.  Assume
that the counterpart relation $cr_0$ is the empty relation. The formula $\nextop{\existss{x_\tau}{x = c}}$ is trivially
satisfied by extending the empty substitution, since the formula is closed, with the value $c$ for the variable $x$. The
other side is problematic, because the substitution can still be extended with the value $c$ for the variable $x$,
however the composition with the counterpart relation reduces the association to be undefined, which does not satisfy
the equality. Systems that include some form of the Barcan formula as an axiom, require domains to be constant while
traversing accessible worlds, or that domains do not grow or shrink, allowing at least one side of the implication.
Beside philosophical questions~\cite{williamson_modal_2015} regarding the value of such axiom, we will not extend our
model to include such axiom, since one of the main drive of this alternative interpretation of temporal formulae is the
necessity of modeling systems with creation and deallocation of resources.

\section{Completeness and Decidability}
The model presented in this work is not exempt from all the consideration on completeness available in literature for
quantified temporal logic~\cite{merz_decidability_1992}. Since the signature and the axioms for the theory of natural
numbers can be encoded in our system, it follows that our logic is
incomplete~\cite{barwise_incompleteness_1977,tarski_undecidable_1953}.
\begin{example}
  The theory of natural numbers can be encoded with the following signature $\Sigma = (S_\Sigma, F_\Sigma)$ where
  \[
    \begin{split}
      S_\Sigma &= \set{\tau_\mathbb{N}} \\
      F_\Sigma &= \set{0 : \tau_\mathbb{N}, s : \tau_\mathbb{N} \to \tau_\mathbb{N}, + : \tau_\mathbb{N} \times
      \tau_\mathbb{N} \to \tau_\mathbb{N}, \times : \tau_\mathbb{N} \times \tau_\mathbb{N} \to \tau_\mathbb{N}}
    \end{split}
  \]
  and by models where the following additional axioms are true:
  \begin{itemize}
    \item $s(x) \neq 0$
    \item $s(x) = s(y) \to x = y$
    \item $s(x) \neq 0 \to \existss{y_{\tau_\mathbb{N}}}{x = s(y)}$
    \item $x + 0 = x$
    \item $x + s(y) = s(x + y)$
    \item $x \times 0 = 0$
    \item $x \times s(y) = (x \times y) + x$
  \end{itemize}
  and the following axiom for the scheme of induction, where $\phi$ is a formula with $x$ free:
  \[
    \phi[0/x] \land (\foralls{x_{\tau_\mathbb{N}}}{\phi \to \phi[s(x)/x]}) \to \foralls{x_{\tau_\mathbb{N}}}{\phi}
  \]
\end{example}

More articulated is, instead, the case for decidability. Previous results show that \ac{FOLTL} is not only undecidable
in general, but in a multi-sorted context is not even semi-decidable~\cite{merz_decidability_1992}. However, researches
also showed that there are reasonable fragments that are, indeed, decidable but with serious
restrictions~\cite{hodkinson_decidable_2000,hodkinson_decidable_2002,goos_monodic_2001,peyras_decidable_2020,peyras_bounded_2019}.
The main issue is the interaction between temporal operators and quantifiers, in particular the until operator. One
approach is to limit the number of variables quantified within the scope of temporal operators, in particular
restricting temporal formulae to be monodic, thanks to decidability results on monodic second-order
logic~\cite{rozenberg_expression_1997}. Another approach is to limit the nestedness of operators, for example we must
avoid nested forever operators, or the usage of temporal operators inside the scope of a universal quantifier.
Nonetheless, there are still issues with classical operators, and we must restrict also the usage of those to an
arbitrary decidable fragment of first-order classical logic. Lastly, rather than focus of finding fragments that
guarantee decidability for every model, we can limit the class of models to those there are decidable, essentially
finitary ones, therefore not only the set of worlds is finite, but also the algebras must be finite. Both kinds of
restriction are strict enough to obtain a decidable fragment or decidable models of the second-order
$\mu$-calculus~\cite{hutchison_counterpart_2010}, which subsumes temporal logic.

A future objective is to investigate further this matter of decidability, as one of the objective of this model is to be
useful, and even simplier, for model checking purposes.
