\section{Axioms}
\begin{definition}[Axiomatic system]\label{def:axioms}
  The axiomatic system for FO-LTL contains the following axioms:
  \begin{enumerate}
    \item $\phi_1 \to (\phi_2 \to \phi_1)$;
    \item $(\phi_1 \to (\phi_2 \to \phi_3)) \to ((\phi_1 \to \phi_2) \to (\phi_1 \to \phi_3))$;
    \item $(\neg\phi_1 \to \neg\phi_2) \to (\phi_2 \to \phi_1)$;
    \item $\neg\nextop{\phi} \to \nextop{\neg\phi}$;
    \item $\nextop{(\phi_1 \to \phi_2)} \to (\nextop{\phi_1} \to \nextop{\phi_2})$;
    \item $\phi \to \nextop{\phi}$ if $\phi$ is rigid;
    \item $\until{\phi_1}{\phi_2} \leftrightarrow \phi_2 \lor (\phi_1 \land \nextop{(\until{\phi_1}{\phi_2})})$;
    \item $\until{\phi_1}{\phi_2} \to \eventually{\phi_2}$;
    \item $\phi[v/x] \to \existss{x}{\phi}$, with $x$ free in $\phi$ and $v$ a value in the domain;
    \item $x = x$;
    \item $x = y \to (\phi \to \phi[y/x])$.
  \end{enumerate}
  And the following induction rules:
  \begin{enumerate}
    \item[mp] $\phi_1, \phi_1 \to \phi_2 \vdash \phi_2$;
    \item[nex] $\phi \vdash \nextop{\phi}$;
    \item[ind] $\phi_1 \to \phi_3 \lor (\phi_2 \land \nextop{\phi_1}) \vdash \phi_1 \to \until{\phi_2}{\phi_3}$;
    \item[par] $\phi_1 \to \phi_2 \vdash (\existss{x}{\phi_1}) \to \phi_2$ with $x$ not free in $\phi_2$.
  \end{enumerate}
\end{definition}

\begin{lemma}\label{lem:propaxioms}
  Let $\phi_1, \phi_2, \phi_3 \in \mathcal{F}_\Sigma$.
  \begin{enumerate}
    \item $\vDash \phi_1 \to (\phi_2 \to \phi_1)$;
    \item $\vDash (\phi_1 \to (\phi_2 \to \phi_3)) \to ((\phi_1 \to \phi_2) \to (\phi_1 \to \phi_3))$;
    \item $\vDash (\neg\phi_1 \to \neg\phi_2) \to (\phi_2 \to \phi_1)$.
  \end{enumerate}
\end{lemma}
\begin{proof}
  Let $M$ be a counterpart model and a context $\Gamma$, a world $w \in W$ and an assignment $\sigma$:
  \begin{enumerate}
    \item From trivial applications of the semantic rules it follows that $\sem{\sigma}{w}{\Gamma}{\phi_1 \to (\phi_2
      \to \phi_1)} \Leftrightarrow \semn{\sigma}{w}{\Gamma}{\phi_1} \text{ or } \semn{\sigma}{w}{\Gamma}{\phi_2}
      \text{ or } \sem{\sigma}{w}{\Gamma}{\phi_1}$ which is a tautology by the law of excluded middle.

    \item By~\Cref{eq:iffsem} we show that the stronger formula $\vDash (\phi_1 \to (\phi_2 \to \phi_3)) \leftrightarrow ((\phi_1 \to \phi_2)
      \leftrightarrow (\phi_1 \to \phi_3))$. From the semantic rules:
      \[
        \begin{split}
          &\sem{\sigma}{w}{\Gamma}{\phi_1 \to (\phi_2 \to \phi_3)} \\
            &\quad\Leftrightarrow \semn{\sigma}{w}{\Gamma}{\phi_1} \text{ or } \semn{\sigma}{w}{\Gamma}{\phi_2} \text{ or }
              \sem{\sigma}{w}{\Gamma}{\phi_3} \\
            &\quad\Leftrightarrow (\sem{\sigma}{w}{\Gamma}{\phi_1} \text{ and } \semn{\sigma}{w}{\Gamma}{\phi_2}) \text{ or }
              \semn{\sigma}{w}{\Gamma}{\phi_1} \text{ or } \sem{\sigma}{w}{\Gamma}{\phi_3} \\
            &\quad\Leftrightarrow \sem{\sigma}{w}{\Gamma}{(\phi_1 \to \phi_2) \to (\phi_1 \to \phi_3)}
        \end{split}
      \]

    \item By~\Cref{eq:iffsem} we show the stronger formula $\vDash (\neg\phi_1 \to \neg\phi_2) \leftrightarrow (\phi_2 \to \phi_1)$.
      From the semantic rules:
      \[
        \begin{split}
          \sem{\sigma}{w}{\Gamma}{\neg\phi_1 \to \neg\phi_2}
            &\Leftrightarrow \sem{\sigma}{w}{\Gamma}{\phi_1} \text{ or } \semn{\sigma}{w}{\Gamma}{\phi_2} \\
            &\Leftrightarrow \sem{\sigma}{w}{\Gamma}{\phi_2 \to \phi_1}
        \end{split}
      \]
  \end{enumerate}
\end{proof}

\begin{lemma}\label{lem:negnextdist}
  Let $\phi \in \mathcal{F}_\Sigma$, $\vDash \neg\nextop{\phi} \to \nextop{\neg\phi}$.
\end{lemma}
\begin{proof}
  Let $M$ be a counterpart model and a context $\Gamma$, a world $w \in W$ and an assignment $\sigma$.
  \[
    \begin{split}
      \sem{\sigma}{w}{\Gamma}{\neg\nextop{\phi}}
        &\Leftrightarrow \neg\foralls{\worldcr{w}{cr}{w'}}{\sem{cr \circ \sigma}{w'}{\Gamma}{\phi}} \\
        &\Rightarrow \neg\existss{\worldcr{w}{cr}{w'}}{\sem{cr \circ \sigma}{w'}{\Gamma}{\phi}} \\
        &\Leftrightarrow \foralls{\worldcr{w}{cr}{w'}}{\semn{cr \circ \sigma}{w'}{\Gamma}{\phi}} \\
        &\Leftrightarrow \sem{\sigma}{w}{\Gamma}{\nextop{\neg\phi}}
    \end{split}
  \]
\end{proof}

\begin{lemma}\label{lem:impnextdist}
  Let $\phi_1, \phi_2 \in \mathcal{F}_\Sigma$, $\vDash \nextop{(\phi_1 \to \phi_2)} \to (\nextop{\phi_1} \to
  \nextop{\phi_2})$.
\end{lemma}
\begin{proof}
  Let $M$ be a counterpart model and a context $\Gamma$, a world $w \in W$ and an assignment $\sigma$:
  \[
    \begin{split}
      &\sem{\sigma}{w}{\Gamma}{\nextop{(\phi_1 \to \phi_2)}} \\
        &\quad\Leftrightarrow \foralls{\worldcr{w}{cr}{w'}}{\sem{cr \circ \sigma}{w'}{\Gamma}{\phi_1} \text{ implies }
            \sem{cr \circ \sigma}{w'}{\Gamma}{\phi_2}} \\
        &\quad\Rightarrow \foralls{\worldcr{w}{cr}{w'}}{\sem{cr \circ \sigma}{w'}{\Gamma}{\phi_1}} \text{ implies }
            \foralls{\worldcr{w}{cr}{w'}}{\sem{cr \circ \sigma}{w'}{\Gamma}{\phi_2}} \\
        &\quad\Leftrightarrow \sem{\sigma}{w}{\Gamma}{\nextop{\phi_1} \to \nextop{\phi_2}}
    \end{split}
  \]
\end{proof}

\begin{lemma}\label{lem:nextintro}
  Let $\phi \in \mathcal{F}_\Sigma$ be a rigid formula, $\vDash \phi \to \nextop{\phi}$.
\end{lemma}
\begin{proof}
  Let $M$ be a counterpart model and a context $\Gamma$, a world $w \in W$ and an assignment $\sigma$.
  Let $\sem{\sigma}{w}{\Gamma}{\phi}$. By definition, $\sem{\sigma}{w}{\Gamma}{\nextop{\phi}}$ iff
  $\foralls{\worldcr{w}{cr}{w'}}{\sem{cr \circ \sigma}{w'}{\Gamma}{\phi}}$. Let $w'$ be counterpart related to $w$.
  Since $\phi$ is a rigid formula, $\sem{cr \circ \sigma}{w'}{\Gamma}{\phi}$ must hold.
\end{proof}

\begin{lemma}\label{lem:untilexp}
  Let $\phi_1, \phi_2 \in \mathcal{F}_\Sigma$, $\vDash \until{\phi_1}{\phi_2} \leftrightarrow \phi_2 \lor (\phi_1 \land
  \nextop{(\until{\phi_1}{\phi_2})})$.
\end{lemma}
\begin{proof}
  Follows trivially by the definition of the operators until and next.
\end{proof}

\begin{lemma}\label{lem:untileventually}
  Let $\phi_1, \phi_2 \in \mathcal{F}_\Sigma$, $\vDash \until{\phi_1}{\phi_2} \to \eventually{\phi_2}$.
\end{lemma}
\begin{proof}
  Let $M$ be a counterpart model and a context $\Gamma$, a world $w \in W$ and an assignment $\sigma$.
  By the definitions of the operators:
  \[
    \begin{split}
      \sem{\sigma}{w}{\Gamma}{\until{\phi_1}{\phi_2}}
      &\Leftrightarrow \sem{\sigma}{w}{\Gamma}{\phi_2} \text{ or } (\sem{\sigma}{w}{\Gamma}{\phi_1} \text{ and }
          \sem{\sigma}{w}{\Gamma}{\nextop{(\until{\phi_1}{\phi_2})}}) \\
      &\Rightarrow \sem{\sigma}{w}{\Gamma}{\phi_2} \text{ or } \sem{\sigma}{w}{\Gamma}{\nextop{(\until{\phi_1}{\phi_2})}} \\
      &\Rightarrow \sem{\sigma}{w}{\Gamma}{\phi_2} \text{ or } \sem{\sigma}{w}{\Gamma}{\nextop{(\until{tt}{\phi_2})}} \\
      &\Leftrightarrow \sem{\sigma}{w}{\Gamma}{\eventually{\phi_2}}
    \end{split}
  \]
\end{proof}

\begin{lemma}\label{lem:exintro}
  Let $\phi \in \mathcal{F}_\Sigma$ with $x \in X$ free in $\phi$, $\vDash \phi[v/x] \to \existss{x}{\phi}$, if $v$ is a
  value in the domain.
\end{lemma}
\begin{proof}
  Let $M$ be a counterpart model and a context $\Gamma$, a world $w \in W$ and an assignment $\sigma$.
  By hypothesis $\sem{\sigma}{w}{\Gamma}{\phi[v/x]}$, with $v \in d(w)$, then it holds that
  $\sem{\ext{x}{\sigma}{v}}{w}{\Gamma,x}{\phi}$, therefore $\sem{\sigma}{w}{\Gamma}{\existss{x}{\phi}}$ holds.
\end{proof}

\begin{lemma}\label{lem:equality}
  Let $\phi \in \mathcal{F}_\Sigma$ and $x, y \in X$,
  \begin{enumerate}
    \item $\vDash x = x$;
    \item $\vDash x = y \to (\phi \to \phi[y/x])$.
  \end{enumerate}
\end{lemma}
\begin{proof}
  Let $M$ be a counterpart model and a context $\Gamma$, a world $w \in W$ and an assignment $\sigma$.
  \begin{enumerate}
    \item By definition, $\sem{\sigma}{w}{\Gamma}{x = x}$ holds if $\sigma(x) = \sigma(x)$, which is trivially true.
    \item By assumption it holds that $\sem{\sigma}{w}{\Gamma}{x = y}$, thus $\sigma(x) = \sigma(y)$. Assume
      $\sem{\sigma}{w}{\Gamma}{\phi}$ holds, then by the congruence property of equality it must also hold
      $\sem{\sigma}{w}{\Gamma}{\phi[y/x]}$.
  \end{enumerate}
\end{proof}

\begin{lemma}\label{lem:mp}
  If $F \vDash \phi_1$ and $F \vDash \phi_1 \to \phi_2$, then $F \vDash \phi_2$.
\end{lemma}
\begin{proof}
  Let $M$ be a counterpart model that satisfies $\vDash_M \psi$ for every $\psi \in F$.
  By definition $\sem{\sigma}{w}{\Gamma}{\phi_1 \to \phi_2} \Leftrightarrow \sem{\sigma}{w}{\Gamma}{\phi_1}
  \text{ implies } \sem{\sigma}{w}{\Gamma}{\phi_2}$. By assumption it holds $\sem{\sigma}{w}{\Gamma}{\phi_1}$, thus it must
  hold that $\sem{\sigma}{w}{\Gamma}{\phi_2}$.
\end{proof}
\begin{lemma}\label{lem:nex}
  If $F \vDash \phi$, then $F \vDash \nextop{\phi}$ and $F \vDash \forever{\phi}$.
\end{lemma}
\begin{proof}
  Let $M$ be a counterpart model that satisfies $\vDash_M \psi$ for every $\psi \in F$ and $\Gamma$.
  By assumption $\sem{\sigma}{w}{\Gamma}{\phi}$ holds for every pair $(\sigma, w)$. In particular given a pair $(\sigma,
  w)$, it must also hold $\sem{cr \circ \sigma}{w'}{\Gamma}{\phi}$ for every $\worldcr{w}{cr}{w'}$.
\end{proof}
\begin{lemma}\label{lem:ind}
  If $F \vDash \phi_1 \to \phi_2$ and $F \vDash \phi_1 \to \nextop{\phi_1}$, then $F \vDash \phi_1 \to \forever{\phi_2}$.
\end{lemma}
\begin{proof}
  Let $M$ be a counterpart model that satisfies $\vDash_M \psi$ for every $\psi \in F$ and $\Gamma$.  If
  $\semn{\sigma}{w}{\Gamma}{\phi_1}$ than the conclusion trivially holds. Assume, instead,
  $\sem{\sigma}{w}{\Gamma}{\phi_1}$. By expanding the formula we need to show that $\sem{\sigma'}{w'}{\Gamma}{\phi_2}$
  holds for any pair $(\sigma', w')$ such that is the composition of an arbitrary amount of counterpart relations, if $w
  \overset{cr_0}{\rightsquigarrow} \cdots \overset{cr_n}{\rightsquigarrow} w'$ then $\sigma' = cr_n \circ \cdots \circ
  cr_0 \circ \sigma$. However, by hypothesis $\sem{\sigma}{w}{\Gamma}{\phi_2}$ and
  $\sem{\sigma}{w}{\Gamma}{\nextop{\phi_1}}$ are valid in every $(\sigma, w)$ such that
  $\sem{\sigma}{w}{\Gamma}{\phi_1}$ is valid, therefore $\phi_1$ is valid in every sequence described before and thus
  $\phi_2$ is valid in every sequence.
\end{proof}
\begin{lemma}\label{lem:par}
  If $F \vDash \phi_1 \to \phi_2$ and $x$ not free in $\phi_2$, then $F \vDash (\existss{x}{\phi_1}) \to \phi_2$.
\end{lemma}
\begin{proof}
  Let $M$ be a counterpart model that satisfies $\vDash_M \psi$ for every $\psi \in F$ and $\Gamma$.
  Assume $\semn{\sigma}{w}{\Gamma}{(\existss{x}{\phi_1}) \to \phi_2}$ for some pair $(\sigma, w)$. Then
  $\sem{\sigma}{w}{\Gamma}{\existss{x}{\phi_1}}$ and $\semn{\sigma}{w}{\Gamma}{\phi_2}$.
  Then there is a $v \in d(w)$ such that $\sem{\ext{x}{\sigma}{v}}{w}{\Gamma, x}{\phi_1}$, by definition of existential
  quantifier. Since $\phi_2$ does not contain $x$ as a free variable, adding an evaluation for $x$ does not change the
  evaluation of $\phi_2$, i.e. $\semn{\ext{x}{\sigma}{v}}{w}{\Gamma, x}{\phi_2}$, therefore
  $\semn{\ext{x}{\sigma}{v}}{w}{\Gamma, x}{\phi_1 \to \phi_2}$, which is a contradiction to the hypothesis $\vDash_M
  \phi_1 \to \phi_2$.
\end{proof}

\begin{theorem}[Soundness]
  Let $\phi \in \mathcal{F}_\Sigma$ and $F \subseteq \mathcal{F}_\Sigma$, if $F \vdash \phi$ then $F \vDash \phi$.
\end{theorem}
\begin{proof}
  By induction on the derivation of $\phi$ from $F$:
  \begin{enumerate}
    \item if $\phi$ is an axiom: $F \vDash \phi$ is proven by~\Cref{lem:propaxioms} for axioms 1,2,3;
      by~\Cref{lem:negnextdist} for axiom 4; by~\Cref{lem:impnextdist} for axiom 5; by~\Cref{lem:nextintro} for axiom 6;
      by~\Cref{lem:untilexp} for axiom 7; by~\Cref{lem:untileventually} for axiom 8; by~\Cref{lem:exintro} for axiom 9;
      by~\Cref{lem:equality} for axioms 10, 11;
    \item if $\phi \in F$ then $F \vDash \phi$ holds trivially;
    \item if $\phi$ is the conclusion of a (mp) rule with premises $F \vdash \psi$ and $F \vdash \psi \to \phi$: by
      induction hypothesis we have $F \vDash \psi$ and $F \vDash \psi \to \phi$, hence $F \vDash \phi$ follows by
      \Cref{lem:mp};
    \item if $\phi$ is the conclusion of a (nex) rule with premises $F \vdash \psi$, thus $\phi \equiv \nextop{\psi}$: by
      induction hypothesis we have $F \vDash \psi$, hence $F \vDash \nextop{\psi}$ follows by \Cref{lem:nex};
    \item if $\phi$ is the conclusion of a (ind) rule with premises $F \vdash \psi_1 \to \psi_2$ and $F \vdash \psi_1
      \to \nextop{\psi_1}$, thus $\phi \equiv \psi_1 \to \forever{\psi_2}$: by induction hypothesis we have $F \vDash
      \psi_1 \to \psi_2$ and $F \vDash \psi_1 \to \nextop{\psi_2}$, hence $F \vDash \psi_1 \to \forever{\psi_2}$ follows
      by \Cref{lem:ind};
    \item if $\phi$ is the conclusion of a (par) rule with premises $F \vdash \psi_1 \to \psi_2$, thus $\phi \equiv
      \existss{x}{\psi_1} \to \psi_2$ with $x$ not free in $\psi_2$: by induction hypothesis we have $F \vDash \psi_1
      \to \psi_2$, hence $F \vDash \exists{x}{\psi_1} \to \psi_2$ follows by \Cref{lem:par}.
  \end{enumerate}
\end{proof}

\begin{example}\label{ex:alw}
  We show the derivation of some rules to show the capabilities of the deduction system. First a variant of the
  (ind) rule, $\phi \to \nextop{\phi} \vdash \phi \to \forever{\phi}$, which we will call (ind1):

  \medskip
  \begin{tabularx}{300pt}{cXl}
    (1) & $\phi \to \nextop{\phi}$ & assumption \\
    (2) & $\phi \to \phi$ & tautology \\
    (3) & $\phi \to \forever{\phi}$ & (ind), (1), (2)
  \end{tabularx}

  \medskip
  Next we show the (alw) rule, $\phi \vdash \forever{\phi}$, which will be useful later for the deduction theorems.

  \medskip
  \begin{tabularx}{300pt}{cXl}
    (1) & $\phi$ & assumption \\
    (2) & $\nextop{\phi}$ & (nex), (1) \\
    (3) & $\phi \to \nextop{\phi}$ & (axiom 1), (2) \\
    (4) & $\phi \to \forever{\phi}$ & (ind1), (3) \\
    (5) & $\forever{\phi}$ & (mp), (1), (4) \\
  \end{tabularx}
\end{example}

\begin{theorem}
  Let $\phi_1, \phi_2 \in \mathcal{F}_\Sigma$ and $F \subseteq \mathcal{F}_\Sigma$. If $F \cup \set{\phi_1} \vdash
  \phi_2$ and this deriviation of $\phi_2$ does not contains application of the rule (par) for a variable occurring free
  in $\phi_1$, then $F \vdash \forever{\phi_1} \to \phi_2$.
\end{theorem}
\begin{proof}
  TODO
\end{proof}

\begin{theorem}
  Let $\phi_1, \phi_2 \in \mathcal{F}_\Sigma$ and $F \subseteq \mathcal{F}_\Sigma$. If $F \vdash \forever{\phi_1} \to
  \phi_2$, then $F \cup \set{\phi_1} \vdash \phi_2$.
\end{theorem}
\begin{proof}
  Assume $F \vdash \forever{\phi_1} \to \phi_2$, then $F \cup \set{\phi_1} \vdash \forever{\phi_1} \to \phi_2$ also
  holds. By the (alw) rule~(\Cref{ex:alw}) and the trivial derivation $F \cup \set{\phi_1} \vdash \phi_1$ it follows
  that $F \cup \set{\phi_1} \vdash \forever{\phi_1}$, then by (mp) rule it follows $F \cup \set{\phi_1} \vdash \phi_2$.
\end{proof}

\section{Missing axioms}
Readers may have notice that some formulae that intuitively should be bidirectional are instead unidirectional in our
model, e.g. the expansion of the forever operator~(\Cref{eq:foreversem}) and the axiom 4~(\Cref{def:axioms}). Moreover,
another well studied formula that is often available in axiomatic systems for temporal logic, is the so-called Barcan
formula, $\nextop{\existss{x_\tau}{\phi}} \to \existss{x_\tau}{\nextop{\phi}}$, which describes distributivity between
first-order quantifiers and temporal operators.
These issue are direct consequence of the increased generality of our model. In particular the first two examples can be
solved by constraining the accessibility relation to a function, and thus each world has a single accessible world in
its future, a standard assumption for the common linear temporal logic models.
With this model we can transform axiom 4 into the formula $\neg\nextop{\phi} \leftrightarrow \nextop{\neg\phi}$, since
the uniqueness of the transition solves the issue with the interplay between negation and universal quantification:
\[
  \begin{split}
    \sem{\sigma}{w}{\Gamma}{\neg\nextop{\phi}}
      &\Leftrightarrow \semn{cr \circ \sigma}{w'}{\Gamma}{\phi} \text{ where } \worldcr{w}{cr}{w'} \\
      &\Leftrightarrow \sem{\sigma}{w}{\Gamma}{\nextop{\neg\phi}}
  \end{split}
\]

This change, however, is still not sufficient for the Barcan formula.
\[
  \begin{split}
    \sem{\sigma}{w}{\Gamma}{\nextop{\existss{x_\tau}{\phi}}}
      &\Leftrightarrow \foralls{\worldcr{w}{cr}{w'}}{\existss{v \in d(w')}{\sem{\ext{x}{cr \circ \sigma}{v}}{w'}{\Gamma,x}{\phi}}} \\
    \sem{\sigma}{w}{\Gamma}{\existss{x_\tau}{\nextop{\phi}}}
      &\Leftrightarrow \existss{v \in d(w)}{\foralls{\worldcr{w}{cr}{w'}}{\sem{cr \circ \ext{x}{\sigma}{v}}{w'}{\Gamma,x}{\phi}}}
  \end{split}
\]
The existential operators on the two sides quantifies on different domains. Assume we have an accessibility relation
where a world $w$ is related only to a world $w'$ where both domains are non-empty and the same algebra.  We will see if
the Barcan formula holds for the formula $\phi \equiv x_\tau = c$ where $c$ is a constant of type $\tau$ in the algebra.
Assume the two worlds are related by the counterpart relation $cr$ that is the empty relation. The formula
$\nextop{\existss{x_\tau}{x = c}}$ is trivially satisfied by extending the empty substitution, since the formula is
closed, with the value $c$ for the variable $x$. The other side is problematic, because the substitution can
still be extended with the value $c$ for the variable $x$, however the composition with the counterpart relation reduces
the association to be undefined, which does not satisfy the equality. Systems that include some form of the Barcan
formula as an axiom, require domains be constant while traversing accessible worlds, or that domains do not grow or
shrink, allowing at least one side of the implication. Beside philosophical question regarding the value of such axiom,
we will not extend our model to include such axiom, as one of the main drive of this alternative interpretation of
temporal formulae is the necessity of modeling systems with creation and deallocation of resources.
