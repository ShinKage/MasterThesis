\section{Algebra}\label{sec:algebra}

\begin{definition}[Many-sorted Signature]
  A \emph{many-sorted signature} $\Sigma$ is a pair $(S_\Sigma, F_\Sigma)$ where $S_\Sigma = \set{\tau_1, \ldots,
  \tau_n}$ is a set of sorts, and $F_\Sigma = \set{f_\Sigma : \tau_1 \times \cdots \times \tau_n \to \tau | \tau_i, \tau
  \in \Sigma_\tau, i = 1, \ldots, n}$ is a set of function symbols.
\end{definition}

We show a couple of signatures that will be used throughout the work. First, let's model simple directed graphs, which
have two sorts: vertices, $\tau_v$, and edges $\tau_e$. Naturally to identify edges we require two operations, $s$, $t$
both of type $\tau_e \to \tau_v$, which respectively identify the source and target of a given edge. Thus the complete
signature for simple directed graph is $\Sigma = (\set{\tau_v, \tau_e}, \set{s : \tau_e \to \tau_v, t : \tau_e \to
\tau_v})$.

Next we present the signature for a simplified memory management model. Once again we have two sorts: objects, $\tau_o$, and
memory locations $\tau_l$. The only operation is the function that associate to each object it's memory location,
$\rho$. Thus the complete signature is $\Sigma = (\set{\tau_o, \tau_l}, \set{\rho : \tau_o \to \tau_l})$.

\begin{definition}[Many-sorted Algebra]
  A \emph{many-sorted algebra} $\algebra{A}$ with signature $\Sigma = (S_\Sigma, F_\Sigma)$, or $\Sigma$-algebra, is a
  pair $(A, F_\Sigma^\algebra{A})$ such that:
  \begin{itemize}
    \item $A$ is a family of carrier sets, indexed by the sorts of $\Sigma$;
    \item $F_\Sigma^\algebra{A}$ is a family of functions, indexed by the function symbols in $F_\Sigma$,
      $\set{f_\Sigma^\algebra{A} : A_{\tau_1} \times \cdots \times A_{\tau_n} \to A_\tau | f_\Sigma : \tau_1 \times
      \cdots \times \tau_n \to \tau \in F_\Sigma}$.
  \end{itemize}
\end{definition}

\begin{definition}[Homomorphism]
  Given two $\Sigma$-algebras $\algebra{A}$ and $\algebra{B}$, a \emph{(partial)
  homomorphism} is a family of (partial) functions indexed by the sorts of $\Sigma$, $\set{\rho_\tau : A_\tau
  \rightharpoonup B_\tau | \tau \in S_\Sigma}$, such that for each function symbol $f_\Sigma : \tau_1 \times \cdots \times
  \tau_n \in F_\Sigma$ and list of elements $a_1 \in A_{\tau_1}, \ldots, a_n \in A_{\tau_n}$, if each function
  $\rho_{\tau_i}$ is defined for the element $a_i$ then $\rho_\tau$ is defined for the element $f_\Sigma^\algebra{A}(a_1,
  \ldots, a_n)$ and $\rho_\tau(f_\Sigma^\algebra{A}(a_1, \ldots, a_n)) = f_\Sigma^\algebra{B}(\rho_{\tau_1}(a_1),
  \ldots, \rho_{\tau_n}(a_n))$.
\end{definition}

\begin{figure}
  \begin{center}
    \begin{tikzpicture}[> = stealth', shorten > = 2pt, shorten < = 2pt, auto, node distance = 2cm, semithick]
      \tikzstyle{vertex} = [circle, draw = black, thick, fill = white, minimum size = 2mm]
      \node[vertex] (n0) [] {$n_0$};
      \node[vertex] (n1) [above right = 0.5cm and 1.2cm of n0] {$n_1$};
      \node[vertex] (n2) [below right = 0.5cm and 1.2cm of n0] {$n_2$};
      \path[->] (n0) edge [bend left] node (e0) {$e_0$} (n1);
      \path[->] (n1) edge [bend left] node (e1) {$e_1$} (n2);
      \path[->] (n2) edge [bend left] node (e2) {$e_2$} (n0);

      \node[vertex] (n3) [right = 3cm of n2] {$n_3$};
      \node[vertex] (n4) [right = 3cm of n1] {$n_4$};
      \path[->] (n3) edge [bend right] node [swap] (e3) {$e_3$} (n4);
      \path[->] (n4) edge [bend right] node [swap] (e4) {$e_4$} (n3);

      \node[vertex] (n5) [right = 3cm of n4] {$n_5$};
      \node[vertex] (n6) [right = 3cm of n3] {$n_6$};
      \node[vertex] (n7) [right = 1.2cm of n5] {$n_7$};
      \path[->] (n5) edge [bend right] node [swap] (e5) {$e_5$} (n6);
      \path[->] (n6) edge [bend right] node [swap] (e6) {$e_6$} (n5);
      \path[->] (n5) edge [] node (e7) {$e_7$} (n7);
      \path[->] (n7) edge [loop below] node (e8) {$e_8$} (n7);

      \path[->, dotted, black!50] (n1) edge [bend left] (n4);
      \path[->, dotted, black!50] (n2) edge [bend right] (n3);
      \path[->, dotted, black!50, shorten > = -4pt, shorten < = -4pt] (e1) edge [] (e4);

      \path[->, dotted, black!50] (n4) edge [bend left] (n5);
      \path[->, dotted, black!50] (n3) edge [bend right] (n6);
      \path[->, dotted, black!50, shorten > = -4pt, shorten < = -4pt] (e4) edge [bend left] (e5);
      \path[->, dotted, black!50, shorten > = -4pt, shorten < = -4pt] (e3) edge [bend right] (e6);

    \end{tikzpicture}
  \end{center}
  \caption{Examples of simple directed graphs}
  \label{fig:egsdgraphs}
\end{figure}

Recall the signature for simple directed graphs defined earlier, a $\Sigma$-algebra for that signature is a particular
instance of a simple directed graph and homomorphism between them are partial homomorphism between graphs. For example
the graphs in \autoref{fig:egsdgraphs} are visual representation of the following algebras: $\algebra{G}_0 =
(\set{\set{n_0, n_1, n_2},\set{e_0, e_1, e_2}}, \set{s^{\algebra{G}_0}, t^{\algebra{G}_0}})$, where $s^{\algebra{G}_0} =
\set{e_0 \mapsto n_0, e_1 \mapsto n_1, e_2 \mapsto n_2}$ and $t^{\algebra{G}_0} = \set{e_0 \mapsto n_1, e_1 \mapsto n_2,
e_2 \mapsto n_0}$; $\algebra{G}_1 = (\set{\set{n_3,n_4},\set{e_3,e_4}},\set{s^{\algebra{G}_1},t^{\algebra{G}_1}})$,
where $s^{\algebra{G}_1} = \set{e_3 \mapsto n_3, e_4 \mapsto n_4}$ and $t^{\algebra{G}_1} = \set{e_3 \mapsto e_3, e_4
\mapsto n_3}$.  $\algebra{G}_2 =
(\set{\set{n_5,n_6,n_7},\set{e_5,e_6,e_7,e_8}},\set{s^{\algebra{G}_2},t^{\algebra{G}_2}})$, where $s^{\algebra{G}_2} =
\set{e_5 \mapsto n_5,e_6 \mapsto n_6,e_7\mapsto n_5, e_8 \mapsto n_7}$ and $t^{\algebra{G}_2} = \set{e_5 \mapsto n_6,e_6
\mapsto n_5,e_7 \mapsto n_7, e_8 \mapsto n_7}$.

In the same figure are also visually represented partial homomorphism between those graphs that can be translated to:
$\algebra{G}_0 \to \algebra{G}_1 = \set{\set{n_1 \mapsto n_4,n_2 \mapsto n_3},\set{e_1 \mapsto e_4}}$ and $\algebra{G}_1
\to \algebra{G}_2 = \set{\set{n_4 \mapsto n_5, n_3 \mapsto n_6},\set{e_4 \mapsto e_5, e_3 \mapsto e_6}}$, where the
requirements for homomorphism are trivially checkable. Notice that the first example of homomorphism shows an interesting
behaviour for our purpose. Due to partiality it removes the node $n_0$ and the edges $e_0$, $e_2$ while preserving the
structure of the nodes $n_1$, $n_2$ and the edge $e_1$ between them.

To model existential properties, we can extend a signature $\Sigma$ with a denumerable set $X$ of variables typed over
$S_\Sigma$, obtaining the signature $\Sigma_X$. The $\tau$-typed subset of $X$ is denoted with $X_\tau$, and typed
variables are denoted with $x_\tau$ or $x : \tau$. $\tau$-sorted terms are denoted with $\epsilon_\tau$ or $\epsilon :
\tau$.

\begin{definition}[Term]
  Let $\Sigma$ be a signature, let $X$ be a denumerable set of individual variables typed over $S_\Sigma$, and let
  $\Sigma_X$ be the signature obtained by extending $\Sigma$ with $X$. The (many-sorted) set $\terms{\Sigma}{X}$ of
  \emph{terms} is the smallest set such that:
  \[
    \begin{prooftree}
      \hypo{\phantom{[1,n]}} % ALIGNMENT ONLY
      \infer1{X \subseteq \terms{\Sigma}{X}}
    \end{prooftree}
    \qquad
    \begin{prooftree}
      \hypo{f : \tau_1 \times \cdots \tau_n \to \tau \in F_\Sigma}
      \hypo{\forall i \in [1,n] \ldotp \epsilon_i : \tau_i \in \terms{\Sigma}{X}}
      \infer2{f(\epsilon_1, \ldots, \epsilon_n) : \tau \in \terms{\Sigma}{X}}
    \end{prooftree}
  \]
\end{definition}

\begin{example}[Terms]
  Let be $\Sigma = (\set{\tau}, \set{1 : \tau, \otimes : \tau \to \tau})$ a monoidal signature with a single sort. Let
  be $X$ a denumerable set of variables with $x, y, z \in X$, then some example of terms in $\terms{\Sigma}{X}$ are $1
  \otimes 1 : \tau$, $x \otimes (y \otimes 1) : \tau$ and $(x \otimes 1) \otimes (1 \otimes y) : \tau$. Examples from
  the previously defined graph algebra $\Sigma$ are $s(x)$ and $t(y)$ which represent respectively the source vertex for
  an edge $x : \tau_e$ and the target vertex for an edge $y : \tau_e$.
\end{example}
