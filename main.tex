\documentclass[a4paper,11pt]{report}
\usepackage[T1]{fontenc}
\usepackage[utf8]{inputenc}
\usepackage[english]{babel}

\usepackage{amsmath}
\usepackage{amssymb}
\usepackage{amsthm}
\usepackage{mathtools}
\usepackage{thmtools}
\usepackage{stmaryrd}
\usepackage{braket}
\usepackage{ebproof}
\usepackage{tabularx}

\usepackage{hyperref}
\usepackage{cleveref}

\usepackage{tikz}
\usetikzlibrary{arrows, cd, positioning}
\tikzcdset{arrow style=tikz, diagrams={>=stealth'}}

\usepackage{microtype}

\declaretheorem[name=Theorem,numberwithin=section,%
                refname={theorem,theorems},%
                Refname={Theorem,Theorems}]{theorem}
\declaretheorem[name=Definition,numberwithin=section,%
                refname={definition,definitions},
                Refname={Definition,Definitions}]{definition}
\declaretheorem[name=Corollary,sibling=theorem,%
                refname={corollary,corollaries},
                Refname={Corollary,Corollaries}]{corollary}
\declaretheorem[name=Lemma,sibling=theorem,%
                refname={lemma,lemmata},
                Refname={Lemma,Lemmata}]{lemma}
\declaretheorem[name=Example,numberwithin=section,%
                refname={example,examples},
                Refname={Example,Examples}]{example}

% GENERAL MATH
\newcommand{\restrict}[2]{\left.{#1}\right|_{#2}}
% ALGEBRAIC STUFF
\newcommand{\algebra}[1]{\mathbf{#1}}
\newcommand{\algset}[1]{\mathcal{#1}}
\newcommand{\terms}[2]{T({#1}_{#2})}
\newcommand{\worldcr}[3]{#1 \overset{#2}{\rightsquigarrow} #3}
\newcommand{\next}[1]{X{#1}}
\newcommand{\until}[2]{{#1}\,U {#2}}
\newcommand{\forever}[1]{G{#1}}
\newcommand{\eventually}[1]{F{#1}}
\newcommand{\weak}[2]{{#1}\,W {#2}}
\newcommand{\existss}[2]{\exists {#1} \ldotp {#2}}
\newcommand{\foralls}[2]{\forall {#1} \ldotp {#2}}
\newcommand{\sem}[2]{\llbracket #1 \rrbracket_{#2}}
\newcommand{\semtwo}[3]{\llbracket #1 \rrbracket_{#2}^{#3}}
\newcommand{\universe}[1]{\Omega_{#1}}
\newcommand{\universetwo}[2]{\Omega_{#1}^{#2}}


\begin{document}

\chapter{Introduction}
Modal and temporal logics have seen widespread usage in the industry as languages for expressing properties about the
evolution of complex systems, ranging from single programs to distributed multi-agent systems. Researchers have been
developing model checkers with various combinations of temporal logics, e.g. probabilistic temporal logic for satellite
positioning systems~\cite{lu_availability_2014} or communication protocols~\cite{ammar_formal_2011}, spatio-temporal
logic for vehicular movement~\cite{ciancia_spatio-temporal_2018} or image analysis~\cite{vojnar_voxlogica_2019}, and
visual formalisms such as graphs~\cite{fiadeiro_temporal_2007} and automata~\cite{montanari_structured_2005}.

Research efforts have been mainly devoted to propositional flavors of temporal logic with surveys and
monographies~\cite{pnueli_temporal_1977,kroger_temporal_2008}. Investigations on quantified version of temporal logics
are few and mainly focused on finding fragments of such logics, due to negative results about completeness and
decidability, thus favoring efficiency and more importantly computability in lieu of expressivity. Nonetheless there are
relevant works that provide not only axiomatic systems or decidability results with some restriction, but also model
checker and prototypes both on first-order and second-order temporal
logics~\cite{boneva_graph_2007,hutchison_model_2006,baeza-yates_model_2002}.

They still suffer, though, from problem related to the identification of objects across worlds that are not only
relevant from a philosophical perspective, but also from a more pragmatic computer science perspective, since we are
interested in modeling properties with allocations, deallocations and merging of entities. Thus, the so-called
Kripke-semantics make use of universal domains and restrictive rules to tackle at least partially the issue with the
\emph{trans-world identity problem}~\cite{lewis_plurality_2001}. However, even with such modification, Kripke-semantics
are still not well suited for modeling identity between entities and programming concepts such as dynamic allocation and
deallocation of objects.

We introduce a novel semantic for quantified \ac{LTL} in the tradition of counterpart relations, which explicitly relate
elements of different worlds, which are in principle unique to such worlds. Various results have been published on
models and axiomatic systems with counterpart relation and modal
logic~\cite{hutchison_counterpart_2010,belardinelli_quantified_2021}, however we wanted to prove the goodness of such
models with the simpler and more intuitive language of \ac{LTL}, by providing an equally simple semantic that could be
easily followed by hand without the lack of immediate clarity introduced by fixpoint operators in the semantics of modal
languages such as the $\mu$-calculus.

In~\Cref{chap:ltl} we will provide a brief introduction to \ac{LTL} and \ac{FOLTL} syntax and interpretation with a
Kripke-style semantic. In~\Cref{chap:counterpart} we will introduce the multi-sorted model that we will use to evaluate
the truthness of \ac{FOLTL} formulae with our novel approach based on counterpart relations, then we introduce the
adapted syntax and novel semantics, with illustrative examples, models and evaluations. Finally in~\Cref{chap:system},
we introduce an Hilbert-style deductive system that is consistent with our novel semantics, and we will provide
comparisons to other systems in the literature. We will show the correctness of such system and discuss completeness and
decidability, which are in line with previous results. We conclude in~\Cref{chap:future} with some remarks about
improvements and further questions about the model presented in this work, and possible extensions.


\chapter{Linear Temporal Logic}
In the following sections we will shortly summarise both syntax and semantics of linear temporal logic in the
propositional form and show the issue of directly translating it to the quantified first-order version.
Since many combinations of both unary and binary operators have been studied as forms of linear temporal logic, we will
investigate only the somewhat common case of a logic with a singular primitive unary operation, \emph{next}, and a
singularo primitive binary operation, \emph{until}. Nonetheless, both the classical approach and the one presented in
this work can be trivially extended with other operators known in the literature, like \emph{unless}.

\section{Syntax}

\begin{definition}[Linear Temporal Logic]
  Let $\text{PROP}$ be a set of atomic propositions. The set $\mathcal{F}$ of formulae for Linear Temporal Logic is the
  set generated by the following grammar:
  \[
    \phi \Coloneqq tt \;|\; p
                      \;|\; \neg\phi
                      \;|\; \phi \lor \phi
                      \;|\; \nextop{\phi}
                      \;|\; \until{\phi}{\phi}
  \]
  where $p \in \text{PROP}$, $X$ is a unary operator which states that $\phi$ must hold at the next step and $U$ is a
  binary operator which states that the first formula must hold until the second formula holds at some point in the
  current or next steps.
\end{definition}

Classical propositional logic connetives can be derived trivially:
\[
  \phi_1 \land \phi_2 \equiv \neg(\neg\phi_1 \lor \neg\phi_2) \qquad
  \phi_1 \to \phi_2 \equiv \neg\phi_1 \lor \phi_2
\]

Other temporal operators used in temporal logic literature are derivable as such:
\[
  \eventually{\phi} \equiv \until{tt}{\phi} \qquad
  \forever{\phi} \equiv \neg\eventually(\neg\phi) \qquad
  \weak{\phi_1}{\phi_2} \equiv (\until{\phi_1}{\phi_2}) \lor \forever{\phi_1}
\]
Intuitively $\eventually{\phi}$ means that eventually at some next step in time the formula $\phi$ holds,
$\forever{\phi}$ means that the formula $\phi$ holds at each possible next step.

\begin{figure}
  \begin{center}
  \begin{tikzpicture}[> = stealth', shorten > = 2pt, shorten < = 2pt, auto, node distance = 2cm, semithick]
    \tikzstyle{vertex} = [circle, draw = black, thick, fill = white, minimum size = 2mm]
    \node[vertex] (0u) [label=$\phi_1$] {};
    \node[vertex] (1u) [right = 1cm of 0u, label=$\phi_1$] {};
    \node[vertex] (2u) [right = 1cm of 1u, label=$\phi_1$] {};
    \node[vertex] (3u) [right = 1cm of 2u, label=$\phi_2$] {};
    \coordinate[right = 1cm of 3u] (4u);
    \node[left = 1cm of 0u] (until) {$\phi_1\;U\,\phi_2$};

    \path[->] (0u) edge [] (1u);
    \path[->, dotted] (1u) edge [] (2u);
    \path[->] (2u) edge [] (3u);
    \path[->, dotted] (3u) edge [] (4u);

    \node[vertex] (0f) [below = 1cm of 0u] {};
    \node[vertex] (1f) [right = 1cm of 0f] {};
    \node[vertex] (2f) [right = 1cm of 1f, label=$\phi$] {};
    \node[vertex] (3f) [right = 1cm of 2f] {};
    \coordinate[right = 1cm of 3f] (4f);
    \node[left = 1cm of 0f] (eventually) {$F\phi$};

    \path[->] (0f) edge [] (1f);
    \path[->, dotted] (1f) edge [] (2f);
    \path[->] (2f) edge [] (3f);
    \path[->, dotted] (3f) edge [] (4f);

    \node[vertex] (0g) [below = 1cm of 0f, label=$\phi$] {};
    \node[vertex] (1g) [right = 1cm of 0g, label=$\phi$] {};
    \node[vertex] (2g) [right = 1cm of 1g, label=$\phi$] {};
    \node[vertex] (3g) [right = 1cm of 2g, label=$\phi$] {};
    \coordinate[right = 1cm of 3g] (4g);
    \node[left = 1cm of 0g] (forever) {$G\phi$};

    \path[->] (0g) edge [] (1g);
    \path[->, dotted] (1g) edge [] (2g);
    \path[->] (2g) edge [] (3g);
    \path[->, dotted] (3g) edge [] (4g);
  \end{tikzpicture}
  \end{center}
  \caption{Example of temporal traces for operators}
\end{figure}

LTL formulae can encode safety properties, commonly with the form $G\neg\phi$, and liveness properties of systems,
commonly with the form $GF\phi$ or $G(\phi_1 \to F\phi_2)$. Let's assume that we have two concurrent processes and that
we can encode with two atomic propositions, $\text{crit}_1$ and $\text{crit}_2$, when the two processes can access a
critical section, then the property of mutual exclusivity can be model by the formula $G(\neg\text{crit}_1 \lor
\neg\text{crit}_2)$. If we also have atomic propositions for wait conditions, $\text{wait}_1$, then we can model
liveness conditions, for example $G(\text{wait}_1 \to F\text{crit}_1)$ which means that whenever a process reaches a
wait condition, eventually will enter the critical section; we can also model fairness conditions, for example
$GF\text{wait}_1 \to GF\text{crit}_1$ means that if the process infinitely often reaches a wait condition than
infinitely often will reach the critical section.

\section{Semantic}

The de-facto standard semantics for Linear Temporal Logic and Temporal Logics in general are Kripke-style semantics, as
for modal logics. In modal logic, the so-called Kripke-frames are pairs of set of worlds and an accessibility relation
between worlds. For temporal logic, each world is a point in time and the accessibility relation models the flow of
time, thus a temporal model is defined as:

\begin{definition}[Temporal model]
  A \emph{temporal model} $M$ is a triple $(T, \prec, L)$ where $T$ is a set of time-points, $\prec$ is an accessibility
  relations over $T$ and $L : T \to \mathcal{P}(\text{PROP})$ is a labelling function that for each point in time
  returns the subset of all atomic propositions which are valid at that point.
\end{definition}

Additionally, the pair $(T, \prec)$ is called a temporal frame.

\begin{definition}[LTL semantic]
  The truthness of a LTL formula $\phi$ at the time-point $t$ over the temporal model $M$, denoted as
  $M, t \vDash \phi$ is defined inductively as follows:
  \begin{align*}
    M, t &\vDash tt && \\
    M, t &\vDash p &&\text{ iff } p \in L(t) \\
    M, t &\vDash \neg\phi &&\text{ iff } M, t \nvDash \phi \\
    M, t &\vDash \phi_1 \lor \phi_2 &&\text{ iff } M, t \vDash \phi_1 \text{ or } M, t \vDash \phi_2 \\
    M, t &\vDash \nextop{\phi} &&\text{ iff } \foralls{t \prec t'}{M, t' \vDash \phi} \\
    M, t &\vDash \until{\phi_1}{\phi_2} &&\text{ iff } M, t \vDash \phi_2 \text{ or } (M, t \vDash \phi_1 \text{ and }
      \foralls{t \prec t'}{M, t' \vDash \until{\phi_1}{\phi_2}})
  \end{align*}
\end{definition}

\section{First-Order Extension}

LTL can be extended to a first-order calculus by introducing predicates and variables, in the same vein as the
transition from propositional to first-order logic in the classical framework. Naturally, because we are additionally
working with time-dependent components, predicate symbols and variables can be interpreted differently at different
point in time. Symbols that are established to be time-dependent are called \emph{flexible}, conversely time-independent
symbols are called \emph{rigid}.

Nonetheless both classical logic and linear temporal logic have in common the trait of becoming indecidable when passing
from propositional calculi to first-order calculi. Furthermore First-Order Linear Temporal Logic is proven to be also
incomplete.

In the following paragraphs we will show a general syntax and semantics of First-Order Linear Temporal Logic, what are
the condition for completeness and some of the issue with the most common semantic.

\begin{definition}[First-Order Linear Temporal Logic]\label{def:kripkefoltlsyn}
  Let $\mathcal{P}$ be a set of predicates each with a specific arity, and let $X$ be a denumerable set of variables.
  The set $\mathcal{F}_{FO}$ of formulae for First-Order Linear Temporal Logic is the set generated by the following grammar:
  \[
    \phi \Coloneqq tt \;|\; P(x_1, \ldots, x_n)
                      \;|\; \neg\phi
                      \;|\; \phi \lor \phi
                      \;|\; \nextop{\phi}
                      \;|\; \until{\phi}{\phi}
                      \;|\; \existss{x}{\phi}
  \]
  where $P \in \mathcal{P}$ and $x_1, \ldots, x_n \in X$ are individual variables that match the arity required by $P$.
\end{definition}
Universal quantification can be modeled with the translation from the existential quantifier and negation:
\[
  \foralls{x}{\phi} \equiv \neg\existss{x}{\neg\phi}
\]

Example of properties can be modeled with the first-order extension are:
\[
  \existss{x}{G(\text{channel}(x) \land \text{open}(x))}
\]
with the intuitive meaning that there must always exists at least one open channel; or:
\[
  \foralls{x}{G(\text{channel}(x) \to (\until{\text{open}(x)}{(\neg\existss{y}{\text{message}(y) \land \text{pending}(x, y)})}))}
\]
which intuitively means that channels must always remain open until there are no more pending messages on that channel.

For quantified LTL we need to extend the temporal frames with the domain of values that makes sense to talk about at
each points in time.

\begin{definition}[Kripke frame]
  A \emph{Kripke frame} $M$ is a quadruple $(T, \prec, D, d)$ where $T$ is a set of time-points, $\prec$ is an
  accessibility relation over $T$, $D$ is a function assigning to each point in time $t$ a non-empty set $D(t)$ s.t.
  if $t \prec t'$ then $D(t) \subseteq D(t')$, $d$ is a function assigning to each point in time $t$ a set $d(t)
  \subseteq D(t)$.
\end{definition}
Intuitively, the so-called outer domains $D(t)$ represent the collection of object that can be referenced at the point
in time $t$, conversely the so-called inner domains $d(t)$ represent the collection of object actually existing the
point in time $t$. An assignement function $\sigma$ is a function from the set of variables $X$ to the outer-domain
$D(t)$ of a given point in time $t$. Finally, we need an interpretation function $I$ such that $I(P, t)$ assign for each
predicate constant $P \in \mathcal{P}$ and point in time $t$ a subset of $D^n(t)$ where $n$ is the arity of $P$, and
$I(x, t) = \sigma(x)$ where $x \in X$ and $\sigma$ is a suitable assignment that is said to be inducing $I$, noted as
$I^\sigma$.

\begin{definition}[FO-LTL semantic]
  The truthness of a FO-LTL formula $\phi$ at the point in time $t$ over the Kripke model $M$ and the induced
  interpretation $I^\sigma$, denoted as $M, t, I^\sigma \vDash \phi$ is defined inductively as follows:
  \begin{align*}
    M, t, I^\sigma &\vDash tt && \\
    M, t, I^\sigma &\vDash P(x_1, \ldots, x_n) &&\text{iff } (\sigma(x_1), \ldots, \sigma(x_2)) \in I^\sigma(P, t) \\
    M, t, I^\sigma &\vDash \neg\phi &&\text{iff } M, t, I^\sigma \nvDash \phi \\
    M, t, I^\sigma &\vDash \phi_1 \lor \phi_2 &&\text{iff } M, t, I^\sigma \vDash \phi_1 \text{ or } M, t, I^\sigma
      \vDash \phi_2 \\
    M, t, I^\sigma &\vDash \nextop{\phi} &&\text{iff } \foralls{t \prec t'}{M, t', I^\sigma \vDash \phi} \\
    M, t, I^\sigma &\vDash \until{\phi_1}{\phi_2} &&\text{iff } M, t, I^\sigma \vDash \phi_2 \text{ or }
      (M, t, I^\sigma \vDash \phi_1 \text{ and } M, t', I^\sigma \vDash \until{\phi_1}{\phi_2}) \\
    M, t, I^\sigma &\vDash \existss{x}{\phi} &&\text{iff } \existss{v \in d(t)}{M, t, I^{\sigma[v/x]} \vDash \phi} \\
  \end{align*}
\end{definition}

\section{Trans-World Identity}

Kripke-frames requires that the so-called outer domains $D(t)$ are always increasing with time, i.e. if $t \prec t'$
then $D(t) \subseteq D(t')$. This condition is required to evaluate temporal operators, otherwise it would not be
possible to denote a variable $x$ in the future. Other works fix a universal domain equal for each point in time, thus
$D(t) = D(t')$ for each $t, t' \in T$, which also follows from accessibility relations that contain loops.
However, by imposing such condition we are identifying object \emph{a priori}
and universally, irrespective of time. This problem is called \emph{trans-world identity} and extensive literature about
it has been produced in the last half-century, be either philosophycal questions or possible pratical solutions with
Kripke-style semantics.

Even with the additional constraints on object domains, there are still other issues with Kripke-style semantics and the
system we are trying to model. Assume we are trying to model resource allocation in a complex system. Let $i$ be a
resource. Due to the outer domain condition, we can identify the exact point in time where the resource $i$ is
allocated, i.e. $i \in D(t)$ for some time point $t$. Note that the resource can be allocated, but still can be unused
for a potentially infinite amount of time, since it may not belong to any inner domain, not even $d(t)$. However, at the
same time, we cannot deallocate the resource $i$ since it must always be referentiable at future point in times. This
can be solved with infinite outer domains to ensure uniqueness or by restricting the class of admissible evolutions, but
such solutions tend to hamper usability.

Another desiderable behaviour is merging, for example while modeling memory allocations, we may want to merge two memory
allocated segments into a single segment and treat it as a single resource.

In the next chapter we will explore an alternative approach based on \emph{counterpart relations}, which rejects the
possibility of universally identifying objects among possible worlds.


\chapter{Counterpart Semantic}
In this chapter we will introduce both syntax and semantics of the calculus, however we will use many-sorted algebras as
models for the evaluation of the formulae.
\section{Algebra}

\begin{definition}[Many-sorted Signature]
  A \emph{many-sorted signature} $\Sigma$ is a pair $(S_\Sigma, F_\Sigma)$ where $S_\Sigma = \set{\tau_1, \ldots,
  \tau_n}$ is a set of sorts, and $F_\Sigma = \set{f_\Sigma : \tau_1 \times \cdots \times \tau_n \to \tau | \tau_i, \tau
  \in \Sigma_\tau, i = 1, \ldots, n}$ is a set of function symbols.
\end{definition}

\begin{definition}[Many-sorted Algebra]
  A \emph{many-sorted algebra} $\algebra{A}$ with signature $\Sigma$, or $\Sigma$-algebra, is a pair $(A,
  F_\Sigma^\algebra{A})$ such that:
  \begin{itemize}
    \item $A$ is a family of carrier sets indexed by the sorts of $\Sigma$;
    \item $F_\Sigma^\algebra{A}$ is a family of functions indexed by the function symbols of $\Sigma$,
      $\set{f_\Sigma^\algebra{A} : A_{\tau_1} \times \cdots \times A_{\tau_n} \to A_\tau | f_\Sigma : \tau_1 \times
      \cdots \times \tau_n \to \tau \in F_\Sigma}$.
  \end{itemize}
\end{definition}

\begin{definition}[Homomorphism]
  Given two $\Sigma$-algebras $\algebra{A}$ and $\algebra{B}$, a \emph{(partial)
  homomorphism} is a family of (partial) functions indexed by the sorts of $\Sigma$, $\set{\rho_\tau : A_\tau
  \rightharpoonup B_\tau | \tau \in S_\Sigma}$, such that for each function symbol $f_\Sigma : \tau_1 \times \cdots \times
  \tau_n \in F_\Sigma$ and list of elements $a_1 \in A_{\tau_1}, \ldots, a_n \in A_{\tau_n}$, if each function
  $\rho_{\tau_i}$ is defined for the element $a_i$ then $\rho_\tau$ is defined for the element $f_\Sigma^\algebra{A}(a_1,
  \ldots, a_n)$ and $\rho_\tau(f_\Sigma^\algebra{A}(a_1, \ldots, a_n)) = f_\Sigma^\algebra{B}(\rho_{\tau_1}(a_1),
  \ldots, \rho_{\tau_n}(a_n))$.
\end{definition}

\begin{example}[Graph algebra]
  Simple directed graphs can be modeled by the signature $\Sigma = (\set{\tau_v, \tau_e}, \set{s : \tau_e \to \tau_v, t
  : \tau_e \to \tau_v})$, where $\tau_v$ is the sort of vertices, $\tau_e$ is the sort of edges and $s$, $t$ determine
  respectively the source and target vertex for a given edge. Each $\Sigma$-algebra for this signature is a particular
  graph, i.e. the graphs in figure 1 are visual representations of the following algebras: $\algebra{G}_0 =
  (\set{\set{n_0, n_1, n_2},\set{e_0, e_1, e_2}}, \set{s^{\algebra{G}_0}, t^{\algebra{G}_0}})$, where $s^{\algebra{G}_0} =
  \set{e_0 \mapsto n_0, e_1 \mapsto n_1, e_2 \mapsto n_2}$ and $t^{\algebra{G}_0} = \set{e_0 \mapsto n_1, e_1 \mapsto
  n_2, e_2 \mapsto n_0}$.
\end{example}

To model existential properties, we can extend a signature $\Sigma$ with a denumerable set $X$ of variables typed over
$S_\Sigma$, obtaining the signature $\Sigma_X$. The $\tau$-typed subset of $X$ is denoted with $X_\tau$, and typed
variables are denoted with $x_\tau$ or $x : \tau$. $\tau$-sorted terms are denoted with $\epsilon_\tau$ or $\epsilon :
\tau$.

\begin{definition}[Term]
  Let $\Sigma$ be a signature, let $X$ be a denumerable set of individual variables typed over $S_\Sigma$, and let
  $\Sigma_X$ be the signature obtained by extending $\Sigma$ with $X$. The (many-sorted) set $\terms{\Sigma}{X}$ of
  \emph{terms} is the smallest set such that:
  \[
    \begin{prooftree}
      \hypo{\phantom{[1,n]}} % ALIGNMENT ONLY
      \infer1{X \subseteq \terms{\Sigma}{X}}
    \end{prooftree}
    \qquad
    \begin{prooftree}
      \hypo{f : \tau_1 \times \cdots \tau_n \to \tau \in F_\Sigma}
      \hypo{\forall i \in [1,n] \ldotp \epsilon_i : \tau_i \in \terms{\Sigma}{X}}
      \infer2{f(\epsilon_1, \ldots, \epsilon_n) : \tau \in \terms{\Sigma}{X}}
    \end{prooftree}
  \]
\end{definition}

\begin{example}[Terms]
  Let be $\Sigma = (\set{\tau}, \set{1 : \tau, \otimes : \tau \to \tau})$ a monoidal signature with a single sort. Let
  be $X$ a denumerable set of variables with $x, y, z \in X$, then some example of terms in
  $\terms{\Sigma}{X}$ are $1 \otimes 1 : \tau$, $x \otimes y \otimes 1$ and $(x \otimes 1) \otimes (1
  \otimes y)$. Examples from the previously defined graph algebra $\Sigma$ are $s(x_{\tau_e})$ and $t(y_{\tau_e})$ which
  represent respectively the source vertex for an edge $x_{\tau_e}$ and the target vertex for an edge $y_{\tau_e}$.
\end{example}

\section{Counterpart model}

We now introduce the concept of \emph{counterpart model}, as in ``counterpart theory'' of David
Lewis~\cite{lewis_counterpart_1968,lewis_plurality_2001}.

\begin{definition}
Let $\Sigma$ be a signature, and $\algset{A}$ the set of algebras over the signature $\Sigma$. A \emph{counterpart
model} $M$ is a triple $(W, \rightsquigarrow, d)$ such that:
\begin{itemize}
  \item $W$ is a set representing worlds;
  \item $d : W \to \algset{A}$ is a function, assigning to each world $w \in W$ a $\Sigma$-algebra, $\algebra{A} \in
    \algset{A}$;
  \item $\rightsquigarrow\;\subseteq W \times (\algset{A} \rightharpoonup \algset{A}) \times W$ is the
    \emph{accessibility relation} over $W$, enriched with (partial) homomorphisms \emph{(counterpart relations)} between
  the algebras of the connected worlds, i.e. for every $(w_1, cr, w_2) \in\;\rightsquigarrow$ it must hold that $cr :
  d(w_1) \rightharpoonup d(w_2)$ is a (partial) homomorphism.
\end{itemize}
\end{definition}

As a shorthand notation, counterpart relations between two worlds in the accessibility relation, $(w_1, cr, w_2) \in
\;\rightsquigarrow$, will be also denoted as $\worldcr{w_1}{cr}{w_2}$.  The accessibility relation $\rightsquigarrow$
defines the counterparts in the target world of the source world, effectively modeling the evolution of a system,
modeled by the algebras, and avoiding the \emph{trans-world identity} problem. Names are local to the belonging world,
and components are identified across worlds only by the relation between names instead of by a universal name. This, and
partiality, allows for creation, deletion, renaming and merging of elements in a type-preserving manner, however
duplication is not permitted as counterpart relations are functions thus can only associate a single element of the
target world to an element of the source world.  In other terminology, the counterpart model can be seen as a
generalisation of a \emph{graph transition system}~\cite{fiadeiro_temporal_2007} where transition are labelled with
homomorphism between algebras, which are the state of the transition system and can be arbitrarily complex within
themselves. Worlds can be considered points in time when we take the traces of these transition systems, the application
of the partial homomorphism from each world to the next models the transition in the system and thus the evolution of
time.

For the rest of the work, we will implicitly assume that every counterpart model $M$ has no deadlock worlds, i.e.
worlds without outgoing transitions. This condition is not a limitation since a counterpart model that does not satisfy
the condition can be transformed into one that does satisfy it by adding a reflexive transition to the accessibility
relation, i.e. for each $w$ that is a deadlock world, $(w, \text{id}_w, w) \in\;\rightsquigarrow$, or an ad-hoc world
with only a reflexive transition. Such modification is typical~\cite{baier_principles_2008} of other works in the field,
it does not impact the results and it largely simplifies the semantics presented here.

\begin{definition}
  Let $X$ be a denumerable sets of variables, and $M = (W, \rightsquigarrow, d)$ be a counterpart model over a signature
  $\Sigma$. A \emph{variable assignment} $\sigma$ for a world $w \in W$ is a (partial) function such that $\sigma : X
  \rightharpoonup d(w)$.
\end{definition}

Given a term $\epsilon \in \terms{\Sigma}{X}$ and an assignment $\sigma$, we will denote as
$\sigma(\epsilon)$ the lifting of $\sigma$ to the set $\terms{\Sigma}{X}$, i.e. applying the substition $\sigma$ to each
free variable in the term $\epsilon$. If any of the free variables in $\epsilon$ are not in the domain of $\sigma$ than $\sigma(\epsilon)$ is undefined.

From now on, we will consider only \emph{formulae-in-context}, i.e. formulae with an appropriate context $\Gamma$ that
must contain at least all the free variables in the formula. Consequently, substitutions for a formula $\phi$ in a
context $\Gamma$ will be defined over $\Gamma$. Note that the substitution can still be undefined over some
or all the values in the domain of $\Gamma$, addressing the need of modeling deallocation of items.

\section{Syntax}

\begin{definition}[First-Order Linear Temporal Logic]
Let $\Sigma$ be a (multi-sorted) signature and $X$ a denumerable set of variables typed over $S_\Sigma$. The set
$\mathcal{F}_\Sigma$ of formulae for First-Order Linear Temporal Logic is the set generated by the following grammar:
\[
  \phi \Coloneqq tt \;|\; \epsilon_\tau = \epsilon_\tau
                    \;|\; \neg\phi
                    \;|\; \phi \lor \phi
                    \;|\; \existss{x_\tau}{\phi}
                    \;|\; \nextop{\phi}
                    \;|\; \until{\phi}{\phi}
\]
where $\epsilon \in \terms{\Sigma}{X}$, $\exists x_\tau$ ranges over variables of sort $\tau \in S_\Sigma$,
$O$ is a unary operator which states that $\phi$ must hold at the next step and $U$ is a binary operator which states
that the first formula must hold until the second formula holds at some point in the current or next steps.
\end{definition}

Classical propositional logic and other temporal operators can be derived as for LTL, with the addition of the trivially
derivable universal quantifier, $\foralls{x_\tau}{\phi} \equiv \neg\existss{x_\tau}{\neg\phi}$. We will also define a
shorthand for inequality, $x_\tau \neq \epsilon_\tau \equiv \neg(x_\tau = \epsilon_\tau)$.

The syntax is almost identical to the one introduced in~\autoref{def:kripkefoltlsyn}, but predicates are replaced by
equality of terms, which we will further specify in the next section, and the existential quantifier is instead a family
of operators indexed by the sorts of the signature.

\begin{example}
  Let's model some general properties that can be resonable in a large space of signatures, as a consequence these will
  really be a family of typed predicates. We will start with simple predicates that exemplifies some of the feature of
  such a language in isolation: we first need to identify entities which are available in a given world,
  $\text{presence}(x_\tau) \equiv \existss{y_\tau}{x = y}$, then we can construct simple temporal formulae, for example
  we can require that an entity is always present $\forever{\text{presence}(x_\tau)}$ or that it will be deleted in the next
  step, $\text{will-delete}(x_\tau) \equiv \text{presence}(x_\tau) \land \nextop{\neg\text{presence}(x_\tau)}$ or at some
  point in the future, never to be allocated again: $\text{perm-deallocate}(x_\tau) \equiv \text{presence}(x_\tau) \land \eventually{\forever{\neg\text{presence}(x_\tau)}}$
  Conversely to deallocation, we can model allocations and creation of entities by reversing the deletion formula
  $\text{one-new}(x_\tau) \equiv \neg\text{presence}(x_\tau) \land \nextop{\neg\text{presence}(x_\tau)}$. For example
  we can require entities to be always allocated and then deallocated in a managed memory context:
  $\forever{\foralls{x_\tau}{\eventually{\text{one-new}(x) \land \eventually{\text{perm-deallocate}(x_\tau)}}}}$.
  Another example is $\text{will-merge}(x_\tau, y_\tau) \equiv \until{(presence(x) \land presence(y) \land x \neq y)}{(presence(x) \land presence(y) \land x = y)}$, which intuitively means that the arguments of
  the predicate will necessarily merge into a single entity in the current world or at some point in the future.

  Recall the signature for simple directed graphs introduced in~\Cref{sec:algebra}, we can define predicates on the
  structure of the worlds, e.g. $\text{loop}(x_{\tau_e}) \equiv s(x) = t(x)$ which means that the edge $x$ must be a
  loop in the worlds that satisfy the predicate. We can also model the evolution of complex structures, e.g.
  $\text{at-most-2}_\tau \equiv \foralls{x_\tau}{\foralls{y_\tau}{\foralls{z_\tau}{x = y \lor y = z \lor z = x}}}$ is
  true for worlds where the number of entities of sort $\tau$ is bounded by 2. This predicate can be extended to larger
  boundaries or enforced during the evolution of the system, e.g. assume that the graph signature is used to model a net
  of unidirectional inter-process communication channel, we may want to verify that the number of channel, or edges,
  spawned during the whole execution is bounded by some number $n$ than we must verify the formula
  $\forever{\text{at-most-n}}$; or we may want to make sure that after reaching maximum capacity, the programs
  continously deallocate channels until a minimum bound $m$ is reached, thus $\forever{(\text{at-most-n} \land (\text{exactly-n} \to
  \until{(\existss{x}{\text{will-delete}(x)})}{\text{at-most-m}}))}$, where $\text{exactly-n}$ is a predicate that
  is true when there are exactly $n$ distinct entities of a given sort.
\end{example}

\section{Semantics}

\begin{definition}[Semantic of FO-LTL]
Let $M$ be a counterpart model, $\Gamma$ a context and $(\sigma, w)$ a pair substitution-world in the context $\Gamma$.
The validity of a formula-in-context $\phi$, denoted $\sem{\sigma}{w}{\Gamma}{\phi}$, is defined inductively as follows:
\begin{align*}
  (\sigma, w) &\vDash_\Gamma tt \\
  (\sigma, w) &\vDash_\Gamma \epsilon_\tau = \eta_\tau &&\text{ iff } \sigma(\epsilon) = \sigma(\eta) \\
  (\sigma, w) &\vDash_\Gamma \neg\phi &&\text{ iff } \semn{\sigma}{w}{\Gamma}{\phi} \\
  (\sigma, w) &\vDash_\Gamma \phi_1 \lor \phi_2
      &&\text{ iff } \sem{\sigma}{w}{\Gamma}{\phi_1} \text{ or } \sem{\sigma}{w}{\Gamma}{\phi_2} \\
  (\sigma, w) &\vDash_\Gamma \existss{x_\tau}{\phi}
      &&\text{ iff } \existss{v \in d(w)}{\sem{\text{ext}_x(\sigma, v)}{w}{\Gamma, x}{\phi}} \\
  (\sigma, w) &\vDash_\Gamma \nextop{\phi}
      &&\text{ iff } \foralls{\worldcr{w}{cr}{w'}}{\sem{cr \circ \sigma}{w'}{\Gamma}{\phi}} \\
  (\sigma, w) &\vDash_\Gamma \until{\phi_1}{\phi_2}
      &&\text{ iff } \sem{\sigma}{w}{\Gamma}{\phi_2} \text{ or } (\sem{\sigma}{w}{\Gamma}{\phi_1}
          \text { and } \foralls{\worldcr{w}{cr}{w'}}{\sem{cr \circ \sigma}{w'}{\Gamma}{\until{\phi_1}{\phi_2}}}) \\
\end{align*}
  Where $\text{ext}_x : (\Gamma \rightharpoonup d(w)) \times d(w) \to (\Gamma \cup \set{x} \rightharpoonup d(w))$ with
  $x \not\in \Gamma$, defined as:
  \[
    \text{ext}_x(\sigma, v)(y) = \begin{cases} v & \text{if } y = x \\ \sigma(y) & \text{otherwise}\end{cases}
  \]
\end{definition}

The formula $tt$ holds for any possible pair assignement-world. The predicate $=$ models equality for typed terms, where
$\epsilon_\tau = \eta_\tau$ is valid if the evaluation is either undefined for both terms or they are equal according to
the standard notion of equality. The negation of a formula is satisfied if the formula without the negation is not
valid. The disjunction of two formulae is valid if either of them is valid. Existentially quantified formulae are valid
if the formula is valid in the context extended with the quantified variable. The sub-formula must be valid in the same
world and with the assignment extended to the context $\Gamma, x$ with a new value within the domain of the algebra,
thus behaving identically on all the variables in the original context $\Gamma$. Next we look at temporal operators.
Formulae containing the next operator are valid if the sub-formula is valid in all the world accessible from $w$. The
sub-formula is evaluated with the assignement composed with the counterpart relation between the accessible worlds.
Formulae with the until operator, instead, are satisfied either if the post-condition is valid in the current world or 
if the pre-condition is valid and the whole formula is valid in all the accessible world, the same as for the next
operator.

\begin{definition}
  A formula $\phi \in \mathcal{F}_\Sigma$ is called \emph{valid for the counterpart model $M$}, denoted by $\vDash_M
  \phi$, if $\sem{\sigma}{w}{\Gamma}{\phi}$ for every context $\Gamma$, world $w$ and substitution $\sigma$.
\end{definition}
\begin{definition}
  A formula $\phi \in \mathcal{F}_\Sigma$ is called a \emph{consequence} of a set $F \subseteq \mathcal{F}_\Sigma$,
  denoted $F \vDash \phi$, if $\vDash_M \phi$ holds for every model $M$ with $\vDash_M \psi$ for all $\psi \in F$.
\begin{definition}
\end{definition}
  A formula $\phi \in \mathcal{F}_\Sigma$ is called \emph{valid}, denoted by $\vDash \phi$, if $\emptyset \vDash \phi$
  holds.
\end{definition}

\section{Other operators}
Let's explore the semantics of the other common operators defined in the previous chapters.

\begin{align}
  \sem{\sigma}{w}{\Gamma}{\phi_1 \land \phi_2}
    &\Leftrightarrow \semn{\sigma}{w}{\Gamma}{\neg\phi_1 \lor \neg\phi_2} \notag \\
    &\Leftrightarrow \semn{\sigma}{w}{\Gamma}{\neg\phi_1} \text{ and } \semn{\sigma}{w}{\Gamma}{\neg\phi_2} \notag \\
    &\Leftrightarrow \sem{\sigma}{w}{\Gamma}{\phi_1} \text{ and } \sem{\sigma}{w}{\Gamma}{\phi_2} \label{eq:andsem} \\[1ex]
  \sem{\sigma}{w}{\Gamma}{\phi_1 \to \phi_2}
    &\Leftrightarrow \sem{\sigma}{w}{\Gamma}{\neg\phi_1 \lor \phi_2} \notag \\
    &\Leftrightarrow \semn{\sigma}{w}{\Gamma}{\neg\phi_1} \text{ or } \sem{\sigma}{w}{\Gamma}{\phi_2} \notag \\
    &\Leftrightarrow \sem{\sigma}{w}{\Gamma}{\phi_1} \text{ implies } \sem{\sigma}{w}{\Gamma}{\phi_2} \label{eq:impliessem} \\[1ex]
  \sem{\sigma}{w}{\Gamma}{\phi_1 \leftrightarrow \phi_2}
    &\Leftrightarrow (\sem{\sigma}{w}{\Gamma}{\phi_1 \to \phi_2}) \text{ and } (\sem{\sigma}{w}{\Gamma}{\phi_2 \to \phi_1}) \notag \\
    &\Leftrightarrow (\sem{\sigma}{w}{\Gamma}{\phi_1} \text{ implies } \sem{\sigma}{w}{\Gamma}{\phi_2}) \notag \\
        &\qquad\quad\text{and } (\sem{\sigma}{w}{\Gamma}{\phi_2} \text{ implies } \sem{\sigma}{w}{\Gamma}{\phi_1}) \notag \\
    &\Leftrightarrow \sem{\sigma}{w}{\Gamma}{\phi_1} \text{ iff } \sem{\sigma}{w}{\Gamma}{\phi_2} \label{eq:iffsem} \\[1ex]
  \sem{\sigma}{w}{\Gamma}{\foralls{x}{\phi}}
    &\Leftrightarrow \semn{\sigma}{w}{\Gamma}{\existss{x}{\neg\phi}} \notag \\
    &\Leftrightarrow \nexists v \in d(w) \ldotp \semn{\sigma[v/x]}{w}{\Gamma}{\phi} \notag \\
    &\Leftrightarrow \foralls{v \in d(w)}{\sem{\sigma[v/x]}{w}{\Gamma}{\phi}} \label{eq:forallsem} \\[1ex]
  \sem{\sigma}{w}{\Gamma}{\eventually{\phi}}
    &\Leftrightarrow \sem{\sigma}{w}{\Gamma}{\until{tt}{\phi}} \notag \\
    &\Leftrightarrow \sem{\sigma}{w}{\Gamma}{\phi} \text{ or }
        (\sem{\sigma}{w}{\Gamma}{tt} \text { and } \foralls{\worldcr{w}{cr}{w'}}{\sem{cr \circ \sigma}{w'}{\Gamma}{\until{tt}{\phi}}}) \notag \\
    &\Leftrightarrow \sem{\sigma}{w}{\Gamma}{\phi} \text{ or }
        \foralls{\worldcr{w}{cr}{w'}}{\sem{cr \circ \sigma}{w'}{\Gamma}{\until{tt}{\phi}}} \notag \\
    &\Leftrightarrow \sem{\sigma}{w}{\Gamma}{\phi} \text{ or } \sem{\sigma}{w}{\Gamma}{\nextop{\eventually{\phi}}} \notag \\
    &\Leftrightarrow \sem{\sigma}{w}{\Gamma}{\phi} \lor \nextop{\eventually{\phi}} \label{eq:eventuallysem} \\[1ex]
  \sem{\sigma}{w}{\Gamma}{\forever{\phi}}
    &\Leftrightarrow \sem{\sigma}{w}{\Gamma}{\neg\eventually{\neg\phi}} \notag \\
    &\Leftrightarrow \sem{\sigma}{w}{\Gamma}{\sem{\sigma}{w}{\Gamma}{\phi} \text{ and }
        \neg\foralls{\worldcr{w}{cr}{w'}}{\sem{cr \circ \sigma}{w'}{\Gamma}{\neg\phi}}} \notag \\
    &\Leftrightarrow \sem{\sigma}{w}{\Gamma}{\sem{\sigma}{w}{\Gamma}{\phi} \text{ and }
        \existss{\worldcr{w}{cr}{w'}}{\sem{cr \circ \sigma}{w'}{\Gamma}{\phi}}} \notag \\
    &\Leftarrow \sem{\sigma}{w}{\Gamma}{\phi \land \nextop{\forever{\phi}}} \label{eq:foreversem}
\end{align}

\section{Example of execution}

\begin{figure}
  \begin{center}
  \begin{tikzpicture}[scale=0.25,>=stealth']
    \MemoryLayout{A,B,C}
    \node[] () at (0, 0) {$w_0$};
    \begin{scope}[shift={(24cm,0cm)}]
      \MemoryLayout{A,B,C,D}
      \node[] () at (0, 0) {$w_1$};
    \end{scope}
    \begin{scope}[shift={(0cm,-10cm)}]
      \MemoryLayout{A,B,C}
      \node[] () at (0, 0) {$w_2$};
    \end{scope}
    \begin{scope}[shift={(24cm,-10cm)}]
      \MemoryLayout{A,B,E,C}
      \node[] () at (0, 0) {$w_3$};
    \end{scope}
    \begin{scope}[shift={(0cm,-20cm)}]
      \MemoryLayout{A,B,C,E}
      \node[] () at (0, 0) {$w_4$};
    \end{scope}
    \draw[->] (20cm,0) -- ++(2cm,0);
    \draw[->] (20cm,-10cm) -- ++(2cm,0);
    \draw[->] (9cm,-4cm) -- ++(0,-2cm);
    \draw[->] (33cm,-4cm) -- ++(0,-2cm);
    \draw[->] (9cm,-14cm) -- ++(0,-2cm);
    \draw[->] (28cm,-14cm) -- ++(-2cm,-2cm);
  \end{tikzpicture}
  \end{center}
  \caption{Example of counterpart model for a memory layout signature.}
  \label{fig:exmemsignature}
\end{figure}

Here we will provide the evaluation of some formulae, using some of the predicates introduced earlier in the chapter.
Assume we are evaluating according to the counterpart model visually represented by~\Cref{fig:exmemsignature}, with
the signature being the idealised memory model in~\Cref{sec:algebra} and the counterpart relations acting as the
identity on the objects sort, and on memory location sort as visually represented.
Let's evaluate the formula $\phi_1 \equiv \existss{x_{\tau_o}}{\existss{l_{\tau_l}}{\rho(x) = l \land \nextop{(\rho(x)
\neq l)}}}$, intuitively it identifies worlds where there is at least one object that is moved in memory after one step
of computation. Because this formula has no free variables it will be evaluated in the empty context, thus the only
admissible substitution is the empty substitution, which we will denote with $\bot$.
First, by virtue of the definition of the existential quantifier, we need to evaluate the conjunction in the empty
context extended by the two quantified variables, $\sem{\sigma}{w}{x,l}{\rho(x) = l \land \nextop{(\rho(x) \neq l)}}$.
The left-hand side of the conjunction is valid in all worlds when both variables are evaluated to be undefined or if
both are defined and the location is exactly the one represented visually in the figure, e.g. in world $w_0$, with
the assignement $\sigma$ such that $\sigma(x) = A$, $\sigma(l) = 0$ it follows that $\sem{\sigma}{w}{x,l}{\rho(x) = l}$.
For the right-hand side we need to check the accessibility relation of the model. The pairs for which the formula
$\rho(x) \neq l$ is valid are exactly the complement of the left-hand side, therefore we need to check which pair evolve
due to the counterpart relations into one of those, for each possible accessible world. All the assignements where only
one of the two variables are undefined, evolve via the counterpart relations into one in this set, however they are not
interesting as they cannot satisfy the conjuction. There are also pair for which the sub-formula inside the next
operator is valid in one trasition but not all of them, an example is $(\sigma, w_2)$ with $\sigma$ such that $\sigma(x) = C$,
$\sigma(l) = 2$. After applying the counterpart relation $\worldcr{w_2}{cr}{w_3}$, $(cr \circ \sigma)(l) = 3$ that
satisfies the inequality $\rho(C) \neq 3$, however with the counterpart relation $\worldcr{w_2}{cr}{w_4}$,
$(cr \circ sigma)(l) = 2$ does not satisfy the inequality, thus $\semn{\sigma}{w_2}{x,l}{\nextop{(\rho(x) \neq l)}}$.
An example of pair that, instead, makes the whole conjuction valid is $(\sigma, w_3)$ with $\sigma$ such that $\sigma(x)
= E$,$\sigma(l) = 2$. Readers may notice that there is still the assignement always undefined for which equalities are
always trivially sastisfied, however, the existential quantifier requires that assignments are extended with defined
values, thus ultimately matching our intuition for the semantic of the formula. As noted before, the final pairs must be
composed of empty substitution, therefore the only pairs for which the formula $\phi_1$ is valid are $(\bot, w_1)$ and
$(\bot, w_3)$.

If we wanted to recover the information about the specific object and/or location that moved with time, we need to evaluate open
formulae, thus we need to remove one or both existential quantification. However, since there is no quantification that
avoids undefined values that trivially satisfy equalities, we need a presence predicate, i.e. $\text{presence}(x_\tau) =
\existss{y_\tau}{x = y}$, which is satisfiable only if $y$ is equal to $x$ and $y$ must be defined, therefore
also $x$ must be defined.


\chapter{Proof System}
\section{Axioms}

The following definition will introduce a Hilbert-style deductive system for \ac{FOLTL}. Formally, the formal system is
constituted by a set of formulae called \emph{axioms}, and a set of rules called \emph{inference rules}, of the form
$\phi_1, \ldots, \phi_n \vdash \psi$. The formulae in the non-empty set $\phi_1, \ldots, \phi_n$ are called
\emph{premises}, the formula $\psi$ is called \emph{conclusion}. The relation $\vdash$ defines the derivability of a
formula, given a set of formulae $F$ the derivability relation is defined inductively as follows:
\begin{enumerate}
  \item $F \vdash \psi$ if $\psi$ is in the set of axioms;
  \item $F \vdash \psi$ if $\psi \in F$;
  \item $F \vdash \psi$ if $\psi$ is a conclusion of an inference rule and for every premise $\phi$ of the rule it must
    hold that $F \vdash \phi$.
\end{enumerate}
A formula $\psi$ is called \emph{derivable}, denoted $\vdash \psi$, if $\emptyset \vdash \psi$.

A requirement for any reasonable formal system is the so-called \emph{soundness}, i.e. the derivability of a formula in
the formal system implies the validity of the formula as defined in~\Cref{eq:valid}:
\[
  F \vdash \phi \Rightarrow F \vDash \phi
\]
Some systems model logics, e.g. propositional calculus, that allows for the converse relation, the so-called \emph{completeness}, i.e. if a formula is valid than there must exist a derivation:
\[
  F \vDash \phi \Rightarrow F \vdash \phi
\]

The axioms and inference rules follow the pre-existing tradition on finitary axiomatic systems for
\ac{FOLTL}~\cite{burgess_axioms_1982,xu_us-tense_1988}, with one notable and somewhat controversial missing axiom that
will be discussed in the following sections.

\begin{definition}\label{def:axioms}
  The Hilbert-style deductive system for \ac{FOLTL} contains the following axioms:
  \begin{enumerate}
    \item $\phi_1 \to (\phi_2 \to \phi_1)$;
    \item $(\phi_1 \to (\phi_2 \to \phi_3)) \to ((\phi_1 \to \phi_2) \to (\phi_1 \to \phi_3))$;
    \item $(\neg\phi_1 \to \neg\phi_2) \to (\phi_2 \to \phi_1)$;
    \item $\neg\nextop{\phi} \leftrightarrow \nextop{\neg\phi}$;
    \item $\nextop{(\phi_1 \to \phi_2)} \to (\nextop{\phi_1} \to \nextop{\phi_2})$;
    \item $\phi \to \nextop{\phi}$ if $\phi$ is rigid;
    \item $\until{\phi_1}{\phi_2} \leftrightarrow \phi_2 \lor (\phi_1 \land \nextop{(\until{\phi_1}{\phi_2})})$;
    \item $\until{\phi_1}{\phi_2} \to \eventually{\phi_2}$;
    \item $\phi[v/x] \to \existss{x}{\phi}$, with $x$ free in $\phi$ and $v$ a value in the domain;
    \item $x = x$;
    \item $x = y \to (\phi \to \phi[y/x])$.
  \end{enumerate}
  And the following inference rules:
  \begin{enumerate}
    \item[mp] $\phi_1, \phi_1 \to \phi_2 \vdash \phi_2$;
    \item[nex] $\phi \vdash \nextop{\phi}$;
    \item[ind] $\phi_1 \to \phi_3 \lor (\phi_2 \land \nextop{\phi_1}) \vdash \phi_1 \to \until{\phi_2}{\phi_3}$;
    \item[par] $\phi_1 \to \phi_2 \vdash (\existss{x}{\phi_1}) \to \phi_2$ with $x$ not free in $\phi_2$.
  \end{enumerate}
\end{definition}

The axioms can be approximately grouped as such: the axioms 1-3 are one of the many combination of axioms that express
classical propositional logic; axioms 4-6 describe the unary temporal operator next and the interaction with
propositional operators; axioms 7-8 describe the binary temporal operator until and its expansion rule; axiom 9
introduces first-order quantification; axioms 10-11 describe equality. Inference rules are also self-explanatory, with
(mp) and (par) derived from \ac{FOL}, \emph{nex} and \emph{ind} allows instead for the introduction of temporal operators.

With the following lemmas and theorems we will provide correctness results for this axiomatic system.

\begin{lemma}\label{lem:propaxioms}
  Let $\phi_1, \phi_2, \phi_3 \in \mathcal{F}_\Sigma$.
  \begin{enumerate}
    \item $\vDash \phi_1 \to (\phi_2 \to \phi_1)$;
    \item $\vDash (\phi_1 \to (\phi_2 \to \phi_3)) \to ((\phi_1 \to \phi_2) \to (\phi_1 \to \phi_3))$;
    \item $\vDash (\neg\phi_1 \to \neg\phi_2) \to (\phi_2 \to \phi_1)$.
  \end{enumerate}
\end{lemma}
\begin{proof}
  Let $M$ be a counterpart model and a context $\Gamma$, a sequence $\pi = w_0 \overset{cr_0}{\rightsquigarrow} w_1 \overset{cr_1}{\rightsquigarrow} w_2 \overset{cr_2}{\rightsquigarrow} \ldots$ and an assignment $\sigma$ over the world $w_0$:
  \begin{enumerate}
    \item From trivial applications of the semantic rules it follows that $\sem{\pi}{\sigma}{\Gamma}{\phi_1 \to (\phi_2
      \to \phi_1)} \Leftrightarrow \semn{\pi}{\sigma}{\Gamma}{\phi_1} \text{ or } \semn{\pi}{\sigma}{\Gamma}{\phi_2}
      \text{ or } \sem{\pi}{\sigma}{\Gamma}{\phi_1}$ which is a tautology by the law of excluded middle.

    \item By~\Cref{eq:iffsem} we show that the stronger formula $\vDash (\phi_1 \to (\phi_2 \to \phi_3)) \leftrightarrow ((\phi_1 \to \phi_2)
      \leftrightarrow (\phi_1 \to \phi_3))$. From the semantic rules:
      \[
        \begin{split}
          &\sem{\pi}{\sigma}{\Gamma}{\phi_1 \to (\phi_2 \to \phi_3)} \\
            &\quad\Leftrightarrow \semn{\pi}{\sigma}{\Gamma}{\phi_1} \text{ or } \semn{\pi}{\sigma}{\Gamma}{\phi_2} \text{ or }
              \sem{\pi}{\sigma}{\Gamma}{\phi_3} \\
            &\quad\Leftrightarrow (\sem{\pi}{\sigma}{\Gamma}{\phi_1} \text{ and } \semn{\pi}{\sigma}{\Gamma}{\phi_2}) \text{ or }
              \semn{\pi}{\sigma}{\Gamma}{\phi_1} \text{ or } \sem{\pi}{\sigma}{\Gamma}{\phi_3} \\
            &\quad\Leftrightarrow \sem{\pi}{\sigma}{\Gamma}{(\phi_1 \to \phi_2) \to (\phi_1 \to \phi_3)}
        \end{split}
      \]

    \item By~\Cref{eq:iffsem} we show the stronger formula $\vDash (\neg\phi_1 \to \neg\phi_2) \leftrightarrow (\phi_2 \to \phi_1)$.
      From the semantic rules:
      \[
        \begin{split}
          \sem{\pi}{\sigma}{\Gamma}{\neg\phi_1 \to \neg\phi_2}
            &\Leftrightarrow \sem{\pi}{\sigma}{\Gamma}{\phi_1} \text{ or } \semn{\pi}{\sigma}{\Gamma}{\phi_2} \\
            &\Leftrightarrow \sem{\pi}{\sigma}{\Gamma}{\phi_2 \to \phi_1}
        \end{split}
      \]
  \end{enumerate}
\end{proof}

\begin{lemma}\label{lem:negnextdist}
  Let $\phi \in \mathcal{F}_\Sigma$, $\vDash \neg\nextop{\phi} \leftrightarrow \nextop{\neg\phi}$.
\end{lemma}
\begin{proof}
  Let $M$ be a counterpart model and a context $\Gamma$, a sequence $\pi = w_0 \overset{cr_0}{\rightsquigarrow} w_1
  \overset{cr_1}{\rightsquigarrow} w_2 \overset{cr_2}{\rightsquigarrow} \ldots$ and an assignment $\sigma$ over the world $w_0$:
  \[
    \begin{split}
      \sem{\pi}{\sigma}{\Gamma}{\neg\nextop{\phi}}
        &\Leftrightarrow \semn{\suffix(\pi,1)}{cr_0 \circ \sigma}{\Gamma}{\phi} \\
        &\Leftrightarrow \sem{\suffix(\pi,1)}{cr_0 \circ \sigma}{\Gamma}{\neg\phi} \\
        &\Leftrightarrow \sem{\pi}{\sigma}{\Gamma}{\nextop{\neg\phi}}
    \end{split}
  \]
\end{proof}

\begin{lemma}\label{lem:impnextdist}
  Let $\phi_1, \phi_2 \in \mathcal{F}_\Sigma$, $\vDash \nextop{(\phi_1 \to \phi_2)} \to (\nextop{\phi_1} \to
  \nextop{\phi_2})$.
\end{lemma}
\begin{proof}
  Let $M$ be a counterpart model and a context $\Gamma$, a sequence $\pi = w_0 \overset{cr_0}{\rightsquigarrow} w_1
  \overset{cr_1}{\rightsquigarrow} w_2 \overset{cr_2}{\rightsquigarrow} \ldots$ and an assignment $\sigma$ over the world $w_0$:
  \[
    \begin{split}
      &\sem{\pi}{\sigma}{\Gamma}{\nextop{(\phi_1 \to \phi_2)}} \\
        &\quad\Leftrightarrow \sem{\suffix(\pi,1)}{cr_0 \circ \sigma}{\Gamma}{\phi_1 \to \phi_2} \\
        &\quad\Leftrightarrow \sem{\suffix(\pi,1)}{cr_0 \circ \sigma}{\Gamma}{\phi_1} \text{ implies }
            \sem{\suffix(\pi,1)}{cr_0 \circ \sigma}{\Gamma}{\phi_2} \\
        &\quad\Leftrightarrow \sem{\pi}{\sigma}{\Gamma}{\nextop{\phi_1} \to \nextop{\phi_2}}
    \end{split}
  \]
\end{proof}

\begin{lemma}\label{lem:nextintro}
  Let $\phi \in \mathcal{F}_\Sigma$ be a rigid formula, $\vDash \phi \to \nextop{\phi}$.
\end{lemma}
\begin{proof}
  Let $M$ be a counterpart model and a context $\Gamma$, a sequence $\pi = w_0 \overset{cr_0}{\rightsquigarrow} w_1
  \overset{cr_1}{\rightsquigarrow} w_2 \overset{cr_2}{\rightsquigarrow} \ldots$ and an assignment $\sigma$ over the
  world $w_0$.
  Let $\sem{\pi}{\sigma}{\Gamma}{\phi}$. By definition, $\sem{\pi}{\sigma}{\Gamma}{\nextop{\phi}}$ iff
  $\sem{\suffix(\pi,1)}{cr_0 \circ \sigma}{\Gamma}{\phi}$.
  Since $\phi$ is a rigid formula, it must hold irregardless of the particular counterpart relation that is applied to
  it, thus $\sem{\suffix(\pi,1)}{cr_0 \circ \sigma}{\Gamma}{\phi}$ must also hold.
\end{proof}

\begin{lemma}\label{lem:untilexp}
  Let $\phi_1, \phi_2 \in \mathcal{F}_\Sigma$, $\vDash \until{\phi_1}{\phi_2} \leftrightarrow \phi_2 \lor (\phi_1 \land
  \nextop{(\until{\phi_1}{\phi_2})})$.
\end{lemma}
\begin{proof}
  Follows trivially by the definition of the operators until and next.
\end{proof}

\begin{lemma}\label{lem:untileventually}
  Let $\phi_1, \phi_2 \in \mathcal{F}_\Sigma$, $\vDash \until{\phi_1}{\phi_2} \to \eventually{\phi_2}$.
\end{lemma}
\begin{proof}
  Let $M$ be a counterpart model and a context $\Gamma$, a sequence $\pi = w_0 \overset{cr_0}{\rightsquigarrow} w_1
  \overset{cr_1}{\rightsquigarrow} w_2 \overset{cr_2}{\rightsquigarrow} \ldots$ and an assignment $\sigma$ over the
  world $w_0$.
  By the definitions of the operators:
  \begin{align*}
    &\sem{\pi}{\sigma}{\Gamma}{\until{\phi_1}{\phi_2}} \\
    &\qquad\Leftrightarrow \existss{j \geq 0}{\sem{\suffix(\pi,j)}{cr_{j-1} \circ \cdots \circ cr_0 \circ \sigma}{\Gamma}{\phi_2}} \\
    &\qquad\qquad\text{ and } \foralls{0 \leq i < j}{\sem{\suffix(\pi,i)}{cr_{i-1} \circ \cdots \circ cr_0 \circ
    \sigma}{\Gamma}{\phi_1}} \\
    &\qquad\Rightarrow \existss{j \geq 0}{\sem{\suffix(\pi,j)}{cr_{j-1} \circ \cdots \circ cr_0 \circ \sigma}{\Gamma}{\phi_2}} \\
    &\qquad\Leftrightarrow \sem{\pi}{\sigma}{\Gamma}{\eventually{\phi_2}}
  \end{align*}
\end{proof}

\begin{lemma}\label{lem:exintro}
  Let $\phi \in \mathcal{F}_\Sigma$ with $x \in X$ free in $\phi$, $\vDash \phi[v/x] \to \existss{x}{\phi}$, if $v$ is a
  value in the domain.
\end{lemma}
\begin{proof}
  Let $M$ be a counterpart model and a context $\Gamma$, a sequence $\pi = w_0 \overset{cr_0}{\rightsquigarrow} w_1
  \overset{cr_1}{\rightsquigarrow} w_2 \overset{cr_2}{\rightsquigarrow} \ldots$ and an assignment $\sigma$ over the
  world $w_0$.
  By hypothesis $\sem{\pi}{\sigma}{\Gamma}{\phi[v/x]}$, with $v \in d(w_0)$, then it holds that
  $\sem{\pi}{\ext{x}{\sigma}{v}}{\Gamma,x}{\phi}$, therefore $\sem{\pi}{\sigma}{\Gamma}{\existss{x}{\phi}}$ holds.
\end{proof}

\begin{lemma}\label{lem:equality}
  Let $\phi \in \mathcal{F}_\Sigma$ and $x, y \in X$,
  \begin{enumerate}
    \item $\vDash x = x$;
    \item $\vDash x = y \to (\phi \to \phi[y/x])$.
  \end{enumerate}
\end{lemma}
\begin{proof}
  Let $M$ be a counterpart model and a context $\Gamma$, a sequence $\pi = w_0 \overset{cr_0}{\rightsquigarrow} w_1
  \overset{cr_1}{\rightsquigarrow} w_2 \overset{cr_2}{\rightsquigarrow} \ldots$ and an assignment $\sigma$ over the
  world $w_0$.
  \begin{enumerate}
    \item By definition, $\sem{\pi}{\sigma}{\Gamma}{x = x}$ holds if $\sigma(x) = \sigma(x)$, which is trivially true.
    \item By assumption it holds that $\sem{\pi}{\sigma}{\Gamma}{x = y}$, thus $\sigma(x) = \sigma(y)$. Assume
      $\sem{\pi}{\sigma}{\Gamma}{\phi}$ holds, then by the congruence property of equality it must also hold
      $\sem{\pi}{\sigma}{\Gamma}{\phi[y/x]}$.
  \end{enumerate}
\end{proof}

\begin{lemma}\label{lem:mp}
  If $F \vDash \phi_1$ and $F \vDash \phi_1 \to \phi_2$, then $F \vDash \phi_2$.
\end{lemma}
\begin{proof}
  Let $M$ be a counterpart model that satisfies $\vDash_M \psi$ for every $\psi \in F$.
  By definition $\sem{\pi}{\sigma}{\Gamma}{\phi_1 \to \phi_2} \Leftrightarrow \sem{\pi}{\sigma}{\Gamma}{\phi_1}
  \text{ implies } \sem{\pi}{\sigma}{\Gamma}{\phi_2}$. By assumption it holds $\sem{\pi}{\sigma}{\Gamma}{\phi_1}$, thus it must
  hold that $\sem{\pi}{\sigma}{\Gamma}{\phi_2}$.
\end{proof}
\begin{lemma}\label{lem:nex}
  If $F \vDash \phi$, then $F \vDash \nextop{\phi}$ and $F \vDash \forever{\phi}$.
\end{lemma}
\begin{proof}
  Let $M$ be a counterpart model that satisfies $\vDash_M \psi$ for every $\psi \in F$ and $\Gamma$.
  By assumption $\sem{\pi}{\sigma}{\Gamma}{\phi}$ holds for every pair sequence $\pi$ and assignment $\sigma$.
  In particular given a sequence $\pi = w_0 \overset{cr_0}{\rightsquigarrow} w_1 \overset{cr_1}{\rightsquigarrow}
  \ldots$ and an assignment $\sigma$, it must also hold $\sem{\suffix(\pi,i)}{cr_{i-1} \circ \cdots \circ cr_0 \circ
  \sigma}{\Gamma}{\phi}$ for every $i \geq 0$ thus $F \vDash \forever{\phi}$ and in particular with $i = 1$ thus $F
  \vDash \nextop{\phi}$.
\end{proof}
\begin{lemma}\label{lem:ind}
  If $F \vDash \phi_1 \to \phi_2$ and $F \vDash \phi_1 \to \nextop{\phi_1}$, then $F \vDash \phi_1 \to \forever{\phi_2}$.
\end{lemma}
\begin{proof}
  Let $M$ be a counterpart model that satisfies $\vDash_M \psi$ for every $\psi \in F$ and $\Gamma$.  If
  $\semn{\pi}{\sigma}{\Gamma}{\phi_1}$ than the conclusion trivially holds. Assume, instead,
  $\sem{\pi}{\sigma}{\Gamma}{\phi_1}$, with $\pi = w_0 \overset{cr_0}{\rightsquigarrow} w_1
  \overset{cr_1}{\rightsquigarrow} \ldots$.
  By expanding the formula we need to show that $\foralls{j \geq 0}{\sem{\suffix(\pi, j)}{cr_{i - 1} \circ \cdots cr_0 \circ
  \sigma}{\Gamma}{\phi_2}}$ holds.
  However, by hypothesis $\sem{\sigma}{w}{\Gamma}{\phi_2}$ and
  $\sem{\sigma}{w}{\Gamma}{\nextop{\phi_1}}$ are valid in every $(\sigma, w)$ such that
  $\sem{\sigma}{w}{\Gamma}{\phi_1}$ is valid, therefore $\phi_1$ is valid in every sequence described before and thus
  $\phi_2$ is valid in every sequence.
\end{proof}
\begin{lemma}\label{lem:par}
  If $F \vDash \phi_1 \to \phi_2$ and $x$ not free in $\phi_2$, then $F \vDash (\existss{x}{\phi_1}) \to \phi_2$.
\end{lemma}
\begin{proof}
  Let $M$ be a counterpart model that satisfies $\vDash_M \psi$ for every $\psi \in F$ and $\Gamma$.
  Assume $\semn{\pi}{\sigma}{\Gamma}{(\existss{x}{\phi_1}) \to \phi_2}$ for some sequence $\pi = w_0 \overset{cr_0}{\rightsquigarrow} w_1 \overset{cr_1}{\rightsquigarrow} \ldots$
  and assignment $\sigma$, it follows that $\sem{\pi}{\sigma}{\Gamma}{\existss{x}{\phi_1}}$ and $\semn{\pi}{\sigma}{\Gamma}{\phi_2}$.
  Then there is a $v \in d(w_0)$ such that $\sem{\pi}{\ext{x}{\sigma}{v}}{\Gamma, x}{\phi_1}$, by definition of existential
  quantifier. Since $\phi_2$ does not contain $x$ as a free variable, adding an evaluation for $x$ does not change the
  evaluation of $\phi_2$, i.e. $\semn{\pi}{\ext{x}{\sigma}{v}}{\Gamma, x}{\phi_2}$, therefore
  $\semn{\pi}{\ext{x}{\sigma}{v}}{\Gamma, x}{\phi_1 \to \phi_2}$, which is a contradiction to the hypothesis $\vDash_M
  \phi_1 \to \phi_2$.
\end{proof}

\begin{theorem}[Soundness]
  Let $\phi \in \mathcal{F}_\Sigma$ and $F \subseteq \mathcal{F}_\Sigma$, if $F \vdash \phi$ then $F \vDash \phi$.
\end{theorem}
\begin{proof}
  By induction on the derivation of $\phi$ from $F$:
  \begin{enumerate}
    \item if $\phi$ is an axiom: $F \vDash \phi$ is proven by~\Cref{lem:propaxioms} for axioms 1,2,3;
      by~\Cref{lem:negnextdist} for axiom 4; by~\Cref{lem:impnextdist} for axiom 5; by~\Cref{lem:nextintro} for axiom 6;
      by~\Cref{lem:untilexp} for axiom 7; by~\Cref{lem:untileventually} for axiom 8; by~\Cref{lem:exintro} for axiom 9;
      by~\Cref{lem:equality} for axioms 10, 11;
    \item if $\phi \in F$ then $F \vDash \phi$ holds trivially;
    \item if $\phi$ is the conclusion of a (mp) rule with premises $F \vdash \psi$ and $F \vdash \psi \to \phi$: by
      induction hypothesis we have $F \vDash \psi$ and $F \vDash \psi \to \phi$, hence $F \vDash \phi$ follows by
      \Cref{lem:mp};
    \item if $\phi$ is the conclusion of a (nex) rule with premises $F \vdash \psi$, thus $\phi \equiv \nextop{\psi}$: by
      induction hypothesis we have $F \vDash \psi$, hence $F \vDash \nextop{\psi}$ follows by \Cref{lem:nex};
    \item if $\phi$ is the conclusion of a (ind) rule with premises $F \vdash \psi_1 \to \psi_2$ and $F \vdash \psi_1
      \to \nextop{\psi_1}$, thus $\phi \equiv \psi_1 \to \forever{\psi_2}$: by induction hypothesis we have $F \vDash
      \psi_1 \to \psi_2$ and $F \vDash \psi_1 \to \nextop{\psi_2}$, hence $F \vDash \psi_1 \to \forever{\psi_2}$ follows
      by \Cref{lem:ind};
    \item if $\phi$ is the conclusion of a (par) rule with premises $F \vdash \psi_1 \to \psi_2$, thus $\phi \equiv
      (\existss{x}{\psi_1}) \to \psi_2$ with $x$ not free in $\psi_2$: by induction hypothesis we have $F \vDash \psi_1
      \to \psi_2$, hence $F \vDash (\existss{x}{\psi_1}) \to \psi_2$ follows by \Cref{lem:par}.
  \end{enumerate}
\end{proof}

\begin{example}\label{ex:alw}
  We show the derivation of some rules to show the capabilities of the deduction system.
  In the following derivation we will denote as (prop) all the rules that are tautology in propositional
  logic, since it is completely described by the first three axioms of the system.

  We will start with a variant of the (ind) rule, $\phi \to \nextop{\phi} \vdash \phi \to \forever{\phi}$, which we will call (ind1):

  \medskip
  \begin{tabularx}{300pt}{cXl}
    (1) & $\phi \to \nextop{\phi}$ & assumption \\
    (2) & $\phi \to \phi$ & (prop) \\
    (3) & $\phi \to \forever{\phi}$ & (ind), (1), (2)
  \end{tabularx}

  \medskip
  Next we show the (alw) rule, $\phi \vdash \forever{\phi}$, which will be useful later for the deduction theorems.

  \medskip
  \begin{tabularx}{300pt}{cXl}
    (1) & $\phi$ & assumption \\
    (2) & $\nextop{\phi}$ & (nex), (1) \\
    (3) & $\phi \to \nextop{\phi}$ & (axiom 1), (2) \\
    (4) & $\phi \to \forever{\phi}$ & (ind1), (3) \\
    (5) & $\forever{\phi}$ & (mp), (1), (4) \\
  \end{tabularx}

  \medskip
  Here we show the (for) rule, i.e. the expansion rule for the forever operator.

  \medskip
  \begin{tabularx}{300pt}{cXl}
    (1) & $(\neg\phi \lor \nextop{(\until{tt}{\neg\phi})}) \to \until{tt}{\neg\phi}$ & axiom 7 \\
    (2) & $\neg\until{tt}{\neg\phi} \to (\neg(\neg\phi \lor \nextop{(\until{tt}{\neg\phi})}))$ & (1), axiom 3 \\
    (3) & $\neg\until{tt}{\neg\phi} \to (\phi \land \neg\nextop{(\until{tt}{\neg\phi})})$ & (2), (prop) \\
    (4) & $\forever{\phi} \to (\phi \land \forever{\phi})$ & (3), definition \\
  \end{tabularx}
\end{example}

As in the propositional calculus and \ac{FOL}, in the next pair of theorems we will investigate the connection between
implication and the derivability of a formula.

\begin{theorem}[Deduction Theorem]
  Let $\phi_1, \phi_2 \in \mathcal{F}_\Sigma$ and $F \subseteq \mathcal{F}_\Sigma$. If $F, \phi_1 \vdash
  \phi_2$ and this derivation of $\phi_2$ does not contains application of the rule (par) for a variable occurring free
  in $\phi_1$, then $F \vdash \forever{\phi_1} \to \phi_2$.
\end{theorem}
\begin{proof}
  We will use induction on the derivation of $\phi_2$ from $F, \phi_1$:
  \begin{enumerate}
    \item if $\phi_2$ is an axiom or $\phi_2 \in F$, then $F \vdash \phi_2$ and $F \vdash \forever{\phi_1} \to \phi_2$ holds as
      $\phi_2$ always holds;
    \item if $\phi_2 \equiv \phi_1$, then $F \vdash \forever{\phi_1} \to \phi_1 \land \nextop{\forever{\phi_1}}$ from
      (for) and by tautology $F \vdash \forever{\phi_1} \to \phi_1$;
    \item if $\phi_2$ is a conclusion of (mp) rule with premises $\psi$ and $\psi \to \phi_2$, then we have $F, \phi_1 \vdash
      \psi$ and $F, \phi_1 \vdash \psi \to \phi_2$ that by induction hypothesis imply $F \vdash \forever{\phi_1} \to \psi$
      and $F \vdash \forever{\phi_1} \to (\psi \to \phi_2)$:

      \begin{tabularx}{300pt}{cXl}
        (1) & $\forever{\phi_1} \to \psi$ & assumption \\
        (2) & $\forever{\phi_1} \to (\psi \to \phi_2)$ & assumption \\
        (3) & $(\forever{\phi_1} \to \psi) \to (\forever{\phi_1} \to \phi_2)$ & (2), axiom 2 \\
        (4) & $\forever{\phi_1} \to \phi_2$ & (1), (3), (mp) \\
      \end{tabularx}
    \item if $\phi_2 \equiv \nextop{\psi}$ is a conclusion of (nex) rule with premise $\psi$, then $F, \phi_1 \vdash \psi$
      and by induction hypothesis $F \vdash \forever{\phi_1} \to \psi$:

      \begin{tabularx}{300pt}{cXl}
        (1) & $\forever{\phi_1} \to \psi$ & assumption \\
        (2) & $\nextop{(\forever{\phi_1} \to \psi)}$ & (1), (nex) \\
        (3) & $\nextop{(\forever{\phi_1} \to \psi)} \to (\nextop{\forever{\phi_1}} \to \nextop{\psi})$ & (2), axiom 5 \\
        (4) & $\nextop{\forever{\phi_1}} \to \nextop{\psi}$ & (2), (3), (mp) \\
        (5) & $\forever{\phi_1} \to \phi_1 \land \nextop{\forever{\phi_1}}$ & (for) \\
        (6) & $\forever{\phi_1} \to \nextop{\forever{\phi_1}}$ & (prop) \\
        (7) & $\forever{\phi_1} \to \nextop{\psi}$ & (4), (6), (prop) \\
      \end{tabularx}

    \item if $\phi_2 \equiv \psi_1 \to \until{\psi_2}{\psi_3}$ is a conclusion of a (ind) rule with premise $\psi_1 \to
      \psi_3 \lor (\psi_2 \land \nextop{\psi_1})$, then $F, \phi_1 \vdash \psi_1 \to \psi_3 \lor (\psi_2 \land
      \nextop{\psi_1})$ and by induction hypothesis $F \vdash \forever{\phi_1} \to (\psi_1 \to \psi_3 \lor (\psi_2 \land
      \nextop{\psi_1}))$:

      \begin{tabularx}{300pt}{cXl}
        (1) & $\forever{\phi_1} \to (\psi_1 \to \psi_3 \lor (\psi_2 \land \nextop{\psi_1}))$ & assumption \\
        (2) & $(\psi_1 \to \psi_3 \lor (\psi_2 \land \nextop{\psi_1})) \to (\psi_1 \to \until{\psi_2}{\psi_3})$ & axiom 7\\
        (3) & $(\forever{\phi_1} \land \psi_1 \to \psi_3 \lor (\psi_2 \land \nextop{\psi_1})) \to (\psi_1 \to
        \until{\psi_2}{\psi_3})$ & (2), (prop) \\
        (4) & $\forever{\phi_1} \land \psi_1 \to \psi_3 \lor (\psi_2 \land \nextop{\psi_1})$ & (1), (prop) \\
        (5) & $\forever{\phi_1} \to (\psi_1 \to \until{\psi_2}{\psi_3})$ & (3),(4),(mp) \\
      \end{tabularx}

    \item if $\phi_2 \equiv (\existss{x}{\psi_1}) \to \psi_2$ is a conclusion of a (par) rule with premise $\psi_1 \to
      \psi_2$ and $x$ not free in $\psi_2$ and $\phi_1$, then $F, \phi_1 \vdash \psi_1 \to \psi_2$ and by induction
      hypothesis $F \vdash \forever{\phi_1} \to (\psi_1 \to \psi_2)$:

      \begin{tabularx}{300pt}{cXl}
        (1) & $\forever{\phi_1} \to (\psi_1 \to \psi_2)$ & assumption \\
        (2) & $\psi_1 \to (\forever{\phi_1} \to \psi_2)$ & (1), prop \\
        (3) & $(\existss{x}{\psi_1}) \to (\forever{\phi_1} \to \psi_2)$ & (2), (par) \\
        (4) & $\forever{\phi_1} \to ((\existss{x}{\psi_1}) \to \psi_2)$ & (3), (prop) \\
      \end{tabularx}
  \end{enumerate}
\end{proof}

The Deduction Theorem has two peculiarities that makes it different from the equivalent deduction theorem in \ac{FOL},
i.e. $F, \phi_1 \vdash \phi_2 \Rightarrow F \vdash \phi_1 \to \phi_2$, both are attested since the first axiomatizations
of quantified temporal logics~\cite{kozen_verification_1982}.

First, the introduction of the forever operator in the premise of the implication. It is trivial to show that $F, \phi_1
\vdash \phi_2 \Rightarrow F \vdash \phi_1 \to \phi_2$ does not hold. For example, $\phi_1 \vdash \forever{\phi_1}$
trivially holds by the (alw) induction rule, but $\vdash \phi_1 \to \forever{\phi_1}$ is trivially not a theorem. If the
additional constraint of not containing the application of the (nex) rule is applied then we can recover the more common
form of deduction theorem, although in a restricted variant.

Second, the derivation must not contain an application of (par) for a variable occurring free in the premise.  This
restriction is required for the rule (par) in the derivation (3) of the last point in the proof, as shown by
the~\Cref{lem:par}.

\begin{theorem}
  Let $\phi_1, \phi_2 \in \mathcal{F}_\Sigma$ and $F \subseteq \mathcal{F}_\Sigma$. If $F \vdash \forever{\phi_1} \to
  \phi_2$, then $F, \phi_1 \vdash \phi_2$.
\end{theorem}
\begin{proof}
  Assume $F \vdash \forever{\phi_1} \to \phi_2$, then $F,\phi_1 \vdash \forever{\phi_1} \to \phi_2$ also
  holds. By the (alw) rule~(\Cref{ex:alw}) and the trivial derivation $F, \phi_1 \vdash \phi_1$. Therefore
  $F, \phi_1 \vdash \forever{\phi_1}$, then by (mp) rule it follows $F, \phi_1 \vdash \phi_2$.
\end{proof}

\section{The missing axioms}
Readers accustomed to Kripke-based semantics for quantified temporal logic, will know that axiomatic systems contains a
formula called Barcan formula~\cite{barcan_functional_1946}, $\nextop{\existss{x}{\phi}} \to \existss{x}{\nextop{\phi}}$
or one of its many equivalent variations, which is missing in our system.  Let $\pi = w_0
\overset{cr_0}{\rightsquigarrow} w_1 \overset{cr_1}{\rightsquigarrow} \ldots$ be a generic sequence of accessible worlds
and $\sigma$ an assignment for the world $w_0$.
\begin{align*}
  \sem{\pi}{\sigma}{\Gamma}{\nextop{\existss{x_\tau}{\phi}}}
    &\Leftrightarrow \existss{v \in d(w_1)}{\sem{\suffix(\pi, 1)}{\ext{x}{cr_0 \circ \sigma}{v}}{\Gamma,x}{\phi}} \\
  \sem{\pi}{\sigma}{\Gamma}{\existss{x_\tau}{\nextop{\phi}}}
    &\Leftrightarrow \existss{v \in d(w_0)}{\sem{\suffix(\pi, 1)}{cr_0 \circ \ext{x}{\sigma}{v}}{\Gamma,x}{\phi}}
\end{align*}
The existential operators on the two sides quantify on different domains. 
Assume both domains for the worlds $w_0$ and $w_1$ are non-empty and have the same algebra. We will see if the Barcan
formula holds for the formula $\phi \equiv x_\tau = c$ where $c$ is a constant of type $\tau$ in the algebra.  Assume
that the counterpart relation $cr_0$ is the empty relation. The formula $\nextop{\existss{x_\tau}{x = c}}$ is trivially
satisfied by extending the empty substitution, since the formula is closed, with the value $c$ for the variable $x$. The
other side is problematic, because the substitution can still be extended with the value $c$ for the variable $x$,
however the composition with the counterpart relation reduces the association to be undefined, which does not satisfy
the equality. Systems that include some form of the Barcan formula as an axiom, require domains to be constant while
traversing accessible worlds, or that domains do not grow or shrink, allowing at least one side of the implication.
Beside philosophical questions~\cite{williamson_modal_2015} regarding the value of such axiom, we will not extend our
model to include such axiom, since one of the main drive of this alternative interpretation of temporal formulae is the
necessity of modeling systems with creation and deallocation of resources.

\section{Completeness and Decidability}
The model presented in this work is not exempt from all the considerations on completeness available in literature for
quantified temporal logic~\cite{merz_decidability_1992}. Since the signature and the axioms for the theory of natural
numbers can be encoded in our system, it follows that our logic is
incomplete~\cite{barwise_incompleteness_1977,tarski_undecidable_1953}.
\begin{example}
  The theory of natural numbers can be encoded with the following signature $\Sigma = (S_\Sigma, F_\Sigma)$ where
  \[
    \begin{split}
      S_\Sigma &= \set{\tau_\mathbb{N}} \\
      F_\Sigma &= \set{0 : \tau_\mathbb{N}, s : \tau_\mathbb{N} \to \tau_\mathbb{N}, + : \tau_\mathbb{N} \times
      \tau_\mathbb{N} \to \tau_\mathbb{N}, \times : \tau_\mathbb{N} \times \tau_\mathbb{N} \to \tau_\mathbb{N}}
    \end{split}
  \]
  and by models where the following additional axioms are true:
  \begin{itemize}
    \item $s(x) \neq 0$
    \item $s(x) = s(y) \to x = y$
    \item $s(x) \neq 0 \to \existss{y_{\tau_\mathbb{N}}}{x = s(y)}$
    \item $x + 0 = x$
    \item $x + s(y) = s(x + y)$
    \item $x \times 0 = 0$
    \item $x \times s(y) = (x \times y) + x$
  \end{itemize}
  and the following axiom for the scheme of induction, where $\phi$ is a formula with $x$ free:
  \[
    \phi[0/x] \land (\foralls{x_{\tau_\mathbb{N}}}{\phi \to \phi[s(x)/x]}) \to \foralls{x_{\tau_\mathbb{N}}}{\phi}
  \]
\end{example}

More articulated is, instead, the case for decidability. Previous results show that \ac{FOLTL} is not only undecidable
in general, but in a multi-sorted context is not even semi-decidable~\cite{merz_decidability_1992}. However, it was
also showed that there are reasonable fragments that are, indeed, decidable but with serious
restrictions~\cite{hodkinson_decidable_2000,hodkinson_decidable_2002,goos_monodic_2001,peyras_decidable_2020,peyras_bounded_2019}.
The main issue is the interaction between temporal operators and quantifiers, in particular the until operator. One
approach is to limit the number of variables quantified within the scope of temporal operators, in particular
restricting temporal formulae to be monadic, thanks to decidability results on monadic second-order
logic~\cite{rozenberg_expression_1997}. Another approach is to limit the nestedness of operators, for example we must
avoid nested forever operators, or the usage of temporal operators inside the scope of a universal quantifier.
Nonetheless, there are still issues with classical operators, and we must restrict also the usage of those to an
arbitrary decidable fragment of first-order classical logic. Lastly, rather than focus of finding fragments that
guarantee decidability for every model, we can limit the class of models to those there are decidable, essentially
finitary ones, therefore not only the set of worlds is finite, but also the algebras must be finite. Both kinds of
restriction are strict enough to obtain a decidable fragment or decidable models of the second-order
$\mu$-calculus~\cite{hutchison_counterpart_2010}, which subsumes temporal logic.

A future objective is to investigate further this matter of decidability, as one of the objective of this model is to be
useful, and even simpler, for model checking purposes.


\chapter{Future work}
\section{Second-order quantification}

Following the corpus of results on counterpart relations as models for monodic second-order modal logics, we can
straightforwardly define a syntax for Monodic \ac{SOLTL}, with membership as the sole operator beside the temporal and
propositional ones.

\begin{definition}[Second-Order Linear Temporal Logic]
Let $\Sigma$ be a (multi-sorted) signature and $X$ a denumerable set of first-order variables typed over $S_\Sigma$, and
$\mathcal{X}$ a denumerable set of second-order variables typed over $S_\Sigma$. The set $\mathcal{F}_\Sigma$ of
formulae for Monodic \ac{SOLTL} is the set generated by the following grammar:
\[
  \phi \Coloneqq tt \;|\; \epsilon_\tau \in \chi_\tau
                    \;|\; \neg\phi
                    \;|\; \phi \lor \phi
                    \;|\; \existss{x_\tau}{\phi}
                    \;|\; \existss{\chi_\tau}{\phi}
                    \;|\; \nextop{\phi}
                    \;|\; \until{\phi}{\phi}
\]
where $\epsilon \in \terms{\Sigma}{X}$, $\exists x_\tau$ ranges over first-order variables of sort $\tau \in S_\Sigma$,
$\exists \chi_\tau$ ranges over second-order variables of sort $\tau \in S_\Sigma$,
$O$ is a unary operator which states that $\phi$ must hold at the next step and $U$ is a binary operator which states
that the first formula must hold until the second formula holds at some point in the current or next steps.
\end{definition}

Other operators are derived the same as for \ac{FOLTL}, with the addition of equality, which in
this case becomes itself a derived operator, $\epsilon_\tau = \eta_\tau \equiv \foralls{\chi_\tau}(\epsilon_\tau \in
\chi_\tau \leftrightarrow \eta_\tau \in \chi_\tau)$.

To define a semantic with counterpart relations for the Monodic \ac{SOLTL} we need to extend some concept to the
second-order. First we need a new context, a second-order context, which is a subset of second-order variables, i.e.
$\Delta \subseteq \mathcal{X}$. Second, given a counterpart model $M = (W, \rightsquigarrow, d)$ and a world $w \in W$,
we require a second-order assignment $\xi$, i.e $\xi$ is a partial function such that $\xi : \mathcal{X} \rightharpoonup
2^{d(w)}$, which intuitively assigns a set of values in the domain $d(w)$ to each second-order variable.

\begin{definition}[Semantic of \ac{SOLTL}]
Let $M$ be a counterpart model, $\Gamma,\Delta$ respectively a first-order and second-order context, $\pi = w_0
\overset{cr_0}{\rightsquigarrow} w_1 \overset{cr_1}{\rightsquigarrow} \ldots$ a sequence of accessible arrows,
$\sigma$ and $\xi$ respectively a first-order assignment and a second-order assignment for the world $w_0$:
The validity of a formula-in-context $\phi$, denoted $\semtwo{\sigma}{\xi}{w}{\Gamma}{\Delta}{\phi}$, is defined
inductively as follows:
\begin{align*}
  \pi, \sigma, \xi &\vDash_\Gamma^\Delta tt \\
  \pi, \sigma, \xi &\vDash_\Gamma^\Delta \epsilon_\tau \in \chi_\tau
      &&\text{iff } \sigma(\epsilon) \text{ defined and } \sigma(\epsilon) \in \xi(\chi) \\
  \pi, \sigma, \xi &\vDash_\Gamma^\Delta \neg\phi
      &&\text{iff } \semntwo{\pi}{\sigma}{\xi}{\Gamma}{\Delta}{\phi} \\
  \pi, \sigma, \xi &\vDash_\Gamma^\Delta \phi_1 \lor \phi_2
      &&\text{iff } \semtwo{\pi}{\sigma}{\xi}{\Gamma}{\Delta}{\phi_1} \text{ or }
      \semtwo{\pi}{\sigma}{\xi}{\Gamma}{\Delta}{\phi_2} \\
  \pi, \sigma, \xi &\vDash_\Gamma^\Delta \existss{x_\tau}{\phi}
      &&\text{iff } \existss{v \in d(w_0)}{\semtwo{\pi}{\ext{x}{\sigma}{v}}{\xi}{\Gamma, x}{\Delta}{\phi}} \text{ with }
      x \not\in \Gamma \\
  \pi, \sigma, \xi &\vDash_\Gamma^\Delta \existss{\chi_\tau}{\phi}
      &&\text{iff } \existss{v \in 2^{d(w_0)}}{\semtwo{\pi}{\sigma}{\ext{\chi}{\xi}{v}}{\Gamma}{\Delta,\chi}{\phi}}
        \text{ with } \xi \not\in \Delta \\
  \pi, \sigma, \xi &\vDash_\Gamma^\Delta \nextop{\phi}
      &&\text{iff } \semtwo{\suffix(\pi,1)}{cr_0 \circ \sigma}{2^{cr_0} \circ \xi}{\Gamma}{\Delta}{\phi} \\
  \pi, \sigma, \xi &\vDash_\Gamma^\Delta \until{\phi_1}{\phi_2}
     &&\text{iff } \semtwo{\pi}{\sigma}{\xi}{\Gamma}{\Delta}{\phi_2} \text{ or } (\semtwo{\pi}{\sigma}{\xi}{\Gamma}{\Delta}{\phi_1} \\
         &&&\qquad\text { and } \semtwo{\suffix(\pi,1)}{cr_0 \circ \sigma}{2^{cr_0} \circ \xi}{\Gamma}{\Delta}{\until{\phi_1}{\phi_2}}) \\
\end{align*}
  where $2^{cr}$ is the lifting of the partial function $cr$ to the power-set of the image, and $\extend$ defined as
  in~\Cref{def:fosemantic}.
\end{definition}

\section{Branching time}
A further axis of extension is to expand the logic to a branching time logic, thus by introducing operators that
quantify over the set of possible traces instead of a single one. Here we introduce the syntax and semantic for a
quantified version of \ac{CTL}~\cite{hodkinson_decidable_2002}.

\begin{align*}
  \phi &\Coloneqq tt \;|\; \epsilon_\tau = \epsilon_\tau
                    \;|\; \neg\phi
                    \;|\; \phi \lor \phi
                    \;|\; \existss{x_\tau}{\phi}
                    \;|\; \existpath{\psi}
                    \;|\; \forallpath{\psi} \\
  \psi &\Coloneqq \nextop{\phi} \;|\; \until{\phi}{\phi}
\end{align*}

\begin{align*}
  \pi, \sigma &\vDash_\Gamma tt \\
  \pi, \sigma &\vDash_\Gamma \epsilon_\tau = \eta_\tau &&\text{iff } \sigma(\epsilon) = \sigma(\eta) \\
  \pi, \sigma &\vDash_\Gamma \neg\phi &&\text{iff } \semn{\pi}{\sigma}{\Gamma}{\phi} \\
  \pi, \sigma &\vDash_\Gamma \phi_1 \lor \phi_2
      &&\text{iff } \sem{\pi}{\sigma}{\Gamma}{\phi_1} \text{ or } \sem{\pi}{\sigma}{\Gamma}{\phi_2} \\
  \pi, \sigma &\vDash_\Gamma \existss{x_\tau}{\phi}
      &&\text{iff } \existss{v \in d(w_0)}{\sem{\pi}{\extend_x(\sigma, v)}{\Gamma, x}{\phi}}
        \text{ with } x \not\in \Gamma \\
  \pi, \sigma &\vDash_\Gamma \nextop{\phi}
      &&\text{iff } \sem{\suffix(\pi, 1)}{cr_0 \circ \sigma}{\Gamma}{\phi} \\
  \pi, \sigma &\vDash_\Gamma \until{\phi_1}{\phi_2}
      &&\text{iff } \sem{\pi}{\sigma}{\Gamma}{\phi_2} \\ &&&\quad\text{ or } (\sem{\pi}{\sigma}{\Gamma}{\phi_1}
        \text { and } \sem{\suffix(\pi, 1)}{cr_0 \circ \sigma}{\Gamma}{\until{\phi_1}{\phi_2}}) \\
  \pi, \sigma &\vDash_\Gamma \existpath{\phi}
      &&\text{iff } \existss{\pi' \in \paths(\pi_0)}{\sem{\pi'}{\sigma}{\Gamma}{\phi}} \\
  \pi, \sigma &\vDash_\Gamma \forallpath{\phi}
      &&\text{iff } \foralls{\pi' \in \paths(\pi_0)}{\sem{\pi'}{\sigma}{\Gamma}{\phi}} \\
\end{align*}
Where $\extend$ is defined as in~\Cref{def:fosemantic}, and paths is the set of traces starting from the argument world,
$\paths(w) = \set{\pi | \pi = w \overset{cr_0}{\rightsquigarrow} w_1 \overset{cr_1}{\rightsquigarrow} \ldots}$.

The translation from \ac{LTL} to \ac{CTL} is straightforawrd. The operators that operate on single world are
semantically identical to their \ac{LTL} variants. The path operators instead quantify over the set of paths that start
from the current world.

Logically the next step would be investigating an axiomatic system for \ac{CTL} or extending the semantic to \ac{CTL}* a
superset of both \ac{LTL} and \ac{CTL}. Another approach is to add a probability measure to the transitions in the
counterpart model, thus adding a probability operator, as in \ac{PCTL}~\cite{brazdil_satisfiability_2008}.


\end{document}
