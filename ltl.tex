\section{Syntax}

\begin{definition}[Linear Temporal Logic]
  Let $\text{PROP}$ be a set of atomic propositions. The set $\mathcal{F}$ of formulae for Linear Temporal Logic is the
  set generated by the following grammar:
  \[
    \phi \Coloneqq tt \;|\; p
                      \;|\; \neg\phi
                      \;|\; \phi \lor \phi
                      \;|\; \next{\phi}
                      \;|\; \until{\phi}{\phi}
  \]
  where $p \in \text{PROP}$, $X$ is a unary operator which states that $\phi$ must hold at the next step and $U$ is a
  binary operator which states that the first formula must hold until the second formula holds at some point in the
  current or next steps.
\end{definition}

Classical propositional logic connetives can be derived trivially:
\[
  \phi_1 \land \phi_2 \equiv \neg(\neg\phi_1 \lor \neg\phi_2) \quad
  \phi_1 \to \phi_2 \equiv \neg\phi_1 \lor \phi_2 \quad
  \phi_1 \leftrightarrow \phi_2 \equiv (\phi_1 \to \phi_2) \land (\phi_2 \to \phi_1)
\]

Other temporal operators used in temporal logic literature are derivable as such:
\[
  \eventually{\phi} \equiv \until{tt}{\phi} \quad
  \forever{\phi} \equiv \neg\eventually(\neg\phi) \quad
  \weak{\phi_1}{\phi_2} \equiv (\until{\phi_1}{\phi_2}) \lor \forever{\phi_1}
\]
Intuitively $\eventually{\phi}$ means that eventually at some next step in time the formula $\phi$ holds,
$\forever{\phi}$ means that the formula $\phi$ holds at each possible next step.

LTL formulae can encode safety properties, commonly with the form $G\neg\phi$, and liveness properties of systems,
commonly with the form $GF\phi$ or $G(\phi_1 \to F\phi_2)$. Let's assume that we have two concurrent processes and that
we can encode with two atomic propositions, $\text{crit}_1$ and $\text{crit}_2$, when the two processes can access a
critical section, then the property of mutual exclusivity can be model by the formula $G(\neg\text{crit}_1 \lor
\neg\text{crit}_2)$. If we also have atomic propositions for wait conditions, $\text{wait}_1$, then we can model
liveness conditions, for example $G(\text{wait}_1 \to F\text{crit}_1)$ which means that whenever a process reaches a
wait condition, eventually will enter the critical section; we can also model fairness conditions, for example
$GF\text{wait}_1 \to GF\text{crit}_1$ means that if the process infinitely often reaches a wait condition than
infinitely often will reach the critical section.

\section{Semantic}

The de-facto standard semantics for Linear Temporal Logic and Temporal Logics in general are Kripke-style semantics, as
for modal logics. In modal logic, the so-called Kripke-frames are pairs of worlds and an accessibility relation between
worlds. For temporal logic, each world is a point in time and the accessibility relation models the flow of time, thus a
temporal model is defined as:

\begin{definition}[Temporal model]
  A \emph{temporal model} $M$ is a triple $(T, \prec, L)$ where $T$ is a set of time-points, $\prec$ is an accessibility
  relations over $T$ and $L : T \to \mathcal{P}(\text{PROP})$ is a labelling function that for each point in time
  returns the subset of all atomic propositions which are valid at that point.
\end{definition}

Additionally, the pair $(T, \prec)$ is called a temporal frame. For linear temporal logic in particular each point in
time $t$ as a unique successor that we will denote with $s(t)$, and usually the chosen temporal frame is $(\mathbb{N},
<)$.
\begin{definition}[LTL semantic]
  The truthness of a LTL formula $\phi$ at the time-point $t$ over the temporal model $M$, denoted as
  $M, t \vDash \phi$ is defined inductively as follows:
  \begin{align*}
    M, t &\vDash tt && \\
    M, t &\vDash p &&\text{ iff } p \in L(t) \\
    M, t &\vDash \neg\phi &&\text{ iff } M, t \nvDash \phi \\
    M, t &\vDash \phi_1 \lor \phi_2 &&\text{ iff } M, t \vDash \phi_1 \text{ or } M, t \vDash \phi_2 \\
    M, t &\vDash \next{\phi} &&\text{ iff } M, s(t) \vDash \phi \\
    M, t &\vDash \until{\phi_1}{\phi_2} &&\text{ iff } \exists t' \geq t \ldotp M, t' \vDash \phi_2 \text{ and } \forall t
      \leq t'' < t' \ldotp M, t'' \vDash \phi_1 \\
  \end{align*}
\end{definition}

\section{First-Order Extension}

\begin{definition}[First-Order Linear Temporal Logic]
  Let $\mathcal{P}$ be a set of predicates each with a specific arity, and let $X$ be a denumerable set of variables.
  The set $\mathcal{F}_{FO}$ of formulae for First-Order Linear Temporal Logic is the set generated by the following grammar:
  \[
    \phi \Coloneqq tt \;|\; P(x_1, \ldots, x_n)
                      \;|\; \neg\phi
                      \;|\; \phi \lor \phi
                      \;|\; \next{\phi}
                      \;|\; \until{\phi}{\phi}
                      \;|\; \existss{x}{\phi}
  \]
  where $P \in \mathcal{P}$ and $x_1, \ldots, x_n \in X$ are individual variables that match the arity required by $P$.
\end{definition}
Universal quantification can be modeled with the translation from the existential quantifier and negation:
\[
  \foralls{x}{\phi} \equiv \neg\existss{x}{\neg\phi}
\]

Example of properties can be modeled with the first-order extension are $\existss{x}{G(\text{channel}(x) \land
\text{open}(x))}$ with the intuitive meaning that there must always exists at least one open channel; or
$\foralls{x}{G(\text{channel}(x) \to (\until{\text{open}(x)}{(\neg\existss{y}{\text{message}(y) \land \text{pending}(x,
y)})}))}$
which intuitively means that channels must always remain open until there are no more pending messages on that channel.

For quantified LTL we need to extend the temporal frames with the domain of values that makes sense to talk about at
each points in time.

\begin{definition}[Kripke frame]
  A \emph{Kripke frame} $M$ is a quadruple $(T, \prec, D, d)$ where $T$ is a set of time-points, $\prec$ is an
  accessibility
  relations over $T$, $D$ is a function assigning to each point in time $t$ a non-empty set $D(t)$ s.t. if $t \prec t'$
  then $D(t) \subseteq D(t')$, $d$ is a function assigning to each point in time $t$ a set $d(t) \subseteq D(t)$.
\end{definition}
Intuitively, the so-called outer domains $D(t)$ represent the collection of object that can be referenced at the point
in time $t$, conversely the so-called inner domains $d(t)$ represent the collection of object actually existing the
point in time $t$. An assignement function $\sigma$ is a function from the set of variables $X$ to the outer-domain
$D(t)$ of a given point in time $t$. Finally, we need an interpretation function $I$ such that $I(P, t)$ assign for each
predicate constant $P \in \mathcal{P}$ and point in time $t$ a subset of $D^n(t)$ where $n$ is the arity of $P$, and
$I(x, t) = \sigma(x)$ where $x \in X$ and $\sigma$ is a suitable assignment that is said to be inducing $I$, noted as
$I^\sigma$.

\begin{definition}[FO-LTL semantic]
  The truthness of a FO-LTL formula $\phi$ at the point in time $t$ over the Kripke model $M$ and the induced
  interpretation $I^\sigma$, denoted as $M, t, I^\sigma \vDash \phi$ is defined inductively as follows:
  \begin{align*}
    M, t, I^\sigma &\vDash tt && \\
    M, t, I^\sigma &\vDash P(x_1, \ldots, x_n) &&\text{ iff } (\sigma(x_1), \ldots, \sigma(x_2)) \in I^\sigma(P, t) \\
    M, t, I^\sigma &\vDash \neg\phi &&\text{ iff } M, t, I^\sigma \nvDash \phi \\
    M, t, I^\sigma &\vDash \phi_1 \lor \phi_2 &&\text{ iff } M, t, I^\sigma \vDash \phi_1 \text{ or } M, t, I^\sigma
      \vDash \phi_2 \\
    M, t, I^\sigma &\vDash \next{\phi} &&\text{ iff } M, s(t), I^\sigma \vDash \phi \\
    M, t, I^\sigma &\vDash \until{\phi_1}{\phi_2} &&\text{ iff } \exists t' \geq t \ldotp M, t', I^\sigma \vDash \phi_2
      \text{ and } \forall t \leq t'' < t' \ldotp M, t'', I^\sigma \vDash \phi_1 \\
    M, t, I^\sigma &\vDash \existss{x}{\phi} &&\text{ iff } \existss{v \in d(t)}{M, t, I^{\sigma[v/x]} \vDash \phi} \\
  \end{align*}
\end{definition}

\section{Trans-World Identity}
Kripke-frames requires that the so-called outer domains $D(t)$ are always increasing with time, i.e. if $t \prec t'$
then $D(t) \subseteq D(t')$. This condition is required to evaluate temporal operators, otherwise it would not be
possible to denote a variable $x$ in the future.
However, by imposing such condition we are identifying object \emph{a priori} and universally, irrespective of
time. This problem is called \emph{trans-world identity} and extensive literature about it has been produced in the last
half-century, be either philosophycal questions and possible solutions with Kripke-style semantics.
Even with the additional constraints on object domains, there are still other issues with Kripke-style semantics and the
system we are trying to model. Assume we are trying to model resource allocation in a complex system. Let $i$ be a
resource. Due to the outer domain condition, we can identify the exact point in time where the resource $i$ is
allocated, i.e. $i \in D(t)$ for some time point $t$. Note that the resource can be allocated, but still can be unused
for a potentially infinite amount of time, since it may not belong to any inner domain, not even $d(t)$. However, at the
same time, we cannot deallocate the resource $i$ since it must always be referentiable at future point in times. This
can be solved with infinite outer domains to ensure uniqueness or by restricting the class of admissible evolutions, but
such solutions tend to hamper usability. Another desiderable behaviour is merging, for example while
modeling memory allocations, we may want to merge two memory allocated segments into a single segment and treat it as a
single resource.
In the next chapter we will explore an alternative approach based on \emph{counterpart relations}, which rejects the
possibility of universally identifying objects among possible worlds.
