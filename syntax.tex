\section{Syntax}

\begin{definition}[First-Order Linear Temporal Logic]
Let $\Sigma$ be a (multi-sorted) signature and $X$ a denumerable set of variables typed over $S_\Sigma$. The set
$\mathcal{F}_\Sigma$ of formulae for First-Order Linear Temporal Logic is the set generated by the following grammar:
\[
  \phi \Coloneqq tt \;|\; x_\tau = \epsilon_\tau
                    \;|\; \neg\phi
                    \;|\; \phi \lor \phi
                    \;|\; \existss{x_\tau}{\phi}
                    \;|\; \next{\phi}
                    \;|\; \until{\phi}{\phi}
\]
where $\epsilon$ is a term in $\terms{\Sigma}{X}$, $\exists x_\tau$ ranges over variables of sort $\tau \in S_\Sigma$,
$O$ is a unary operator which states that $\phi$ must hold at the next step and $U$ is a binary operator which states
that the first formula must hold until the second formula holds at some point in the current or next steps.
\end{definition}

Classical propositional logic and other temporal operators can be derived as for LTL, with the addition of the trivially
derivable universal quantifier, $\foralls{x_\tau}{\phi} \equiv \neg\existss{x_\tau}{\neg\phi}$.
