Modal and temporal logics have seen widespread usage in the industry as languages for expressing property about the
evolution of complex systems. Various models based on graphs and automaton have model checkers developed upon.
Most of them are based upon propositional variation of modal and temporal logics, however some reasearcher started
working on first-order or second-order variants. These logics suffer mainly two issues: they are in general undecidable
and the semantics evaluation model suffer various issue with trans-world identity.
Counterpart relations solve most of these issues and recent works successfully use counterpart relations with monadic
second-order modal logics to obtain a decidable calculus in finite domains and a companion model checker prototype.
However they are based upon fixpoints and modal operators which are not really intuitive. The semantics evaluation
function is complex and hard to follow by hand, and work is still needed regarding completeness and inference systems.
The aim of this work is to develop a a counterpart semantics model for first-order/second-order linear temporal logic,
and show that we can obtain a far simpler and understandable semantics but being still able to model plenty of
properties and with an axiomatic model and an inference system.

We will first give an introduction to linear temporal logic and the issue of extending it to the first or
second-order. Then we will present the mathematical tools used in our semantics, many-sorted algebras and counterpart
models. We will defined the counterpart semantics for first-order/second-order linear temporal logic with many examples
of properties and how to compute the semantics. Finally we will show some results on axioms and inference rules.

The main running example will be about multi-process communication over channels, modeled via graphs with processes as
nodes and channels as edges. Some of the properties we will be able to express with our calculus are:
\begin{itemize}
  \item Will all the channel be closed at the end of the execution?
  \item Which channels will be merged during execution?
  \item Are there isolated process at any point in time?
  \item Do the channels connection change at each step?
  \item Are new channels created during the execution?
\end{itemize}
